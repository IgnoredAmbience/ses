\documentclass[a4paper]{report}
\usepackage{fullpage, hyperref}
\usepackage{separationlogic}
\usepackage[english]{babel}
\usepackage[fixlanguage]{babelbib}

\title{Interim Report}
\author{Thomas Wood}

\begin{document}
\maketitle

\chapter{Introduction}
%Citations required
  In recent years with the development of the web, JavaScript has become an
  increasingly-intensively used language. At its invention and for a
  considerable length of time, JavaScript was only used for simple tasks such as
  form validation
  
  The modern nature of web-programming
  is that multiple different programs from different sources may be loaded and
  executed in a single environment. This combination of different code poses
  some interesting problems regarding code-correctness and security. JavaScript
  provides no means to isolate different modules of code from one another, all
  data stored in memory is accessible to any code loaded on the page, which must
  be implicitly trusted.

  Some attempts have been made to solve this problem, examples include AdSafe
  \cite{AdSafe}, and Google's Caja language \cite{miller2008caja}. SES aims to
  integrate some of these features into the core JavaScript language, to promote
  safer web programming.

  Secure ECMAScript is a proposed extension to the ECMAScript 5 specification
  that allows programs to be executed within a restricted context. Restricted
  execution will ideally permit fine-grained control over the permissions
  untrusted executing code has. It is possible for unrestricted code to interact
  with code under restricted execution, it is in this situation that it is
  envisaged that lapses in the encapsulation could occur inadvertantly.
  % could still expand details on mixed execution - examples?

  The language's structure already poses some interesting problems for
  researchers producing formal models of programming languages due to its memory
  structure.  A formal logic for ECMAScript 3 has been recently developed by
  researchers at Imperial College. This project aims to extend this program
  logic to effectively model the restricted execution support proposed by the
  Secure ECMAScript language specification. This will permit proofs asserting
  that the language constructs are secure, and that programs written using the
  restricted execution construct do not accidentally leak private data or
  permissions to untrusted code, especially during interactions between two
  contexts.

  The formal logic for ECMAScript 3 is based upon Separation Logic, an extension
  of Hoare Logic which permits efficient reasoning about data contained within
  heaps by splitting the heap into smaller discrete portions. However, to reason
  about restricted execution, we need to reason about the \emph{entire}
  heap. The first aim of the project is to extend Separation logic with a
  backpointer operator to allow reasoning about which objects point to an
  object of interest.

  Using this new operator, we then aim to produce semantics for the restricted
  \texttt{eval()} operation.

  Finally, we prove the correctness of a contract host algorithm
  \cite{contract-host-algo} which permits
  two mutually suspicious parties to execute an agreed contract on a mutually
  trusted third-party.

% What is the problem?

% Why is it interesting?

% What's the idea to solve it?

\chapter{Background}
\section{Hoare Reasoning}
  This style of reasoning about the functionality and correctness of computer
  programs was first presented in his 1969 paper ``An Axiomatic Basis for
  Computer Programming''\cite{Hoare1969Axiom}, the principles presented were
  based on earlier work by Floyd\cite{floyd1967assigning} for flowchart
  representations of programs, and by others in other branches of mathematics;
  Hoare developed the style of the reasoning to fit textual representation of
  programs.

  Hoare's reasoning depends on the basis that commands used in programming
  languages should be well-defined and thus the consequences of a computer
  program should be determinable directly from the program's text by deductive
  means. Program proofs are deduced through the application of inference rules
  to sets of axioms.

  The main contribution of the paper was to introduce the notation, now known as
  a Hoare triple:
    \[ \{P\} C \{Q\} \]
  $P$ and $Q$ are logical assertions and $C$ is some command or program. The
  triple is interpreted by Hoare as ``if the assertion $P$ is true before
  initiation of a program $C$, then the assertion $Q$ will be true on its
  completion.'' Other interpretations of triples are possible, as will be shown
  later.

  The axioms selected for use in the reasoning alter the semantics of the system
  being reasoned about. Hoare exemplifies this by presenting axioms that apply
  to arithmetic under all sets of integers, followed by three alternative axioms
  that describe different semantics for overflow in sets of finite integers.

  A number of axioms and inference rules are universal or common to nearly all
  imperitive languages, Hoare presents four of these:
  \begin{display}{}
    \stateaxiom{(Assign)}
    {\tr{P[f/x]}{x := f}{P}} \qquad

    \staterule{(Consequence)}
    {\tr{P'}{C}{Q'} \quad P \impl P' \quad Q' \impl Q}
    {\tr{P}{C}{Q}} \\
    \\
    \staterule{(Iteration)}
    {P \quad \tr{B}{S}{P}}
    {\tr{P}{\textbf{while } B \textbf{ do } S}{\lnot B \land P}} \qquad

    \staterule{(Composition)}
    {\tr{P}{C_1}{Q} \quad \tr{Q}{C_2}{R}}
    {\tr{P}{C_1; C_2}{R}}
  \end{display}

  The consequence and composition rules are straight-forward. The iteration rule
  is also reasonably simple once it is noted that the $P$ term is the loop
  invariant. The assignment axiom is the most subtle of the set when not taken
  in the context of a proof. The axiom asserts that the post-condition will hold
  if the \emph{precondition} if the precondition has the variable replaced by
  the assigned value. This axiom makes more sense if it is considered to be used
  whilst reasoning backwards. For example, to prove $\tr{\true}{r := x; q :=
  0}{x = r + y \times q}$:

  \begin{tabular}{ll}
    $\{\true \impl x = x + y \times 0 \}$ & Axiom \\
    $\tr{x = x + y \times 0}{r:=x}{x = r + y \times 0}$ & Assignment \\
    $\tr{x = r + y \times 0}{q:=0}{x = r + y \times q}$ & Assignment \\
    $\tr{\true}{r:=x}{x = r + y \times 0}$ & Consequence \\
    $\tr{\true}{r:=x; q:=0}{x = r + y \times q}$ & Composition\\
  \end{tabular}

  It is noted that proofs in this format are long and tedious, but it is
  possible to derive more general rules from the given ones which would somewhat
  reduce the length of these proofs. This is now the generally accepted method
  of presenting proofs in Hoare's system, so the above proof could also be
  presented as follows:

  \begin{tabular}{l}
    $\{\true\}$ \\
    $\{x = x + y \times 0\}$ \\
    $\ r:=x$ \\
    $\{x = r + y \times 0\}$ \\
    $\ q:=0$ \\
    $\{x = r + y \times q\}$
  \end{tabular}

  Indeed, in his 1971 paper~\cite{Hoare1971proof}, Hoare presents a proof in
  such a style for the non-trivial \textsc{Find} algorithm.

  %Hoare closes his paper by reinforcing the need for formal proof and
  %specification of computer programs and languages.

\section{Separation Logic}

  Whilst Hoare's methodology is suitable for reasoning about programs which only
  operate within the stack, it is not suitable for reasoning about programs that
  use the heap. This is because that data structures within the heap are more
  prone to being shared in some manner (for example, multiple pointers to the
  same heap location, or even arbitrary pointer arithmetic).
 
  Reasoning about these shared data structures has been attempted by a number of
  researchers, but no solution was particularly adequate. The main issue with
  verification of shared data structures was that a global view of the data, and
  attempts to axiomatize use of the heap would result in assertions
  that rapidly grew in complexity, both in terms of length, but also from
  universal quantification, for the need to assert that the rest of the heap was
  not modfified along with the location of interest.

  Burstall considered a local approach to this reasoning in 1972
  \cite{burstall1972some}, but the approach featured little further research and
  was unable to reason about data structures such as doubly-linked lists
  \cite{reynolds2000intuitionistic}. Reynolds began further research into the
  use of a spatial logic to permit local reasoning about the heap and in
  collaboration with O'Hearn, Ishtiaq, Yang produced Separation Logic, a
  combination of the spatial Bunched Implication logic with Hoare's methods for
  reasoning about programs
  \cite{Ishtiaq2001BI,OHearn2001Local,reynolds2000intuitionistic,Reynolds2002Separation}.

  \subsection{Semantics of Separating Operators}
  
  Separation Logic introduces the separating conjunction operator, $P \sep Q$,
  which can be thought of as splitting the heap into two disjoint portions, in
  which $P$ holds in one and $Q$ the other. To denote the contents of the
  heap, the $\mapsto$ operator is also introduced to relate heap locations with
  values, $(x \mapsto 4, y)$ denotes that the pair of the value $4$ and a pointer
  to the heap location $y$ is stored in the heap location $x$.

  The separating conjunction only enforces disjointedness of the storage
  locations of values, it is permitted to point to other heap locations from
  within the heap. For example $(x \mapsto 4, y) \sep (y \mapsto 3, x)$ is
  permissable and required to be able to build complex data structures.

  % diagram of (x -> 4, y) * (y -> 3, x) ?

  $x \mapsto -$ is shorthand meaning that there is some value stored at heap
  location $x$.

  The disjointedness enforced by the $\sep$ operator provides a very powerful
  means of reasoning about operations made to the heap. Take for example the
  assertion $x \mapsto 1 \sep P$ and the command $x := 2$, since we have $x$
  explicitly defined on the left of the $\sep$, we know that it cannot occur
  within $P$. We can therefore safely conclude that $P$ is not modified and that
  the postcondition must be $x \mapsto 2 \sep P$.

  A separating implication operator, $\wand$, also exists. The assertion $((x
  \mapsto 7, 5) \wand P)$ has the meaning that if $x$ is updated to contain the
  value $7, 5$, then $P$ will be true. This operator can thus, in effect, be
  used to generate the weakest precondition for a command given a desired
  postcondition.

  \begin{display}{Assertion satisfaction rules: $s,h \satisfies P$}
    %$s,h \satisfies E \mapsto E_1, E_2 \qquad \iff \{\eval{E} s \} = \domain(h)
    %  \land h(\eval{E}) = \langle \eval{E_1}, \eval{E_2} \rangle \\
    $s,h \satisfies \emp \qquad \iff h = \emptylist$ \\
    $s,h \satisfies P \sep Q \qquad \iff \exists h_0, h_1.h_0 \disj h_1, h_0 \cdot
    h_1 = h, s, h_0 \satisfies P \land s,h_1 \satisfies Q$ \\
    $s,h \satisfies (E \mapsto E_1, E_2) \iff dom(h) = \{[E]_s\} \land h([E]_s)
    = ([E_1]_s, [E_2]_s)$ \\
    ...
  \end{display}

  \subsection{Program Reasoning}

  Just as Hoare's reasoning was built atop of classical logic, we can extend
  Hoare's reasoning to the spatial logic including the separating conjunction,
  the combined system is known as Separation Logic.

  The first concept to translate is that of the Hoare triple, $\tr{P}{C}{Q}$.
  For the remainder of this report, we will consider Separation Logic to use
  \emph{fault-avoiding} semantics for the triple. Under this semantics, then the
  requirements for $C$ are strengthened, such that ``if $C$ is executed in a
  state satisfying $P$, then it will \emph{not fault}, and if it terminates it
  will do so in a state satisfying $Q$''.

  Separation Logic extends the standard Hoare inference rules and axioms, the
  rules for Consequence, Composition, and Simple Assignment are used without
  modification.

  The obvious new axioms to introduce are those for heap access and
  modification, the following set are taken from \cite{OHearn2001Local}.

  \begin{display}{Axioms for heap access}
    $\begin{array}{ll}
      \stateaxiom{(Update)}{\tr{x \mapsto -} {x := v} {x \mapsto v}} &
      \stateaxiom{(Delete)}{\tr{x \mapsto -} {\js{delete}(x)} {\emp}} \\
      \stateaxiom{(Allocate)}{\tr{x \doteq v'} {x := v} {x \mapsto v[v'/x]}} &
      \stateaxiom{(Lookup)}{\tr{x \mapsto v} {y := [x]} {y = v \land x \mapsto v}} \\
    \end{array}$
  \end{display}

  These axioms may be expressed in other ways, for example, in a form suitable
  for backwards reasoning, or with use of existential quantifiers.

  There is one important new inference rule, the frame rule: \\

  $
    \staterule{(Frame Rule)}
    {\tr{P}{C}{Q}}
    {\tr{P \sep R}{C}{Q \sep R}}
    {\mathrm{modifies}(C) \disj \mathrm{free}(R)}
  $\\

  This rule is the prime reason why Separation Logic is suitable for local
  reasoning. It allows assertions to be split down to the \emph{footprint} of
  the command, those heap locations that the command touches. The side condition
  guarantees that the command only touches the part of the assertion which is
  kept, it guarantees that nothing in $R$ can be modified by $C$.

\section{JavaScript}

  JavaScript is a dynamically-typed, prototype-oriented language. The
  early development of the language was haphazard with rival implementations
  produced by Netscape and Microsoft for their browsers. The language's
  specification was later standardised as ECMA-262 and named ECMAScript.
  The specification is unusual with respect to it permitting many
  implementation-specific extensions to the language. Until recently, many
  JavaScript implementations did not conform to the ECMAScript specification.

  Although originally targeted to be a programming language that was easy to
  learn and suitable for non-programmers to use, the language's variable scoping
  rules and semantics of the \js{this} keyword and \js{with} syntax are unusual
  and differ from most programmers' intuitions. In addition, the ability to
  redefine, or \emph{monkey-patch} many primitives that define the semantics of
  the language means that reasoning about programs can become impossible if
  some unknown code can execute within the same execution environment as the
  program under study.

  There have been several efforts to formally specify the semantics of the
  JavaScript language, notable mentions include Maffeis, Mitchell and Taly's
  production of an operational semantics for a wide range of JavaScript
  implementations \cite{maffeis-jsopsem}. The remainder of this section will
  focus on Gardner, Maffeis, Smith's work \cite{gms-popl} into producing a
  Program Logic for JavaScript which provides a language in which to reason
  about JavaScript programs.

  A unique feature of the
  language is that the entire state is stored on the heap in a structure that
  loosely resembles the variable store in conventional programming languages. It
  thus follows that Separation Logic is likely to be a useful basis for the
  JavaScript program logic. However,sizable modifications were required to
  accurately model the variable store and scoping rules.

  %diagram

  The variable store structure consists of an ordered scope chain,
  somewhat alike to
  stack frames in other languages. Each scope entry in the chain is an object,
  on which fields may be defined, as well as a pointer to the object's
  prototype. To perform a variable lookup, the first object in the scope chain has
  its fields, and the fields of its prototypes inspected for a name match, before
  progressing to the next entry in the scope chain to repeat the search.

  To perform a write to a variable, a similar lookup is performed, except the
  field is modified on the scope object for the prototype chain in which the
  variable was found -- this is so that the prototypical variable is overridden,
  but not overwritten.

  Once again, Hoare-style reasoning is adapted to suit the extensions of the
  logic. The fault-avoiding semantics for the Hoare triple is maintained. The
  basic set of Separation Logic axioms and inference rules (excluding those
  referring directly to the C-style heap implementation) are used. One notable
  difference is that the Frame rule loses its side condition as a result of all
  variables now being on the heap -- the footprint of $C$ will be expressed in
  the $P$ and $Q$ terms, so is necessarily disjoint from the framed $R$ portion of
  the rule.

\chapter{Project Plan}
% Project timetable, milestones, fall-back plans
% Extensions?
% Include parts already addressed - some idea of progress
  The project is split into broadly 4 sections, a formal description of Secure
  ECMAScript, an investigation into the implementation of backpointers in
  Separation Logic, using backpointers to produce a program logic for SES, and
  finally producing proofs of algorithms implemented in SES.

  \section{Formally Describe Secure ECMAScript}
  To date, no complete specification of ECMAScript exists, the language is in an
  embryonic state. A draft specification \cite{ses-spec} based upon ECMAScript 5.1 exists but,
  as with all ECMAScript specification documents, is not written formally but in
  English. This can lead to possible ambiguities of the specification and
  doesn't lend itself particularly well towards formal proof of program
  correctness. This specification is also not to be considered canonical, as it
  has not been updated for several years, but the language has developed since.

  The primary sources for information on the language should therefore be
  intuitions recieved verbally from the creator, Mark Miller, via discussions
  with my supervisor, Gareth Smith; implementations of algorithms developed for
  SES \cite{contract-host-algo}; the SES library implementation \cite{ses-impl};
  and other reference documentation regarding features, and differences from the
  ECMAScript specification \cite{securable-es,ses-diffs,ses-caja-diffs}.

  This portion of the project aims to produce a formal specification of the
  language. The first step is the abstract syntax definition, followed by the
  English description of each of the commands, focusing on non-standard
  semantics. Finally, axioms and inference rules for each command should be
  specified.
  \\
  \emph{Target completion: middle of week 4.}

  \section{Investigation into Backpointers in Separation Logic}
  One of the aims of the program proof logic for SES is the ability to reason
  about the (non-)reachability of particular portions of the heap from
  restricted execution contexts. For this we need to combine Separation Logic's
  local reasoning methods with some means of global reasoning about the heap.
  Backpointers are a suggested method of achieving this aim.

  At the inception of the project, the concept of using backpointers in
  Separation Logic was considered to be novel. However, during the production of
  this report, a technical report \cite{KassiosKritikos12} discussing the
  implementation of backpointers in the program verification tool, Chalice, with
  the aim of proving a concurrent copy-on-write list algorithm.

  The aim of this portion of this project remains to investigate the possiblity
  of adding backpointers into Separation Logic, either by further formalising
  the implementation given in this new paper, or by considering alternative
  approaches to the problem.
  \\
  \emph{Target completion: end of week 6.}

  \section{Production of a full SES logic}
  This stage is to integrate the work into the specification of SES with the
  backpointer implementation to permit a descriptive logic to be able to reason
  about the desired properties of the language (heap reachability).
  \\
  \emph{Target completion: end of week 8.}

  \section{Proofs of SES Algorithms}
  This stage will attempt to prove the makeContractHost SES
  algorithm\cite{contract-host-algo}. This will
  require the functionality of a dependent library to be axiomatized. If time
  permits, further proofs for SES algorithms will be attempted.
  \\
  \emph{Target completion: end of term.}

\chapter{Evaluation Plan}
% What functionality needs to be demonstrated?
% How has the project extended the state of the art?
  Since the work is largely based upon \cite{gms-popl} a detailed critique of
  the work can be produced, it could include, but not be limited to a
  discussion of how suited the logic is for proof of JavaScript programs,
  suitability for extension. Any critiques made of the JavaScript logic should
  also be contrasted with the SES logic to pick out improvements or limitations,
  as appropriate.

  Since the language is fairly loosely defined, a concrete specification of the
  language is useful and is one of the items of work for the project. The
  language's inventor, Mark Miller, is available to discuss and review a
  formalised specification. Any work presented to him should be accessible to
  someone not versed in formal semantics and logics of languages, yet still be
  precise enough to be useful. A successful project would be to obtain a
  reasonable match between the formalism and Mark's intuitions of his language.

  Finally, being able to prove the desired security properties of the algorithms
  under study would validate that the SES program logic is capable of achieving
  its desired aim.

\bibliography{bibliography}
\bibliographystyle{babplain}

\end{document}
