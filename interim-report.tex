\documentclass[a4paper]{report}
\usepackage{fullpage, titling, amsmath, footnote, listings, hyperref}
%\usepackage{gds}
\makesavenoteenv{tabular}

\title{Interim Report}
\author{Thomas Wood}

\begin{document}
\maketitle

\begin{abstract}
  % Summary of entire project, including conclusions
  % About half a page
\end{abstract}

\chapter{Introduction}
  In recent years with the development of the web, JavaScript has become an
  increasingly-intensively used language. At its invention and for a
  considerable length of time, JavaScript was only used for simple tasks such as
  form validation
  
  The modern nature of web-programming

  is that multiple different programs from different sources may be loaded and
  executed in a single environment. This combination of different code poses some
  interesting problems regarding code-correctness and security. JavaScript
  provides no means to isolate different modules of code from one another, all
  data stored in memory is accessible to any code loaded on the page, which must
  be implicitly trusted.

  Secure ECMAScript is a proposed extension to the ECMAScript 5 specification
  that allows programs to be executed within a restricted context. Restricted
  execution will ideally permit fine-grained control over the permissions
  untrusted executing code has. It is possible for unrestricted code to interact
  with code under restricted execution, it is in this situation that it is
  envisaged that lapses in the encapsulation could occur inadvertantly.
  % could still expand details on mixed execution - examples?

  The language's structure already poses some interesting problems for researchers
  producing formal models of programming languages due to its memory structure.
  A formal logic for ECMAScript 3 has been recently developed by researchers at
  Imperial College. This project aims to extend this program logic to
  effectively model the restricted execution support proposed by the Secure
  ECMAScript language specification. This will permit proofs asserting that the
  language constructs are secure, and that programs written using the restricted
  execution construct do not accidentally leak private data or permissions to
  untrusted code, especially during interactions between two contexts.

  The formal logic for ECMAScript 3 is based upon Separation Logic, an extension
  of Hoare Logic which permits efficient reasoning about data contained within
  heaps by splitting the heap into smaller discrete portions. However, to reason
  about restricted execution, we need to reason about the \emph{entire}
  heap. The first aim of the project is to extend Separation logic with a
  backpointer operator to allow reasoning about which objects point to an
  object of interest.

  Using this new operator, we then aim to produce semantics for the restricted
  \texttt{eval()} operation.

  Finally, we prove the correctness of a contract host algorithm which permits
  two mutually suspicious parties to execute an agreed contract on a mutually
  trusted third-party.

% What is the problem?

% Why is it interesting?

% What's the idea to solve it?

\chapter{Background}
\section{Hoare Logic}
\section{Separation Logic}
\section{Backpointers in Separation Logic}

\chapter{Project Plan}
% Project timetable, milestones, fall-back plans
% Extensions?
% Include parts already addressed - some idea of progress

\chapter{Evaluation Plan}
% What functionality needs to be demonstrated?
% How has the project extended the state of the art?
% 

\end{document}
