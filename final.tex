\documentclass[a4paper,notitlepage]{report}
\usepackage[usenames,dvipsnames,svgnames,x11names]{xcolor}
\usepackage{amsmath}
\usepackage{amssymb}
\usepackage{stmaryrd}
\usepackage{wasysym}
\usepackage{xspace}
\usepackage{fullpage}
\usepackage{hyperref}
\usepackage{etoolbox}
\usepackage{separationlogic}
\usepackage[numbers]{natbib}
\usepackage[nottoc]{tocbibind} % add bibliography to toc
\usepackage{parskip}

\usepackage[sc]{mathpazo}
\linespread{1.05}         % Palatino needs more leading (space between lines)
\usepackage[T1]{fontenc}

\begin{document}
\begin{titlepage}
\begin{center}
  \textsc{\Large Imperial College London} \\[0.1cm]
  \textsc{\Large Department of Computing}

  \vfill

  % Still rough title
  {\Huge A Program Logic for Verification of Security Properties in Secure
  ECMAScript} \\[1cm]

  {\LARGE Thomas Wood} \\[1cm]

  {\large Supervisors:\\
  Dr. Gareth Smith \\
  Prof. Philippa Gardner}
  \\[2cm]

  \begin{abstract}
    \begin{center}
      We present an Operational Semantics of the Secure ECMAScript (SES) language.
      We extend Separation Logic with a backpointer operator to permit reasoning
      about reachability in the object graph whilst maintaining local reasoning.
      We define inference rules in the extended logic for SES. Finally, we prove
      the correctness of the Membrane design pattern.
    \end{center}
  \end{abstract}

  \vfill

  {\Large June 2013} \\[0.5cm]

  Submitted in partial fulfilment of the requirements for the\\
  MEng Degree in Computing of Imperial College London
\end{center}
\end{titlepage}

%\chapter*{Acknowledgements}
%  Thanks to...
%
%  Gareth, Philippa.
%
%  Mark Miller.
%
%  Housemates.
%
%  Lecturers this year.

\tableofcontents

\chapter{Background}
\section{Introduction}

% * Clear statement of what the project is about, nature and scope to be
%   understood by a lay reader
% * Summarise goals
% * Summarise background
% * Summarise relevance, main contributions
%   * Highlight sections of report for contributions
% * Explain motivation, identify issues to be addressed in remainder of report

  Historically, the JavaScript language was developed with very little intent
  for it to be used for much more than form validation on web pages.
  However, in recent years, web technologies have become a central part of
  day-to-day computer use, an ever growing number of applications now target the
  web as their platform of choice.
  This increase in popularity has forced the JavaScript language to mature
  considerably, through syntax, semantics, and standard library support.

  It is extremely common
  that multiple different programs from different sources may be loaded and
  executed in a single environment. This combination of different code poses
  some interesting problems regarding code-correctness and security. JavaScript
  provides no easy means to isolate different modules of code from one another,
  all data stored is potentially accessible to any code loaded on the page,
  which must therefore be implicitly trusted.

  Some attempts have been made to solve this problem by statically rewriting
  source code, or more recently by making clever use of
  a mixture of JavaScript's deprecated and modern language features.
  Examples of such attempts include AdSafe~\cite{AdSafe}, FBJS, and
  Google Caja~\cite{miller2008caja}.
  All of these have suffered from some security
  flaws in their histories as a
  result of insufficient understanding of the JavaScript language or knowledge
  of one of the myriad of differences that \emph{legitimately} exist between
  common JavaScript implementations~
  \cite{maffeis2009jsisolation, maffeis2010object-cap, ses-semantics}.
  It is therefore beneficial to introduce more security features to the core
  language specification.

  Secure ECMAScript (SES) is a proposed revision of the ECMAScript (ES)
  specification that introduces changes to the language to permit programs to be
  executed within a restricted context, safe from the possibility of unintended
  leak.
  Restricted execution permits fine-grained control over the permissions that
  untrusted executing code has by restricting the set of accessable references
  to other objects.
  This is a particularly powerful method as it is possible for unrestricted code
  to interact with code under restricted execution, it is in this situation that
  it is envisaged that lapses in the encapsulation could inadvertantly occur due
  to programmer error.

  \FIXME{EDIT INTO SHAPE}
  It is useful to be able to verify that a program is correct according to a
  given specification.

  Hence need a logic to express the program states and the specification.

  Up to now, we've got the logic to express program states and can prove
  functional properties of programs.

  However, we're unable to express the security properties in the existing
  separation logic system - we need to be able to reason about the entire heap,
  separation logic forces that we only reason about a small part of the heap
  that a given instruction touches. We could expand the footprint of
  instructions, but this would cause separation logic to quickly lose its
  advantages over normal boolean logics.

  It is useful to be able to reason about security properties at the same time
  as functional properties as the function of a program directly impacts which
  security properties hold.


  Project aims:

  Semantics of SES - to understand the language, differences from JavaScript, in
  style of previous ES3 semantics for direct comparison.

  Add bp to Separation Logic.

  Produce axiomatic rules for SES instructions including bp assertions. Again,
  direct comparison to ES3 logic and Operational Semantics.

  Prove some fundamental programming patterns used in objcap world (Caretaker,
  Membrane)

  Opted not to discuss decidability of logic, nor to prove soundness against
  operational semantics, nor completeness... Basically nothing very formal
  \FIXME{}

% TODO: Consider following for inclusion in introduction

% The language's structure already poses some interesting problems for
% researchers producing formal models of programming languages due to its memory
% structure.  A formal logic for ECMAScript 3 has been recently developed by
% researchers at Imperial College. This project aims to extend this program
% logic to effectively model the restricted execution support proposed by the
% Secure ECMAScript language specification. This will permit proofs asserting
% that the language constructs are secure, and that programs written using the
% restricted execution construct do not accidentally leak private data or
% permissions to untrusted code, especially during interactions between two
% contexts.

% The formal logic for ECMAScript 3 is based upon Separation Logic, an extension
% of Hoare Logic which permits efficient reasoning about data contained within
% heaps by splitting the heap into smaller discrete portions. However, to reason
% about restricted execution, we need to reason about the \emph{entire}
% heap. The first aim of the project is to extend Separation logic with a
% backpointer operator to allow reasoning about which objects point to an
% object of interest.

\section{Hoare Logic}
  This style of reasoning about the functionality and correctness of computer
  programs was first presented in his 1969 paper ``An Axiomatic Basis for
  Computer Programming''\cite{Hoare1969Axiom}, the principles presented were
  based on earlier work by Floyd\cite{floyd1967assigning} for flowchart
  representations of programs, and by others in other branches of mathematics;
  Hoare developed the style of the reasoning to fit textual representation of
  programs.

  Hoare's reasoning depends on the basis that commands used in programming
  languages should be well-defined and thus the consequences of a computer
  program should be determinable directly from the program's text by deductive
  means. Program proofs are deduced through the application of inference rules
  to sets of axioms.

  The main contribution of the paper was to introduce the notation, now known as
  a Hoare triple:
    \[ \{P\} C \{Q\} \]
  $P$ and $Q$ are logical assertions and $C$ is some command or program. The
  triple is interpreted by Hoare as ``if the assertion $P$ is true before
  initiation of a program $C$, then the assertion $Q$ will be true on its
  completion.'' Other interpretations of triples are possible, as will be shown
  later.

  The axioms selected for use in the reasoning alter the semantics of the system
  being reasoned about. Hoare exemplifies this by presenting axioms that apply
  to arithmetic under all sets of integers, followed by three alternative axioms
  that describe different semantics for overflow in sets of finite integers.

  A number of axioms and inference rules are universal or common to nearly all
  imperitive languages, Hoare presents four of these:
  \begin{display}{}
    \stateaxiom{(Assign)}
    {\tr{P[f/x]}{x := f}{P}} \qquad

    \staterule{(Consequence)}
    {\tr{P'}{C}{Q'} \quad P \impl P' \quad Q' \impl Q}
    {\tr{P}{C}{Q}} \\
    \\
    \staterule{(Iteration)}
    {P \quad \tr{B}{S}{P}}
    {\tr{P}{\textbf{while } B \textbf{ do } S}{\lnot B \land P}} \qquad

    \staterule{(Composition)}
    {\tr{P}{C_1}{Q} \quad \tr{Q}{C_2}{R}}
    {\tr{P}{C_1; C_2}{R}}
  \end{display}

  The consequence and composition rules are straight-forward. The iteration rule
  is also reasonably simple once it is noted that the $P$ term is the loop
  invariant. The assignment axiom is the most subtle of the set when not taken
  in the context of a proof. The axiom asserts that the post-condition will hold
  if the \emph{precondition} has the variable replaced by
  the assigned value. This axiom makes more sense if it is considered to be used
  whilst reasoning backwards. For example, to prove $\tr{\true}{r := x; q :=
  0}{x = r + y \times q}$:

  \begin{tabular}{ll}
    $\{\true \impl x = x + y \times 0 \}$ & Axiom \\
    $\tr{x = x + y \times 0}{r:=x}{x = r + y \times 0}$ & Assignment \\
    $\tr{x = r + y \times 0}{q:=0}{x = r + y \times q}$ & Assignment \\
    $\tr{\true}{r:=x}{x = r + y \times 0}$ & Consequence \\
    $\tr{\true}{r:=x; q:=0}{x = r + y \times q}$ & Composition\\
  \end{tabular}

  It is noted that proofs in this format are long and tedious, but it is
  possible to derive more general rules from the given ones \gds{are they really more general? Perhaps ``more palatable''?} which would somewhat
  reduce the length of these proofs. This is now the generally accepted method
  of presenting proofs in Hoare's system, so the above proof could also be
  presented as follows:

  \begin{tabular}{l}
    $\{\true\}$ \\
    $\{x = x + y \times 0\}$ \\
    $\ r:=x$ \\
    $\{x = r + y \times 0\}$ \\
    $\ q:=0$ \\
    $\{x = r + y \times q\}$
  \end{tabular}

  Indeed, in his 1971 paper~\cite{Hoare1971proof}, Hoare presents a proof in
  such a style for the non-trivial \textsc{Find} algorithm.

  %Hoare closes his paper by reinforcing the need for formal proof and
  %specification of computer programs and languages.

\section{Separation Logic}

  Whilst Hoare's methodology is suitable for reasoning about programs which only
  operate within the stack, it is not suitable for reasoning about programs that
  use the heap. This is because that data structures within the heap are more
  prone to being shared in some manner (for example, multiple pointers to the
  same heap location, or even arbitrary pointer arithmetic).

  Reasoning about these shared data structures using conventional logics has been attempted by a number of
  researchers, but no solution was particularly adequate. The main issue with
  verification of shared data structures was that a global view of the data, and
  attempts to axiomatize use of the heap would result in assertions
  that rapidly grew in complexity, both in terms of length, but also from
  universal quantification, for the need to assert that the rest of the heap was
  not modfified along with the location of interest.

  Burstall considered a local approach to this reasoning in 1972
  \cite{burstall1972some}, but the approach featured little further research and
  was unable to reason about data structures such as doubly-linked lists
  \cite{reynolds2000intuitionistic}. Reynolds began further research into the
  use of a spatial logic to permit local reasoning about the heap and in
  collaboration with O'Hearn, Ishtiaq and Yang produced Separation Logic, a
  combination of the spatial Bunched Implication logic with Hoare's methods for
  reasoning about programs
  \cite{Ishtiaq2001BI,OHearn2001Local,reynolds2000intuitionistic,Reynolds2002Separation}.

  \subsection{Semantics of Separating Operators}

  Separation Logic introduces the separating conjunction operator, $P \sep Q$,
  which can be thought of as splitting the heap into two disjoint portions, in
  which $P$ holds in one and $Q$ the other. To denote the contents of the
  heap, the $\mapsto$ operator is also introduced to relate heap locations with
  values, $(x \mapsto 4, y)$ denotes that the pair of the value $4$ and a pointer
  to the heap location $y$ is stored in the heap location $x$.

  The separating conjunction only enforces disjointedness of the storage
  locations of values, it is permitted to point to other heap locations from
  within the heap. For example $(x \mapsto 4, y) \sep (y \mapsto 3, x)$ is
  permissable and required to be able to build complex data structures.

  % diagram of (x -> 4, y) * (y -> 3, x) ?

  $x \mapsto -$ is shorthand meaning that there is some value stored at heap
  location $x$.

  The disjointedness enforced by the $\sep$ operator provides a very powerful
  means of reasoning about operations made to the heap. Take for example the
  assertion $x \mapsto 1 \sep P$ and the command $x := 2$, since we have $x$
  explicitly defined on the left of the $\sep$, we know that it cannot occur
  within $P$. We can therefore safely conclude that $P$ is not modified and that
  the postcondition must be $x \mapsto 2 \sep P$.

  A counterpart for implication also exists,
  the separating implication operator, $\wand$ (commonly known as the
  \emph{magic wand}). The assertion $((x
  \mapsto 7) \wand P)$ has the meaning that if $x$ is updated to contain the
  value $7$, then $P$ will be true. So: $(x \mapsto 7) \sep ((x \mapsto 7) \wand
  P) \impl P$.

  This operator can thus, in effect, be
  used to generate the weakest precondition for a command given a desired
  postcondition. This is often used for backwards reasoning, where a
  postcondition for a command is known, but the precondition is not.

  \subsection{Program Reasoning}

  Just as Hoare's logic was built atop of classical logic, we can extend
  Hoare's logic to the spatial logic including the separating conjunction,
  the combined system is known as Separation Logic.

  The first concept to translate is that of the Hoare triple, $\tr{P}{C}{Q}$.
  For the remainder of this report, we will consider Separation Logic to use
  \emph{fault-avoiding} semantics for the triple. Under these semantics, then the
  requirements for $C$ are strengthened, such that ``if $C$ is executed in a
  state satisfying $P$, then it will \emph{not fault}, and if it terminates it
  will do so in a state satisfying $Q$''.

  Separation Logic extends the standard Hoare inference rules and axioms, the
  rules for Consequence, Composition, and Simple Assignment are used without
  modification.

  The obvious new axioms to introduce are those for heap access and
  modification, the following set are taken from \cite{OHearn2001Local}.

  \begin{display}{Axioms for heap access}
    $\begin{array}{ll}
      \stateaxiom{(Update)}{\tr{x \mapsto -} {x := v} {x \mapsto v}} &
      \stateaxiom{(Delete)}{\tr{x \mapsto -} {\js{delete}(x)} {\emp}} \\
      \stateaxiom{(Allocate)}{\tr{x \doteq v'} {x := v} {x \mapsto v[v'/x]}} &
      \stateaxiom{(Lookup)}{\tr{x \mapsto v} {y := [x]} {y = v \land x \mapsto v}} \\
    \end{array}$
  \end{display}
\gds{I'm used to seeing Allocate given as $\tr{\emp} {x := new()} {\exists L \st x=L \land L \mapsto\_}$. The $\exists$ is important, because otherwise we can use the frame rule to ``prove'' that we know already something about the allocated address.}
  These axioms may be expressed in other ways, for example, in a form suitable
  for backwards reasoning, or with use of existential quantifiers.

  There is one important new inference rule, the frame rule: \\

  $
    \staterule{(Frame Rule)}
    {\tr{P}{C}{Q}}
    {\tr{P \sep R}{C}{Q \sep R}}
    {\mathrm{modifies}(C) \disj \mathrm{free}(R)}
  $\\

  This rule is the prime reason why Separation Logic is suitable for local
  reasoning. It allows assertions to be split down to the \emph{footprint} of
  the command, just those heap locations that the command touches. The side condition
  guarantees that the command only touches the part of the assertion which is
  kept, it guarantees that nothing in $R$ can be modified by $C$.

  As a result of this, we are safe to temporarily disregard other portions of
  the heap. This permits rules to be easily composed without requiring an
  overwhelming number of predicates specifying that other portions of the heap
  haven't changed.

\section{JavaScript}

  JavaScript is a dynamically-typed, prototype-oriented language. The
  early development of the language was haphazard with rival implementations
  produced by Netscape and Microsoft for their browsers. The language's
  specification was later standardised as ECMA-262 and named ECMAScript (ES).
  The specification is unusual as it permits any
  implementation-specific extensions to the language. Even today, many
  JavaScript implementations do not conform to the ES specifications (as
  determined by the incomplete ES test suite), although this situation is slowly
  improving.

  Although originally targeted to be a programming language that was easy to
  learn and suitable for non-programmers to use, the language's variable scoping
  rules, semantics of the \js{this} keyword and \js{with} syntax are unusual
  and differ from most peoples' intuitions. In addition, the ability to
  redefine, or \emph{monkey-patch} many primitives that define the semantics of
  the language means that reasoning about programs can become impossible if
  some unknown code can execute within the same execution environment as the
  program under study.

  \subsection{Formalisation}
  There have been several efforts to formally specify various aspects of the
  JavaScript language, notable mentions include Maffeis, Mitchell and Taly's
  production of an operational semantics for a wide range of ES3
  implementations \cite{maffeis-jsopsem}, this work was continued by
  Gardner, Maffeis, Smith's \cite{gms-popl} by producing a
  Program Logic for JavaScript which is capable of expressing JavaScript
  program states with high precision. Since SES is based upon JavaScript,
  sharing many of the simple statements, this program logic was used as a
  starting point for the formalisation of SES.

  As with most formal models, they approximate and simplify actual behaviour
  where it does not impact the aspects of the language under study. JavaScript
  is a particularly large and awkward language, notable differences between the
  model and the full language are that the syntax is flattened to be purely
  expression-based and that the mechanism for automatic casting of types is
  avoided by requiring the primitive type is specified where needed in command
  pre-conditions.

  Once again, Hoare-style reasoning is adapted to suit the extensions of the
  logic. The fault-avoiding semantics for the Hoare triple are maintained. The
  basic set of Separation Logic axioms and inference rules (excluding those
  referring directly to the C-style heap implementation) are used. One notable
  difference is that the Frame rule loses its side condition as a result of all
  variables now being on the heap -- the footprint of $C$ will be expressed in
  the $P$ and $Q$ terms, so is necessarily disjoint from the framed $R$ portion of
  the rule.

  \subsection{Data Structure}

  A unique feature of the
  language is that the entire state is stored on the heap in a structure that
  loosely resembles the variable store in conventional programming languages. It
  thus follows that Separation Logic is likely to be a useful basis for the
  JavaScript program logic. However, a considerable number of complex predicates
  were required to accurately model the variable store and scoping rules.

  \TODO{diagram}

  A JavaScript object is identified by a location $l \in \loc$, each object may
  have fields $x \in \vars$ which map to values $v \in \vals$. The heap is
  represented by the partial function of locations and fields to
  values, $H \in \loc \times \vars \partialfunc \vals$.

  Field names are subdivided into user fields $\uvars$ and internal fields
  $\ivars$ (prefixed by an @ symbol). No internal fields are permitted to be
  directly accessed by executing code, they are used by the language to store
  contextual information on objects for purposes such as prototypical name
  resolution and function closures.

  Field lookup on an object proceeds by checking whether the field is present on
  the object, if not the \emph{prototypical} object stored on the $@proto$ field
  is recursively checked for the field. The list of
  objects checked for field lookup is known as the \emph{prototype chain}
  of an object. The object lookup operation is performed by the $\proto$
  predicate in the Operational Semantics and Program Logic, the corresponding
  value retrieval operation is performed by $\getValue$.

  During execution a \emph{scope chain}, an ordered list of object locations, is
  maintained. This structure is somewhat alike to stack frames in other languages.
  To perform a variable lookup, a prototypical field lookup is performed for
  each object in the scope chain in turn until a field is matched. This
  operation is performed by the $\scope$ predicate in the formalisations.

  To perform a write to a variable, a similar lookup is performed, except the
  field is modified on the \emph{scope object} for the prototype chain in which the
  variable was found -- this is so that the prototypical variable is overridden,
  but not overwritten.

  \TODO{both the above examples need diagrams}

\chapter{Syntax \& Operational Semantics}
  We now define the syntax of permissable expressions in this model of the SES
  language and present the english definition of each expression along with the
  Operational Semantics for the precise effect it has on the heap and program
  state.

\chapter{Program Logics}
\section{Extending Separation Logic}
\subsection{Global Heap View}
\subsection{Magic Wand}
\subsection{Backpointer}

\chapter{Program Proofs}
\section{Caretaker}
\section{Membrane}

\chapter{Evaluation}
\chapter{Conclusions \& Future Work}
\section{Conclusions}
\section{Future Work}

\bibliography{bibliography}
\bibliographystyle{plainnat}

\appendix
\chapter{Operational Semantics}
\chapter{Program Logic}
\chapter{Proofs}

\end{document}
