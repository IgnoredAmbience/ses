\documentclass[a4paper,notitlepage]{report}
\usepackage[usenames,dvipsnames,svgnames,x11names]{xcolor}
\usepackage{amsmath}
\usepackage{amssymb}
\usepackage{stmaryrd}
\usepackage{wasysym}
\usepackage{xspace}
\usepackage{fullpage}
\usepackage{hyperref}
\usepackage{etoolbox}
\usepackage{separationlogic}
\usepackage[numbers]{natbib}
\usepackage[nottoc]{tocbibind} % add bibliography to toc
\usepackage{parskip}
\usepackage{adjustbox}

\usepackage[sc]{mathpazo}
\linespread{1.05}         % Palatino needs more leading (space between lines)
\usepackage[T1]{fontenc}

\begin{document}
\begin{titlepage}
\begin{center}
  \textsc{\large Imperial College London} \\[0.1cm]
  \textsc{\large Department of Computing}

  \vfill

  % Still rough title
  {\LARGE A Program Logic for Verification of Security Properties of Secure
  ECMAScript Programs} \\[1cm]

  {\Large Thomas Wood} \\[1cm]

  {\large Supervisors:\\
  Dr. Gareth Smith \\
  Prof. Philippa Gardner}
  \\[2cm]

  \begin{abstract}
    \begin{center}
      We present an Operational Semantics of the Secure ECMAScript (SES) language.
      We extend Separation Logic with a backpointer operator to permit reasoning
      about reachability in the object graph whilst maintaining local reasoning.
      We define inference rules in the extended logic for SES. Finally, we prove
      the correctness of the Membrane design pattern.
    \end{center}
  \end{abstract}

  \vfill

  {\Large June 2013} \\[0.5cm]

  Submitted in partial fulfilment of the requirements for the\\
  MEng Degree in Computing of Imperial College London
\end{center}
\end{titlepage}

\chapter*{Acknowledgements}
  My utmost thanks go to Gareth Smith, for introducing me to the crazy world of
  the JavaScript language, for the continual motivation, and for assistance in
  refining my often incoherent bursts of inspiration.

  To Philippa Gardner, for an alternative perspective on the project, and for
  brightening the outlook when things looked bleak.

  To Mark Miller, for the SES language, and for his enlightening views on the
  project during his visit to Imperial.

  To my lecturers this year, for rebooting my fascination and drive for
  theoretical Computer \emph{Science}.

  And to those friends and acquaintances at home, around the department, and
  online late at night, who have been willing to engage in interesting
  discussions when in need of a break.

\tableofcontents

\chapter{Introduction}

% * Clear statement of what the project is about, nature and scope to be
%   understood by a lay reader
% * Summarise goals
% * Summarise background
% * Summarise relevance, main contributions
%   * Highlight sections of report for contributions
% * Explain motivation, identify issues to be addressed in remainder of report

  Historically, the JavaScript language was developed with very little intent
  for it to be used for much more than form validation on web pages.
  However, in recent years, web technologies have become a central part of
  day-to-day computer use, an ever growing number of applications now target the
  web as their platform of choice.
  This increase in popularity has forced the JavaScript language to mature
  considerably, through syntax, semantics, and standard library support.

  It is extremely common
  that multiple different programs from different sources may be loaded and
  executed in a single environment. This combination of different code poses
  some interesting problems regarding code-correctness and security. JavaScript
  provides no easy means to isolate different modules of code from one another,
  all data stored is potentially accessible to any code loaded on the page,
  which must therefore be implicitly trusted.

  Some attempts have been made to solve this problem by statically rewriting
  source code, or more recently by making clever use of
  a mixture of JavaScript's deprecated and modern language features.
  Examples of such attempts include AdSafe~\cite{AdSafe}, FBJS, and
  Google Caja~\cite{miller2008caja}.
  All of these have suffered from some security
  flaws in their histories as a
  result of insufficient understanding of the JavaScript language or knowledge
  of one of the myriad of differences that \emph{legitimately} exist between
  common JavaScript implementations~
  \cite{maffeis2009jsisolation, maffeis2010object-cap, ses-semantics}.
  It is therefore beneficial to introduce more security features to the core
  language specification.

  Secure ECMAScript (SES) is a proposed revision of the ECMAScript (ES)
  specification that introduces changes to the language to permit programs to be
  executed within a restricted context, safe from the possibility of unintended
  leak.
  Restricted execution permits fine-grained control over the permissions that
  untrusted executing code has by restricting the set of accessible references
  to other objects.
  This is a particularly powerful method as it is possible for unrestricted code
  to interact with code under restricted execution, it is in this situation that
  it is envisaged that lapses in the encapsulation could inadvertently occur due
  to programmer error.

  Full program verification against a specification is a good thing -- the
  thought processes of verification often can assist find bugs in the program or
  potentially, in the worst case, the specification itself.
  However, although significant progress has been made in recent years, full
  program verification is too expensive in time to be performed on all code.
  Often it is applied to systems where it is of high benefit, such as safety
  critical control systems.

  It is also cost-effective to verify systems and libraries, particularly those
  in security sensitive contexts, that will underpin a wide variety of many
  other software applications and systems. Because of the rapid growth in demand
  for features in JavaScript, we believe that SES (or a variant of it) will soon
  have a considerable number of users, this drives the benefit for verifying its
  low-level libraries.

  At present, we have a program logic capable of describing the functional
  properties of JavaScript programs, however, it is not capable of expressing
  the security properties that SES aims to enforce. It is highly convenient to
  reason about the security properties of a program at the same time as
  reasoning about the functional properties of it, as very often one is
  intrinsically linked to the other. We thus need to extend the logic to be able
  to express both functional and security properties.

  \section{Project Aims}

  In chapter~\ref{chp:sem} we describe and specify the Syntax and Operational
  Semantics of SES language
  as a means to understand the language, highlight its differences from
  JavaScript, and for direct reference whilst implementing the inference rules
  for the Program Logic.

  In chapter~\ref{chap:proglogic} we extend Separation Logic to support
  reasoning about security properties of SES programs. We define the Hoare
  triple for the new logic, the soundness properties between them and the
  Operational Semantics, and produce axiomatic rules for SES instructions.

  In chapter~\ref{chap:proofs} we prove some fundamental programming patterns
  used regularly in SES programs, namely the Caretaker and Membrane patterns.

\chapter{Background}
\section{Program Verification \& Hoare Logic}
  This style of reasoning about the functionality and correctness of computer
  programs was first presented in his 1969 paper ``An Axiomatic Basis for
  Computer Programming''\cite{Hoare1969Axiom}, the principles presented were
  based on earlier work by Floyd\cite{floyd1967assigning} for flowchart
  representations of programs, and by others in other branches of mathematics;
  Hoare developed the style of the reasoning to fit textual representation of
  programs.

  Hoare's reasoning depends on the basis that commands used in programming
  languages should be well-defined and thus the consequences of a computer
  program should be determinable directly from the program's text by deductive
  means. Program proofs are deduced through the application of inference rules
  to sets of axioms.

  The main contribution of the paper was to introduce the notation, now known as
  a Hoare triple:
    \[ \{P\} C \{Q\} \]
  $P$ and $Q$ are logical assertions and $C$ is some command or program. The
  triple is interpreted by Hoare as ``if the assertion $P$ is true before
  initiation of a program $C$, then the assertion $Q$ will be true on its
  completion.'' Other interpretations of triples are possible, as will be shown
  later.

  The axioms selected for use in the reasoning alter the semantics of the system
  being reasoned about. Hoare exemplifies this by presenting axioms that apply
  to arithmetic under all sets of integers, followed by three alternative axioms
  that describe different semantics for overflow in sets of finite integers.

  A number of axioms and inference rules are universal or common to nearly all
  imperative languages, Hoare presents four of these:
  \begin{display}{}
    \stateaxiom{(Assign)}
    {\tr{P[f/x]}{x := f}{P}} \qquad

    \staterule{(Consequence)}
    {\tr{P'}{C}{Q'} \quad P \impl P' \quad Q' \impl Q}
    {\tr{P}{C}{Q}} \\
    \\
    \staterule{(Iteration)}
    {P \quad \tr{B}{S}{P}}
    {\tr{P}{\textbf{while } B \textbf{ do } S}{\lnot B \land P}} \qquad

    \staterule{(Composition)}
    {\tr{P}{C_1}{Q} \quad \tr{Q}{C_2}{R}}
    {\tr{P}{C_1; C_2}{R}}
  \end{display}

  The consequence and composition rules are straight-forward. The iteration rule
  is also reasonably simple once it is noted that the $P$ term is the loop
  invariant. The assignment axiom is the most subtle of the set when not taken
  in the context of a proof. The axiom asserts that the post-condition will hold
  if the \emph{precondition} has the variable replaced by
  the assigned value. This axiom makes more sense if it is considered to be used
  whilst reasoning backwards. For example, to prove $\tr{\ltrue}{r := x; q :=
  0}{x = r + y \times q}$:

  \begin{tabular}{ll}
    $\{\ltrue \impl x = x + y \times 0 \}$ & Axiom \\
    $\tr{x = x + y \times 0}{r:=x}{x = r + y \times 0}$ & Assignment \\
    $\tr{x = r + y \times 0}{q:=0}{x = r + y \times q}$ & Assignment \\
    $\tr{\ltrue}{r:=x}{x = r + y \times 0}$ & Consequence \\
    $\tr{\ltrue}{r:=x; q:=0}{x = r + y \times q}$ & Composition\\
  \end{tabular}

  It is noted that proofs in this format are long and tedious, but it is
  possible to derive more useful rules from the given ones which would somewhat
  reduce the length of these proofs. This is now the generally accepted method
  of presenting proofs in Hoare's system, so the above proof could also be
  presented as follows:

  \begin{tabular}{l}
    $\{\ltrue\}$ \\
    $\{x = x + y \times 0\}$ \\
    $\ r:=x$ \\
    $\{x = r + y \times 0\}$ \\
    $\ q:=0$ \\
    $\{x = r + y \times q\}$
  \end{tabular}

  Indeed, in his 1971 paper~\cite{Hoare1971proof}, Hoare presents a proof in
  such a style for the non-trivial \textsc{Find} algorithm.

  %Hoare closes his paper by reinforcing the need for formal proof and
  %specification of computer programs and languages.

\section{Separation Logic}

  Whilst Hoare's methodology is suitable for reasoning about programs which only
  operate within the stack, it is not suitable for reasoning about programs that
  use the heap. This is because that data structures within the heap are more
  prone to being shared in some manner (for example, multiple pointers to the
  same heap location, or even arbitrary pointer arithmetic).

  Reasoning about these shared data structures using conventional logics has been attempted by a number of
  researchers, but no solution was particularly adequate. The main issue with
  verification of shared data structures was that a global view of the data, and
  attempts to axiomatize use of the heap would result in assertions
  that rapidly grew in complexity, both in terms of length, but also from
  universal quantification, for the need to assert that the rest of the heap was
  not modified along with the location of interest.

  Burstall considered a local approach to this reasoning in 1972
  \cite{burstall1972some}, but the approach featured little further research and
  was unable to reason about data structures such as doubly-linked lists
  \cite{reynolds2000intuitionistic}. Reynolds began further research into the
  use of a spatial logic to permit local reasoning about the heap and in
  collaboration with O'Hearn, Ishtiaq and Yang produced Separation Logic, a
  combination of the spatial Bunched Implication logic with Hoare's methods for
  reasoning about programs
  \cite{Ishtiaq2001BI,OHearn2001Local,Reynolds2002Separation}.

  \subsection{Semantics of Separating Operators}

  Separation Logic introduces the separating conjunction operator, $P \sep Q$,
  which can be thought of as splitting the heap into two disjoint portions, in
  which $P$ holds in one and $Q$ the other. To denote the contents of the
  heap, the $\mapsto$ operator is also introduced to relate heap locations with
  values, $(x \mapsto 4)$ denotes that the value $4$ is stored in the heap
  location $x$. It is often the case in languages where the heap is directly
  mapped to memory that locations are integers, so the notation $(x \mapsto 4,
  y)$ is shorthand for $(x \mapsto 4) \sep (x+1 \mapsto y)$.

  The separating conjunction only enforces disjointedness of the storage
  locations of values, it is permitted to point to other heap locations from
  within the heap. For example $(x \mapsto 4, y) \sep (y \mapsto 3, x)$ is
  permissible and required to be able to build complex data structures.
  
  $x \mapsto -$ is shorthand meaning that there is some value stored at heap
  location $x$.

  The disjointedness enforced by the $\sep$ operator provides a very powerful
  means of reasoning about operations made to the heap. Take for example the
  assertion $x \mapsto 1 \sep P$ and the command $x := 2$, since we have $x$
  explicitly defined on the left of the $\sep$, we know that it cannot occur
  within $P$. We can therefore safely conclude that $P$ is not modified and that
  the postcondition must be $x \mapsto 2 \sep P$.

  A counterpart for implication also exists,
  the separating implication operator, $\wand$ (commonly known as the
  \emph{magic wand}). The assertion $((x
  \mapsto 7) \wand P)$ has the meaning that if $x$ is updated to contain the
  value $7$, then $P$ will be true. So: $(x \mapsto 7) \sep ((x \mapsto 7) \wand
  P) \impl P$.

  This operator can thus, in effect, be
  used to generate the weakest precondition for a command given a desired
  postcondition. This is often used for backwards reasoning, where a
  postcondition for a command is known, but the precondition is not.

  In separation logic, an assertion is \emph{pure} if it does not make reference
  to the heap. Expressions $P \sep Q$ where $P$ is pure and Q is not, may have
  multiple derivations through splitting of the heap in different ways. To
  prevent this, it is convention to use $a \doteq b$ as an abbreviation for
  $a = b \land \lemp$, and similarly for $\dotin$.

  \subsection{Program Reasoning}
  \label{subsec:programreasoning}

  Just as Hoare's logic was built atop of classical logic, we can extend
  Hoare's logic to the spatial logic including the separating conjunction,
  the combined system is known as Separation Logic.

  The first concept to translate is that of the Hoare triple, $\tr{P}{C}{Q}$.
  For the remainder of this report, we will consider Separation Logic to use
  \emph{fault-avoiding} semantics for the triple. Under these semantics, then the
  requirements for $C$ are strengthened, such that ``if $C$ is executed in a
  state satisfying $P$, then it will \emph{not fault}, and if it terminates it
  will do so in a state satisfying $Q$''.

  Separation Logic extends the standard Hoare inference rules and axioms, the
  rules for Consequence, Composition, and Simple Assignment are used without
  modification.

  The obvious new axioms to introduce are those for heap access and
  modification, the following set are taken from \cite{OHearn2001Local}.

  \begin{display}{Axioms for heap access}
    $\begin{array}{ll}
      \stateaxiom{(Update)}{\tr{x \mapsto -} {x := v} {x \mapsto v}} &
      \stateaxiom{(Delete)}{\tr{x \mapsto -} {\js{delete}(x)} {\emp}} \\
      \stateaxiom{(Allocate)}{\tr{\emp} {x := new()} {\exists L \st x=L \land L \mapsto\_}} &
      \stateaxiom{(Lookup)}{\tr{x \mapsto v} {y := [x]} {y = v \land x \mapsto v}} \\
    \end{array}$
  \end{display}
  These axioms may be expressed in other ways, for example, in a form suitable
  for backwards reasoning, or with use of existential quantifiers.

  There is one important new inference rule, the frame rule: \\

  $
    \staterule{(Frame Rule)}
    {\tr{P}{C}{Q}}
    {\tr{P \sep R}{C}{Q \sep R}}
    {\mathrm{modifies}(C) \disj \mathrm{free}(R)}
  $\\

  This rule is the prime reason why Separation Logic is suitable for local
  reasoning. It allows assertions to be split down to the \emph{footprint} of
  the command, just those heap locations that the command touches. The side condition
  guarantees that the command only touches the part of the assertion which is
  kept, it guarantees that nothing in $R$ can be modified by $C$.

  As a result of this, we are safe to temporarily disregard other portions of
  the heap. This permits rules to be easily composed without requiring an
  overwhelming number of predicates specifying that other portions of the heap
  haven't changed.

\section{JavaScript}

  JavaScript is a dynamically-typed, prototype-oriented language. The
  early development of the language was haphazard with rival implementations
  produced by Netscape and Microsoft for their browsers. The language's
  specification was later standardised as ECMA-262 and named ECMAScript (ES).
  The specification is unusual as it permits any
  implementation-specific extensions to the language. Even today, many
  JavaScript implementations do not conform to the ES specifications (as
  determined by the incomplete ES test suite), although this situation is slowly
  improving.

  Although originally targeted to be a programming language that was easy to
  learn and suitable for non-programmers to use, the language's variable scoping
  rules, semantics of the \js{this} keyword and \js{with} syntax are unusual
  and differ from most peoples' intuitions. In addition, the ability to
  redefine, or \emph{monkey-patch} many primitives that define the semantics of
  the language means that reasoning about programs can become impossible if
  some unknown code can execute within the same execution environment as the
  program under study.

  \subsection{Formalisation}
  There have been several efforts to formally specify various aspects of the
  JavaScript language, notable mentions include Maffeis, Mitchell and Taly's
  production of an operational semantics for a wide range of ES3
  implementations \cite{maffeis-jsopsem}, this work was continued by
  Gardner, Maffeis, Smith's \cite{gms-popl} by producing a
  Program Logic for JavaScript which is capable of expressing JavaScript
  program states with high precision. Since SES is based upon JavaScript,
  sharing many of the simple statements, this program logic was used as a
  starting point for the formalisation of SES.

  As with most formal models, they approximate and simplify actual behaviour
  where it does not impact the aspects of the language under study. JavaScript
  is a particularly large and awkward language, notable differences between the
  model and the full language are that the syntax is flattened to be purely
  expression-based and that the mechanism for automatic casting of types is
  avoided by requiring the primitive type is specified where needed in command
  pre-conditions.

  Once again, Hoare-style reasoning is adapted to suit the extensions of the
  logic. The fault-avoiding semantics for the Hoare triple are maintained. The
  basic set of Separation Logic axioms and inference rules (excluding those
  referring directly to the C-style heap implementation) are used. One notable
  difference is that the Frame rule loses its side condition as a result of all
  variables now being on the heap -- the footprint of $C$ will be expressed in
  the $P$ and $Q$ terms, so is necessarily disjoint from the framed $R$ portion of
  the rule.

  \subsection{Data Structure}
  \label{sec:intro:js:datastructure}

  A unique feature of the
  language is that the entire state is stored on the heap in a structure that
  loosely resembles the variable store in conventional programming languages. It
  thus follows that Separation Logic is likely to be a useful basis for the
  JavaScript program logic. However, a considerable number of complex predicates
  were required to accurately model the variable store and scoping rules.

  A JavaScript object is identified by a location $l \in \loc$, each object may
  have fields $x \in \vars$ which map to values $v \in \vals$. The heap is
  represented by the partial function of locations and fields to
  values, $H \in \loc \times \vars \partialfunc \vals$. For example, $(x, y)
  \pointsto z$ means that the object in location $x$, has a field named $y$
  which points to the location $z$. Note that unlike the original Separation
  Logic semantics, tuples of values are not supported, since the heap locations
  are now unordered.

  Field names are subdivided into user fields $\uvars$ and internal fields
  $\ivars$ (prefixed by an @ symbol). No internal fields are permitted to be
  directly accessed by executing code, they are used by the language to store
  contextual information on objects for purposes such as prototypical name
  resolution and function closures.

  Field lookup on an object proceeds by checking whether the field is present on
  the object, if not the \emph{prototypical} object stored on the $@proto$ field
  is recursively checked for the field. The list of
  objects checked for field lookup is known as the \emph{prototype chain}
  of an object. The object lookup operation is performed by the $\proto$
  predicate in the Operational Semantics and Program Logic, the corresponding
  value retrieval operation is performed by $\getValue$.

  During execution a \emph{scope chain}, an ordered list of object locations, is
  maintained. This structure is somewhat alike to stack frames in other languages.
  To perform a variable lookup, a prototypical field lookup is performed for
  each object in the scope chain in turn until a field is matched. This
  operation is performed by the $\scope$ predicate in the formalisations.

  To perform a write to a variable, a similar lookup is performed, except the
  field is modified on the \emph{scope object} for the prototype chain in which the
  variable was found -- this is so that the prototypical variable is overridden,
  but not overwritten.

  SES is an object-capability style of language, objects are considered to hold
  capabilities to privileged functionality or information. Because of this, it
  is essential that the language does not provide the ability to access
  arbitrary memory locations.
  Although locations are values that are stored in the heap, they
  may not be directly referred to by user programs. This means that
  references to objects are not forgeable. This is a key security concept in the
  SES language: an object must be passed a reference to another object to be
  able to access it.
  Since all functions (including standard libraries and resources) are
  themselves objects, the concept of access extends to one of privilege when a
  security-sensitive interface needs to be protected.

\chapter{SES Language Syntax \& Operational Semantics}
  \label{chp:sem}
  We now define the syntax and semantics of expressions in this model of the SES
  language. We present the English definition of each expression along with the
  Operational Semantics for the precise effect it has on the heap and program
  state. We also discuss how SES differs from JavaScript and the rationale for
  these differences.

  \section{Syntax}
  \label{sec:syntax}
  \newcommand{\syntaxline}[3][\pipe]{#1 & \js{#2} & \textit{#3}\\}
    \begin{display}{Syntax of SES values and expressions, \js{v}, \js{e}}
      $\begin{array}{rll}
        \syntaxline[\js{v} ::=]{n}{Number}
        \syntaxline{s}{String}
        \syntaxline{\und}{Undefined}
        \syntaxline{\nil}{Null}
        \syntaxline{\true \pipe \false}{Booleans}
        \syntaxline[\js{e} ::=]{v}{Value}
        \syntaxline{\jsvar{x}}{Variable declaration}
        \syntaxline{x}{Identifier}
        \syntaxline{\{x_1:e_1, \ldots, x_n:e_n\}}{Object creation}
        \syntaxline{e; e}{Sequence}
        \syntaxline{e \oplus e}{Binary operator}
        \syntaxline{if(e) \{e\} else \{e\}}{Conditional}
        \syntaxline{while(e) \{e\}}{Looping}
        \syntaxline{e = e}{Assignment}
        \syntaxline{e.x}{Member access}
        \syntaxline{e[e]}{Computed member access}
        \syntaxline{e(e)}{Function call}
        \syntaxline{this}{this}
        \syntaxline{\jsfun{e}{e}}{Function creation}
        \syntaxline{\jsfun[x]{e}{e}}{Named function creation}
        \syntaxline{\jsnew{e}{e}}{Object construction}
        \syntaxline{reval(e, e)}{Restricted evaluation -- similar to \js{eval} in JavaScript}
        \syntaxline{freeze(e)}{Freeze -- make one object immutable}
        \syntaxline{def(e)}{Recursive freeze -- make many objects immutable}
      \end{array}$
    \end{display}

  Numbers, strings, booleans and the special values of \und and \nil are the
  only primitive values.

  We model SES as an expression-based language, the full language follows
  JavaScript's Strict Mode parsing and variable declaration rules. For this
  project, we assume that all programs proved additionally pass these checks.

  SES syntax differs from JS syntax only by removal and addition of several
  commands. The most notable removal is that of the \tjs{with} expression, this
  expression would add the given JS object into the scope chain for the duration
  of execution of the given block. This causes the scope chain to potentially
  have multiple prototype chains requiring traversal. Removal of this command
  ensures that all objects in the scope chain, other than the root, have no
  prototype chains. This removal considerably simplifies the reasoning required
  for variable lookup. The deprecation of the \tjs{with} syntax has also been
  started in the most recent ES specifications.

  We are modelling three additional primitives present in the SES language,
  these are restricted evaluation, \tjs{reval(e,e)}, and commands to make objects
  immutable, \tjs{freeze(e)} and \tjs{def(e)}.


  \section{Operational Semantics}
  \label{sec:opsems}

  We now describe the semantics of each expression given in the syntax, both
  informally in English, and formally with the big-step operational semantics
  rule, which precisely describes how the program state is modified by execution
  of each expression.

  Although daunting at first, the rules are quite simple to follow. The syntax
  $H,L,\js{e} \evalsto H',r$ means that the heap $H$, scope chain $L$, and
  expression $\js{e}$ will evaluate to the heap $H'$ and return value $r$. (A
  return value is a value, or a reference, $\fld{l}{\js{x}}$, to a location and
  a field).

  An expression will be evaluated iff the preconditions above the line hold,
  these may be evaluations of subexpressions, or involve the use of predicates
  that test or update the current state.
  Ordering of the evaluation of these preconditions is enforced through the
  labelling of the variables in each rule.

  The semantics make use of several auxiliary functions defined in
  Appendix~\ref{app:opsems} and described above in
  Section~\ref{sec:intro:js:datastructure}. Recall that $\getValue$ is used to
  dereferencing a JavaScript or SES reference value.
  As the execution of a subexpression followed by the dereferencing of its
  return reference into a value is a
  common activity, $\gevalsto$ is defined as shorthand for this:
  \[ H,L,\js{e} \gevalsto H',v \triangleq \exists r. (H,L,\js{e} \evalsto H',r
  \wedge \getValue(H',r) = v) \]

  The semantics here were derived from a number of sources, we used the
  JavaScript program logic~\cite{gms-popl} as the baseline for all commands.
  Deviations from this are noted with the appropriate command.

  \subsection{Structural Expressions}
  \label{subsec:structuralexpressions}
  All of these semantics are common across most imperative languages.

  \stateaxiom{(Value)}{H,L,\js{v} \evalsto H,v}

  A value simply evaluates to itself, the heap is not modified. No preconditions
  are required.

  \stateaxiom{(Variable declaration)}{H,L,\jsvar{x} \evalsto H,\und}

  The variable declaration expression is only checked on entry to a function
  body, they are treated as a no-op during execution.

  \staterule{(Sequence)}
    {H,L,\js{e_1} \evalsto H'',r' \\
     H'',L,\js{e_2} \evalsto H',r}
   {H,L,\js{e_1; e_2} \evalsto H',r}
  \nopagebreak

  The sequencing operator specifies that the expression on the left should be
  evaluated, followed by the expression on the right. The result of the right
  subexpression is returned.

  \staterule{(Binary operator)}
    {H,L,\js{e_1} \gevalsto H'',v_1 \\
     H'',L,\js{e_2} \gevalsto H',v_2\\
     v_1 \oplus v_2 = v}
   {H,L,\js{e_1 \oplus e_2} \evalsto H',v}

  Binary operations are evaluated left-to-right, and include standard
  mathematical and string operations. Formalisation of semantics for all of these
  are beyond the scope of this project, but will correspond closely to similar
  works on the JavaScript family of languages.

  \staterule{(Conditional true)}
  {H,L,\js{e_1} \gevalsto H'',v \quad \istrue(v) \\
   H'',L,\js{e_2} \evalsto H',r}
  {H,L,\js{if(e_1)\{e_2\} else \{e_3\}} \evalsto H',r}
\qquad
  \staterule{(Conditional false)}
  {H,L,\js{e_1} \gevalsto H'',v \quad \isfalse(v) \\
   H'',L,\js{e_3} \evalsto H',r}
  {H,L,\js{if(e_1)\{e_2\} else \{e_3\}} \evalsto H',r}

  The conditional expression evaluates the first sub-expression, if it evaluates
  to \js{true} when cast to a boolean then the second
  sub-expression is evaluated and result returned, otherwise the third
  sub-expression is evaluated and result returned.

  \staterule{(While true)}
  {H,L,\js{e_1} \gevalsto H'', v \quad \istrue(v) \\
   H'',L,e_2\js{;while(e_1)\{e_2\}} \evalsto H', v''}
  {H,L,\js{while(e_1)\{e_2\}} \evalsto H',\und}
\qquad
  \staterule{(While false)}
  {H,L,\js{e_1} \gevalsto H', v \quad \isfalse(v)}
  {H,L,\js{while(e_1)\{e_2\}} \evalsto H',\und}

  Whilst the loop guard is true, the loop is syntactically unrolled once, and
  the resulting expression evaluated. When the guard condition evaluates to
  false, \und is returned. This is the standard loop unrolling definition.

  \subsection{Heap-based Expressions}
  \label{sec:opsem:heapexprs}
  \staterule{(Variable Resolution)}
    {\scope(H,L,x) = l' \quad l' \neq \nil}
    {H,L,\js x \evalsto H, \fld{l'}{x}}

  The $\scope$ predicate is used to search the scope chain for an object which
  contains the field matching the variable name. A reference to the \emph{scope}
  object and the field is returned.

  If the scope object cannot be found, fail to evaluate -- this
  behaviour follows the ES5.1 Strict standard, a considerable difference from
  ES5.1 Non-Strict or earlier JS, which return the reference
  $\fld{\nil}{\js{x}}$.  JS assigned write attempts to the \nil object to the
  \emph{global} object -- this posed an obvious isolation problem.
  This new behaviour enforces that all variables must be declared before use.

  \staterule{(Member access)}
  {H,L,\js e \gevalsto H',l' \\
   l' \neq \nil}
  {H,L,\js{e.x} \evalsto H', \fld{l'}{x}}
\qquad
  \staterule{(Computed member access)}
  {H,L,\js{e_1} \gevalsto H'',l' \\
   l' \neq \nil \\
   H'',L,\js{e_2} \evalsto H', x}
  {H,L,\js{e_1[e_2]} \evalsto H',\fld{l'}{x}}

  Returns a reference to the named field of the object referenced by the first
  subexpression. For the computed member access, the second subexpression is
  evaluated to a field name. In full JS and SES, these rules are additionally
  complicated with details about type casting. For simplicity, here we focus on
  the case where no type casting is necessary.

  \staterule{(Assignment)}
  {H,L,\js{e_1} \evalsto H_1,\fld{l}{x} \qquad
   \ReadWrite(H_1, l) \\
   H_1,L,\js{e_2} \gevalsto H_2, v \\
   H' = H_2[(l,x) \pointsto v]}
  {H,L,\js{e_1 = e_2} \evalsto H', v}

  Assigns the value on the right hand side to the subexpression evaluating to a
  reference on the left. The object being assigned to must be writable (see
  definition of Freeze).

  Assignments to undeclared variables are forbidden (enforced by the Variable
  rule).

  \staterule{(Object creation)}
    {H_0 = H \disju \obj(l,\lop)\\
     \forall i\in 1..n \st \left(\begin{array}{l}
      H_{i-1},L,\js{e_i}\gevalsto H_i',v_i \\
      H_i = H_i' [ (l,\js{x_i}) \pointsto v_i]\end{array}\right)}
    {H,L,\js{\{x_1:e_1,\dots,x_n:e_n\}} \evalsto H_n,l}

  The Object Construction syntax produces a new object at the new location $l$,
  with fields named \js{x_1, \ldots, x_n} which map to the values of expressions
  \js{e_1, \ldots, e_n} when evaluated in order at creation. The location of the
  new object is returned.

  \subsection{Function-related Expressions}

  \staterule{(Function creation)}
  {H' = H \disju \obj(l,\lop) \disju \fun(l',L,\js{x},\js{e},l)}
  {H,L,\jsfun{x}{e} \evalsto H',l'}

  Creates a function object, assigning the body, parameter
  declarations and the current scope to internal fields. The shape of this
  function object is given by the $\fun$ auxiliary function, defined in
  Appendix~\ref{app:opsems}. A new object is also
  created and assigned to the \js{prototype} field for use as the prototype of
  objects produced by using this function as a constructor.

  \staterule{(Named function creation)}
  {H' = H \disju \obj(l,\lop) \disju \fun(l',l_1 \cons L,\js{x},\js{e},l) \disju
    l_1 \pointsto \{@proto:\nil, y:l'\}}
  {H,L,\jsfun[y]{x}{e} \evalsto H',l'}

  This is the same as standard function creation,
  but also adds the name and a self-reference to the function to the function's
  scope record. This permits recursive functions to be created. \emph{Note:} the
  name of the function is not added to the current scope, it is permitted in the
  actual language, but is considered to be syntactic sugar, combining creation and
  assignment.

  \staterule{(Function call)}
  {H,L,\js{e_1} \evalsto H_1,r_1\qquad
   \pickThis(H_1,r_1)=l_2\qquad
   \getValue(H_1,r_1)=l_1\\
   H_1(l_1,@body)=\lambda \js{x.e_3}\qquad
   H_1(l_1,@scope)= L'\\
   H_1,L,\js{e_2} \gevalsto H_2,v\\
   H_3 = H_2\disju\act(l,\js x,v,\js{e_3},l_2) \\
   H_3,l \cons L',\js{e_3} \gevalsto H',v'}
  {H,L,\js{e_1(e_2)} \evalsto H',v'}

Executes the body of the function, using the passed expression
as the value to bind to the parameter. The body of the function is executed in
the scope stored with the function, any variables defined within the function
body are defined on an activation record so that they are lexically scoped.

  \staterule{(Object construction)}
  {H,L,\js{e_1} \gevalsto H_1,l_1 \qquad
   l_1\neq \nil\qquad
   H_1(l_1,@body)=\lambda \js{x.e_3}\\
   H_1(l_1,@scope)= L'\qquad
   H_1(l_1,\js{prototype})= v\\
   H_1,L,\js{e_2} \gevalsto H_2,v_1  \qquad
   l_2 = \objOrGlob(v) \\
   H_3 = H_2\disju \obj(l_3,l_2) \disju\act(l,\js x,v_1,\js{e_3},l_3)\\
   H_3,l \cons L',\js{e_3} \gevalsto H',v_2\qquad
   \getBase(l_3,v_2) = l'}
  {H,L,\jsnew{e_1}{e_2} \evalsto H',l'}

Constructs an object using the given function, an object
is created as usual, it's prototype is assigned to that of the function's
\js{prototype} field. The body of the function is then executed, commonly used
to initialize the newly created object.

% ES5-Strict definition, plus reval global - matches caja impl (and intuition).
  \staterule{(This)}
  {\scope(H,L,@this)=l \\
   (l,@this) \pointsto l'}
  {H,L,\js{this} \evalsto H,l'}

The \js{this} expression is context-dependent. When used
outside of a
function, it should evaluate to the most global accessible scope.
When used within a function, if the function is called using Member Access (such
as \js{ob.f()}), it
will evaluate to the object on which that function was called (in this case: \js{ob}). Otherwise
(most likely a direct function call such as \js{f()}),
\js{this} returns undefined. Note that this is particularly
problematic because functions can easily be aliased off of the object on which
they appear to be defined. For this reason the code \js{g = ob.f ; ob.f()}
may behave significantly differently from the code \js{g = ob.f ; g()}

These semantics match ES5.1 Strict mode semantics and also agree with the
JS-based implementation of the SES language.

  \subsection{SES-specific Expressions}

  \staterule{(Restricted evaluation)}
  {H,L,\js{e_1} \gevalsto H_1,v \qquad
   \js{e_3} = \parse(v) \\
   H_1,L,\js{e_2} \gevalsto H_2, l \\
   H_3 = H_2 \disju l' \pointsto \{@this:l, @proto:\nil\} \disju
     \defs(\_, l', \js{e_3}) \\
   H_3, l' \cons [l], \js{e_3} \gevalsto H',v }
  {H,L,\js{reval(e_1, e_2)} \evalsto H',v}

Parse the first given expression as SES code.
A new scope chain is prepared. Its root is the
object specified by the second parameter, the \emph{imports} to the restricted
environment.
An activation record which initializes any variables declared in the parsed
source is then appended. This may cause imported objects to be shadowed.
The return value of the restricted evaluation statement is the same as that of
the final statement of the source to be executed.

  \staterule{(Freeze)}
  {H,L,\js{e} \gevalsto H'', l\\
   H' = H''[(l, @frozen) \pointsto \true]}
  {H,L,\js{freeze(e)} \evalsto H', l}

Makes the provided object read-only. This prevents field additions,
modifications and deletions.

This command is not present in the previous JS program logic. It is a primitive
specified in the ES5 spec, however this spec has a bug, meaning that frozen
fields on an object's prototype are not permitted to be overridden.
Most browsers opt to ``fix'' the spec bug, however, we follow the ES5 spec in
this case.

  \staterule{(Recursive freeze)}
  {H,L,\js{e} \gevalsto H'', l \\
   H' = H''[\auxDef(H'', l, \{\})]}
  {H,L,\js{def(e)} \evalsto H', l}

Recursively calls freeze on all objects reachable
via user-defined fields from the given object.

We use a new auxiliary function, $\auxDef$ here to handle the recursion onto
fields of the object, whilst avoiding non-termination due to cyclical object
graphs. It is defined as:
\[
  \auxDef(H,l,s) \triangleq \begin{cases}
    \emp & l \in s \lor l \not\in \loc \\
    (l,@frozen) \mapsto \true \cup \bigcup_{(l,x_n) \in H, x_n \in \uvars}
      \auxDef(H,H(l,x_n),s \cup\{l\}) & \mbox{otherwise} \\
  \end{cases}
\]

The Def command is defined in SES as a function that uses Freeze and the
\js{for..in} syntax to loop over fields on the object. We have not chosen to
model this syntax for technical reasons\footnote{The ES specification for
\js{for..in} is non-deterministic! (Even between two consecutive executions of
the same loop under the same conditions!)}.

\chapter{Program Logics}
\label{chap:proglogic}
Although the operational semantics provide a precise description of how a
program will execute, their use for practical program verification is
long-winded, tedious and error-prone. Instead, we use a program logic, which
allows us to reason at the level of abstraction best suited to the particular
program we would like to verify. Since so much of the behaviour of JS and SES is
dependant on the state of the program heap, it is natural to start with
separation logic.

We first extend separation logic to be able to express not only what can be
accessed \emph{from} a given object, but also what paths exist to obtain access
\emph{to} a given object. We then produce a Hoare-style inference rule for each
expression in the syntax of the language.

\section{Extending Separation Logic}
\label{sec:extendingseplog}
As previously discussed in section~\ref{sec:intro:js:datastructure}, SES is an
object capability language where references to objects are not forgeable.
To reason about the security properties of SES programs, we need to be able to
assert that only trusted objects contain references to sensitive data or
functionality. 
Separation logic is unable to make statements like this, since the
statement is inherently \emph{non-local}. If we attempt to make such a
statement, then at any time, we might use the frame rule to introduce an object
which contradicts the statement we made.

One naive method to solve this would be to expand the footprint of instructions
to cover all the portions of the heap that could potentially point to the object
we wish to check is secure. However, this would quickly cause separation logic
to lose its advantages over normal boolean logics -- the pre- and
post-conditions for rules now no longer precisely state which portions of the
heap are required for the command to function.

We instead extend the logic to support these sorts of statements by: reintroducing a global heap state to the
assertion satisfaction relation; redefining the existing logical operators to
carry around the global heap state without examining it; and introducing new
logical operators that allow us to make limited non-local assertions without
breaking the frame rule.

\subsection{Intuition}
\label{sec:extendingseplog:intuition}

Regular separation logic for JS assertions are defined using a
\emph{satisfaction relation} of the following form:

\[
h,L,\env \satisfies P \iff \text{ some conditions }
\]

In this relation: $h$ is the portion of the JS heap currently under
consideration; $L$ is a list of pointers to objects in the heap which are
currently serving as an emulated variable store; and $\env$ is a logical
environment which defines the meanings of the logical variables used in the
assertion $P$.

In order to allow limited assertions about the global heap, we must define our
assertions slightly differently. We use a satisfaction relation of the following
form:

\[
h,h_g,L,\env \satisfies P \iff \text{ some conditions }
\]

The one change is the addition of the \emph{global heap} $h_g$. We require that
$h$ must always be a sub-heap of $h_g$. That is to say: the portion of the heap
under consideration must be a part of the whole heap. In the following sections
we describe how this change threads through the definitions of the existing
regular separation logic assertions, and how it allows us to create new sorts of
assertion.

\subsection{Existing Separation Logic Operators}
The pure assertions, $\ltrue,\lfalse,\land,\lor,\lnot$ extend trivially with the
addition of the new heap state, for example,the new definition of $\land$ is:
\[
  h,h_g,L,\env \satisfies P \land Q \iff
    (h,h_g,L,\env \satisfies P) \land (h,h_g,L,\env \satisfies Q)
\]

The definitions of the separating conjunction $\sep$ and partially separating
conjunction $\sepish$ also simply extend by passing through the global heap state:
\[\begin{array}{c}
  h,h_g,L,\env \satisfies P \sep Q \iff
    \exists h_1,h_2 \st h \equiv h_1 \disju h_2 \land
    (h_1,h_g,L,\env \satisfies P) \land (h_2,h_g,L,\env \satisfies Q) \\
  h,h_g,L,\env \satisfies P \sepish Q \iff
    \exists h_1,h_2,h_3 \st h \equiv h_1 \disju h_2 \disju h_3 \land
    (h_1 \disju h_3, h_g, L, \env \satisfies P) \land
    (h_2 \disju h_3, h_g, L, \env \satisfies Q)
\end{array}\]

\subsection{Magic Wand}
\label{sec:wand}
The separating implication, or magic wand, operator, $\wand$, is usually defined
as follows:
\[
  h \satisfies P \wand Q \iff \forall h' \st (h' \satisfies P) \land h \disj
    h' \impl (h \disju h' \satisfies Q)
\]

It means that if the heap is extended by a disjoint portion that satisfies $P$,
then the resulting heap will satisfy $Q$.

The magic wand is generally used for the generation of weakest preconditions of
a command. Our extension to Separation Logic shows that there are two cases of
weakest precondition generation.

The first of these is when the weakest
precondition is extending the heap footprint with
information consistent with the heap, for example the weakest precondition for
object creation of the form $P \wand Q$ would have the object asserted in $Q$,
and the footprint of the subexpressions used to populate the object's fields in
$P$.

The second case is when the weakest precondition is used for speculating or
\emph{hypothesising} new values in the existing heap. For example, in the
weakest precondition for the overwriting assignment.

The magic wand operator is traditionally defined to be the right adjoint of $\sep$:
\newcommand{\entails}{\vdash}
\[
P \sep Q \entails R \text{ iff } P \entails Q \wand R
\]

Since separation logic usually defines $\wand$
as follows:
\[
  h \satisfies P \wand Q \iff \forall h' \st (h' \satisfies P) \land h \disj
    h' \impl (h \disju h' \satisfies Q)
\]

We might define $\wand$ in our context to preserve the right adjoint property like so:
\[
  h,h_g \satisfies P \wand Q \iff \forall h' \st (h',h_g \satisfies P) \land h \disj
    h' \impl (h \disju h',h_g \satisfies Q)
\]

Now consider $x \pointsto 1 \sep (x \pointsto 2 \wand Q)$, an instance of the
hypothesising weakest precondition noted earlier.

 In regular separation logic, this assertion is satisfiable. Consider the heap
 $x:1$ as follows:
\begin{align*}
           & x:1 \satisfies x \pointsto 1 \sep (x \pointsto 2 \wand Q) \\
\text{iff } & x:1 \satisfies x \pointsto 1 \land \lemp \satisfies (x \pointsto 2 \wand Q) \\
\text{iff } & x:1 \satisfies x \pointsto 1 \land x:2 \satisfies x \pointsto 2 \land (x:2 \disju \lemp) \satisfies Q
\end{align*}

But with our naive definition of $\wand$, this assertion is
\emph{unsatisfiable}. Consider again the heap $x:1$, which satisfied the regular
separation logic version of this assertion:
\begin{align*}
           & x:1,x:1 \satisfies x \pointsto 1 \sep (x \pointsto 2 \wand Q) \\
\text{iff } & x:1,x:1 \satisfies x \pointsto 1 \land \lemp,x:1 \satisfies (x \pointsto 2 \wand Q) \\
\text{iff } & x:1,x:1 \satisfies x \pointsto 1 \land x:2,x:1 \satisfies x \pointsto 2 \land (x:2 \disju \lemp,x:1) \satisfies Q
\end{align*}
Note that the local and global heap states do not agree in the derivation of the
sub-assertions. But the local portion of the heap should always be a subset of the
global portion! Since we cannot find local and global heaps for these
sub-derivations which agree, the heap $x:1$ does not satisfy our original assertion.

This behaviour prevents us from using $\wand$ for hypothetical heap update
operations. To support this use-case, we introduce a new logical operator.

The new operator to be defined is the \emph{box-wand}, $\boxwand$ or
``hypothesising separating implication''. In addition
to the behaviour of the standard wand, the box-wand maintains consistency
between the local and global heap states whilst evaluating the satisfaction of
sub-assertions of the operator. This permits a state to be hypothesised that
conflicts with the current global state of the heap.
\[
  h,h_g \satisfies P \boxwand Q \iff \forall h' \st (h',h_g[h'] \satisfies P)
  \land h \disj h' \impl (h \disju h', h_g[h'] \satisfies Q)
\]

The derivation our example now behaves as expected, and the assertion is
satisfied.
\begin{align*}
           & x:1,x:1 \satisfies x \pointsto 1 \sep (x \pointsto 2 \boxwand Q) \\
\text{iff } & x:1,x:1 \satisfies x \pointsto 1 \land \lemp,x:1 \satisfies (x \pointsto 2 \boxwand Q) \\
\text{iff } & x:1,x:1 \satisfies x \pointsto 1 \land x:2,x:2 \satisfies x \pointsto 2 \land (x:2 \disju \lemp,x:2) \satisfies Q
\end{align*}

% TODO: Can we include this?
%The box-wand obviously does not hold the right adjoint property as a result of
%the heap update operation applied to the satisfaction rule of the $P$
%sub-assertion, which is not present in the $\sep$ rule.

\subsection{Backpointer}
\label{sec:backpointers}
We now define the operator that will primarily examine the global heap state,
the \emph{backpointer}, $E_1 \bp E_2$, which specifies that any heap cell that
has the $E_1$ as its value must appear in the set $E_2$. The set $E_2$ may
additionally contain extra heap cell references. This allows us to express the
permission that a given cell may be changed to point to $E_1$ in the future.

The assertion $E_1 \bp E_2$ means that the location $E_1$ may be pointed to by
\emph{at most} the locations in $E_2$.
\begin{align*}
  h, h_g \satisfies E_1 \bp E_2 \iff &
    \forall (l,x) \in \dom(h_g) \st h_g(l,x) = \evalle{E_1}
    \impl (l,x) \in \evalle{E_2} \land {} \\
    & h \equiv (\evalle{E_1}, @bp) \pointsto \_
\end{align*}

Note that the interaction between the backpointer operator and the consequence rule in the
Hoare reasoning is particularly useful.
For example, we can widen the backpointer set at any time necessary thanks to
the implication $l \bp S \impl l \bp S \cup S'$.

We are also able to shrink the set if it is known that the elements being
removed definitely do not point to the object in question:
\[
  a \bp \{(b,c)\} \sep (b,c) \pointsto d \sep a \not\doteq d \impl a \bp \{\} \sep (b,c) \pointsto d
\]

Although the backpointer operator would at first seem to not require any
footprint on the heap (that it would be a \emph{pure} operator), the operator's
interactions with the spatial elements of the logic may result in unsoundness.

For example, consider the following \emph{broken} proof outline:

\begin{figure}[h!]
\[
\begin{array}{l}
  \{\js{y}\pointsto\_ \sep \js{x} \pointsto a \land a \bp \{\js{x}\} \}\\
  \text{[Consequence]}\\
  \{(\js{y}\pointsto\_ \sep \js{x} \pointsto a \land a \bp \{\js{x}\}) \sep (a \bp \{\js{x}\} \land \emp) \}\\
  \text{[Frame off]}\\
  \{\js{y}\pointsto\_ \sep \js{x} \pointsto a \land a \bp \{\js{x}\} \}\\
  \text{[Consequence]}\\
  \{\js{y}\pointsto\_ \sep \js{x} \pointsto a \land a \bp \{\js{x},\js{y}\} \}\\
  \js{y = x;}\\
  \{\js{y}\pointsto a \sep \js{x} \pointsto a \land a \bp \{\js{x},\js{y}\} \}\\
  \text{[Frame on]}\\
  \{(\js{y}\pointsto a \sep \js{x} \pointsto a \land a \bp \{\js{x},\js{y}\}) \sep (a \bp \{\js{x}\} \land \emp) \}\\
  \text{[Consequence]}\\
  \{\js{y}\pointsto a \sep \js{x} \pointsto a \land a \bp \{\js{x}\} \}\\
\end{array}
\]
\caption{Broken proof showing requirement for $\bp$ to have a footprint}
\label{fig:proof-bp-broken}
\end{figure}

Notice that in this \emph{broken} program proof, we appear to have proved that
our program ensures that both \js{y} and \js{x} point to $a$ and \emph{also}
that the only location pointing to $a$ is \js{x}. To prevent these sorts of
shenanigans, we insist that two backpointer assertions about the same location
cannot be separated. The mechanism we use to enforce this requirement is a
notional internal field $@bp$ on each SES object. This field is the footprint of
any backpointer assertions about that object.

\section{Hoare Triples}
Recall Section~\ref{subsec:programreasoning} where Hoare reasoning was introduced, in
this section, we extend Hoare reasoning to support our extensions to
separation logic and we define inference rules for the SES language.

\subsection{Definition}
We define the syntax $\tr{P}{\js{e}}{Q}$ to be a Hoare triple, where $P$ is the
precondition, $\js{e}$ is the subexpression, and $Q$ is the
postcondition that results from evaluating $\js{e}$ in a state that satisfies $P$.

It is necessary that we relate the Hoare triples to the Operational Semantics of
the language defined earlier.

The first of these properties is the soundness property, it specifies that if
the command \js{e} is evaluated in a heap $h$, such that $h$ is the footprint of
and satisfies the expression's precondition $P$; and it is a subheap of the
entire heap $h \disju h_f$, then the evaluation will not fault and if it
terminates, will terminate with a subheap $h'$ that satisfies the expression's
postcondition $Q$. Or, more simply:
\[ (h, h \disju h_f, L, (\env\setminus\rv) \satisfies P) \land h, L, \js{e} \leadsto h', v
  \impl (h', h' \disju h_f, L, [\env|\rv\takes v] \satisfies Q) \]

As we are using the fault-avoiding Hoare triple semantics, the expression must
not fault if it is evaluated in a state that satisfies the triple's
precondition. We must also define a fault avoidance property to be able to
verify that this is the case:
\[ h, h_g, L, (\env\setminus\rv) \satisfies P \land h\subseteq h_g \impl h, L, \js{e} \not\leadsto \fault \]

We also define the Safety Monotonicity and Frame properties, which guarantee
that the operational semantics and separation logic reasoning framework are
consistent with regard to the spatial reasoning about the heap.

Safety Monotonicity states that if an expression evaluates without fault from a
state that satisfies the precondition, then disjointly extending the heap will
not cause the expression to fault.
\[ (h, h_g, L, \env \satisfies P) \land h \disj h' \land h, L, \js{e} \not\leadsto
  \fault \land h\subseteq h_g \impl h \disju h', L, \js{e} \not\leadsto \fault \]

The Frame property states that if an expression is evaluated from a state with
heap $h$ without faulting to produce $h_2$, then the same expression is
evaluated from the same state but with the heap $h \disju h'$, to produce the
heap $h'_2$, then the second evaluation will not have touched any part of $h'$,
namely that $h'_2 = h_2 \land h'$:
\[ (h, h_g, L, \env \satisfies P) \land h, L, \js{e} \not\leadsto \fault \land
  h \disju h', L, \js{e} \leadsto h_2' \land h\subseteq h_g\impl h, L, \js{e} \leadsto h_2 \land
  h_2' = h_2 \disju h' \]

Ideally, we would prove that each of these properties hold for each pair of the
operational semantics and program logic rules.
These results generally provide little new intuition about the
semantics or logic, so this exercise has been omitted from this project, in
favour of more targeted real-world examples.

\subsection{Inference Rules}
\label{sec:infrules}
Inference rules for Hoare triples are similar in nature to those for operational
semantics. Derivation of the pre- and post-conditions for the Hoare triple of a
given expression usually requires the recursive derivation of its
subexpressions, along with some structural operations on their intermediary
assertions to build the overall pre- and post-conditions.

The compositional nature (and the power of the Frame rule) of Separation
logic is clear from the structure
of the inference rules, only the information required to prove the particular
operation is required -- all assertions required to satisfy subexpressions are
transparently passed through to them and assertions that are not required for a
particular sub-derivation can be hidden by the frame rule.

The benefits of separation logic over the operational semantics are also clear
here -- the pre- and post-conditions of a given expression will state clearly
what the expression will access.

As with the operational semantics section of the project, the Inference Rules
for the SES language are necessarily based upon the JS inference rules produced
in~\cite{gms-popl}. However, here many more adaptions were required to
introduce the backpointer assertions required by the logic to track assignments.

It is useful to note that the Operational Semantics rules need not necessarily
map to Hoare triple inference rules. Indeed, in the case of the while statement,
the two operational semantics rules for the true/false cases are mapped down to
the single Hoare triple inference rule.

All of the inference rules for Structural Expressions listed in
Section~\ref{subsec:structuralexpressions} and the standard Separation Logic
axioms of Frame, Elimination, Consequence and Disjunction are used unmodified.

I will discuss the Binary Operator inference rule in detail to explain how to
interpret some of the techniques used in the these rules.

\subsubsection{How to Read Hoare Triple Inference Rules}
  \staterule{(Binary Operator)}
    {
      \tr P {\js{e_1}} {R \sep \rv \doteq V_1} \quad R = S_1 \sep \getValue(Ls_1,
      V_1, V_3) \\
      \tr R {\js{e_2}} {Q \sep \rv \doteq V_2} \quad Q = S_2 \sep \getValue(Ls_2,
      V_2, V_4) \\
      V = V_3 \mathbin{\bar\oplus} V_4
    }
    {\tr P {\js{e_1} \oplus \js{e_2}} {Q \sep \rv \doteq V}}

It is first useful to note the overall `flow' of the assertions, here, $P$ is
the overall precondition as well as the precondition to \js{e_1}, whose
post-condition is the pre-condition to \js{e_2}, whose post-condition is also
that of the entire expression.

We next look at the return value for each sub-expression, $\rv$. It is presented
as being extra to $Q$ and $R$, both so that its value may be used, and so that
is is not passed into the pre-condition of the next sub-expression.

The structure of the intermediary assertion $R$ is stated as having the
form $S_1 \sep \getValue(Ls_1, V_1, V_3)$, which has the meaning `get value', it
must be a part of $R$, since the derivation for \js{e_1} may use it (or even
produce it) and, if not, it will be passed through from $P$ by the frame rule.

The $\getValue$ predicate has been lifted to the logic from the semantics, the
definition is unchanged from~\cite{gms-popl} and is given in
Appendix~\ref{app:logicauxpreds}.

Finally, from the retrieved values, $V_3$ and $V_4$, $V$ is calculated using the
operator $\bar\oplus$ (lifted to the logic), and the value set as the return
value for the expression.

\subsubsection{Heap-based Expressions}
\staterule{(Variable Resolution)}
  {P = \scope(Ls_1, \ls, \js{x}, L) \sepish
   \getValue(Ls_2, \fld{L}{\js{x}}, V) \sep L \not\doteq \nil}
  {\tr P {\js{x}} {P \sep \rv \doteq \fld{L}{\js{x}}}}

  This rule searches the current scope chain, $\ls$ for the variable \js{x},
  placing all searched heap cells into the assertion. In addition to the
  counterpart operational semantics rule, permission to get the value of the
  found variable must also be specified as \emph{partially} disjoint to the
  scope chain search, as both make reference to the same target heap cell.

  The returned reference is required to be not on the \nil object by the SES
  language, this requirement is not enforced by JS, as discussed in
  Section~\ref{sec:opsem:heapexprs}.

\staterule{(Member access)}
  {\tr P {\js{e}} {Q \sep \rv \doteq V} \quad Q = R \sep \getValue(Ls, V, L)
    \sep L \not\doteq \nil \sep L\dotin\loc}
  {\tr P {\js{e.x}} {Q \sep \rv \doteq \fld{L}{\js{x}}}}

\staterule{(Computed member access)}
  {
    \tr {P} {\js{e_1}} {R \sep \rv \doteq V_1} \quad R = S_1 \sep
    \getValue(Ls_1, V_1, L) \sep L \not\doteq \nil \sep L\dotin\loc\\
    \tr {R} {\js{e_2}} {Q \sep X \dotin \uvars \sep \rv \doteq V_2}
    \quad Q = S_2 \sep \getValue(Ls_2, V_2, X)
  }
  {\tr {P} {\js{e_1[e_2]}} {Q \sep \rv \doteq \fld{L}{X}}}

  Member access and Computed member access rules are the same as for JS. The
  checks for value type being a location or a user-definable field
  ($L \dotin \loc$ and $X \dotin \uvars$) are implicit in
  the operational semantics, but explicit in the logic.

  \staterule{(Assign)}
    {
      \tr P {\js{e_1}} {R \sep \rv \doteq \fld{L}{X}} \\
      \tr R {\js{e_2}} {Q \sep (L,X) \pointsto V_3 \sep
        \bpGen(V_2, s) \sep \rv \doteq V_1} \\
      Q = S \sep \getValue(Ls, V_1, V_2) \sep \ReadWrite(L)
    }
    {\tr P {\js{e_1 = e_2}} {Q \sep (L,X) \pointsto V_2 \sep \bpGen(V_2, L, X,
    s) \sep \rv \doteq V_2}}

    The assignment rule has been considerably simplified as a result of SES not
    permitting assignments to undeclared variables. JS had two rules for this
    case, one to handle assignment to the global object in the case where (\nil,
    X) was the target field, the other to handle the regular assignment, as
    shown above.

    To support the new SES \js{freeze()} command, we introduce a check that the
    object being assigned to is readable using the $\ReadWrite$ predicate:
    \[  \ReadWrite(L) \triangleq (L,@frozen) \pointsto \false  \]

    Where it does not harm readability, it is good to have inference rules that
    are as general as possible. It is often the case in the inference rules that
    we do not know what type will be present in a heap cell. The $\bp$ operator
    is defined over only heap locations, and not pure values, so it would be
    incorrect to use the $\bp$ operator with any value that we cannot guarantee
    must be a location. Given that it is arduous to introduce multiple inference
    rules to handle these cases (and impossible when an unknown number of
    subexpressions are being handled), we instead introduce the $\bpGen$
    predicate, which has the purpose of being a type-agnostic version of the
    $\bp$ operator:
    \begin{align*}
      \bpGen(V,\_) & \triangleq V \notdotin \loc \\
      \bpGen(V,s) & \triangleq  V \dotin \loc \sep V \bp s \\
    \end{align*}
    We additionally also define a 4-ary version to simplify the oft-used
    extension of the backpointer set $s$ by the heap cell $(L,x)$:
    \begin{align*}
      \bpGen(V,\_,\_,\_) & \triangleq  V \notdotin \loc \\
      \bpGen(V,L,x,s) & \triangleq  V \dotin \loc \sep V \bp \{(L,x)\} \cup s
    \end{align*}

    Note that the assign rule could also be written with a single, 4-ary
    instance of $\bpGen$ as part of the definition of $Q$
    This alternative definition relies on the backpointer set
    widening consequence/implication as discussed in
    Section~\ref{sec:backpointers}. However, it is the author's preference to
    give the explicit widenings in the rule definition.

  \staterule{(Object creation)}
    {
      \forall i \in 1..n \st \left(\begin{array}{l}
        P_i = R_i \sep \getValue(Ls_i, Y_i, X_i) \sep
          \bpGen(X_i,s_i) \\
        \tr {P_{i-1}} {\js{e_i}} {P_i \sep \rv \doteq Y_i} \\
      \end{array}\right) \\
      P_n = R \sep \lop \bp s_{op} \\
      Q = R \sep
      \exists L \st \left(\begin{array}{l}
        \newobj_L(@proto, \js{x_1},...,\js{x_n}) \sep {} \\
        \bigsep_{1 \leq i \leq n} (
          (L, \js{x_i}) \pointsto X_i \sep \bpGen(X_i, L,\js{x_i}, s_i)
        ) \sep {} \\
        (L,@proto) \pointsto \lop \sep \lop \bp s_{op} \cup \{(L,@proto)\} \sep {} \\
        \rv \doteq L \sep L \bp \{\} \\
      \end{array}\right) \\
      \js{x_1} \neq \dots \neq \js{x_n} \qquad \rv \not\in \fv(P_n)
    }
    {\tr {P_0} {\js{\{x_1:e_1, ..., x_n:e_n\}}} Q}
    
    This rule traverses the sub-expressions in the construction in left-to-right
    order, collecting the required pre-conditions and post-conditions to
    satisfy all expressions. For each sub-expression, the return value and
    backpointer set for each object that is returned is stored for the
    assignment.

    The object is then created, assigning the values stored in the first pass,
    and updating their backpointer sets as necessary.
    The prototype of the object is set to the language constant $\lop$ (the
    object prototype) and
    its backpointer set updated appropriately. Finally, a backpointer assertion
    for the newly created object is produced for the new object.

    This rule is sound, but is known not to be complete -- it fails for
    the creation of objects which assign the same location to multiple fields.
    Further work is needed to fix this.
  
  \subsubsection{Function-related Expressions}
  \staterule{(Function)}
  {
    P = \lop \bp s_1 \sep \lfp \bp s_2 \sep \scopeBps(\ls, ss) \\
      Q = \exists L_1,L_2 \st \left(\begin{array}{l}
        % L_1 - prototype object
        \fullobj_{L_1}(@proto:\lop) \sep L_1 \bp \{(L_2, \js{prototype})\} \sep
          \lop \bp s_1 \cup \{(L_1,@proto)\}  \sep {} \\
        % L_2 - function
        \newfun_{L_2}(\ls, \js{x}, \js{e}, L_1) \sep L_2 \bp \{\} \sep
          \lfp \bp s_2 \cup \{(L_2,@proto)\} \sep {} \\
        % Other bits
        \rv \doteq L_2 \sep
        \scopeBpsUpd(\ls, ss, ss', \{(L_2, @scope)\})
    \end{array}\right)
  }
  {\tr P {\jsfun{x}{e}} Q}

  \staterule{(Named Function)}
    {
      P = \lop \bp s_1 \sep \lfp \bp s_2 \sep \scopeBps(\ls, ss)  \\
      Q = \exists L_1, L_2, L_3 \st \left(\begin{array}{l}
        % L_1 - prototype object
        \fullobj_{L_1}(@proto:\lop) \sep L_1 \bp \{(L_2, \js{prototype})\} \sep
          \lop \bp s_1 \cup \{(L_1,@proto)\}  \sep {} \\
        % L_2 - function
        \newfun_{L_2}((L_3:\ls), \js{x}, \js{e}, L_1) \sep
          L_2 \bp \{(L_3,\js{y})\} \sep
          \lfp \bp s_2 \cup \{(L_2,@proto)\} \sep {} \\
        % L_3 - new scope object
        \fullobj_{L_3}(@proto:\nil,\js{y}:L_2) \sep L_3 \bp \{(L_2,@scope)\} \sep {} \\
        % Other bits
        \rv \doteq L_2 \sep
        \scopeBpsUpd(\ls, ss, ss', \{(L_2, @scope)\})
      \end{array}\right)
    }
    {\tr P {\jsfun[y]{x}{e}} Q}

    Prior to the extension of the logic, both these functions required just
    $\lemp$ for their preconditions, they now require 3 backpointer assertions,
    the first two are for the language constants $\lop$ and $\lfp$ (the object
    and function prototypes). The third of these assertions, $\scopeBps$
    fetches the backpointer sets for each of the objects in the scope chain,
    $\ls$, and places them in the list $ss$ of backpointer sets.

    The post-conditions for these commands construct 2 or 3 objects: the
    function itself ($L_2$); a prototype object ($L_1$); and, for the named
    variant, a scope record ($L_3$) binding the function's name to itself for
    use in recursive calls.
    Fresh backpointer assertions are made for each of these new objects.

    Since functions store the scope in which they are created the set of objects
    in the scope chain also require their
    backpointer sets updating. This is because the body of the function, and
    thus the function itself now potentially provides a means of accessing those
    objects. The $\scopeBpsUpd$ predicate uses the $ss$ list stored by the
    $\scopeBps$ predicate in the precondition to achieve this.
    \begin{align*}
      \scopeBpsUpd(Ls, ss, ss', n) & \triangleq \bigsep_{0\leq i<\length(Ls)}(
        \lstitem(i,Ls) \bp \lstitem(i,ss') \sep \lstitem(i,ss') \doteq
        \lstitem(i,ss)
        \cup n) \\
      \scopeBps(Ls, ss) & \triangleq \scopeBpsUpd(Ls, ss, ss, \{\})
    \end{align*}
    The following equivalence for the $\scopeBpsUpd$ predicate is useful to note
    for use in proofs:
    \[
      \scopeBpsUpd(L:Ls, s:ss, (s \cup n):ss', n) \iff L \bp s \cup n \sep
      \scopeBpsUpd(Ls, ss, ss', n)
    \]

  \staterule{(Function Call)}
    {
      \tr P {\js{e_1}} {R_1 \sep \rv \doteq F_1} \\
      R_1 = \left(\begin{array}{l}
          S_1 \sepish \pickThis(F_1, T) \sepish \getValue(Ls_1, F_1, F_2) \sep {} \\
          (F_2, @body) \pointsto \lambda X.\js{e_3} \sep (F_2, @scope) \pointsto
          Ls_2
      \end{array}\right) \\
      \tr {R_1} {\js{e_2}} {R_2 \sep \bpGen(T,s) \sep \bpGen(V_2,s') \sep
        \ls \doteq Ls_3 \sep \rv \doteq V_1} \\
      R_2 = S_2 \sep \getValue(Ls_4, V_1, V_2) \\
      R_3 = R_2 \sep \exists L \st \left(\begin{array}{l}
          (L, X) \pointsto V_2 \sep \bpGen(V_2, L, X, s') \sep {} \\
          (L, @this) \pointsto T \sep \bpGen(T,L,@this,s) \sep
          (L, @proto) \pointsto \nil \sep {} \\
          \defs(X,L,\js{e_3}) \sep
          \newobj_L(@proto,@this,\js{x},\vardecls(X, L, \js{e_3})) \sep {} \\
          L \bp \{\} \sep
          \ls \doteq L:Ls_2
      \end{array}\right) \\
      \tr {R_3} {\js{e_3}} {\exists L \st Q \sep \ls \doteq L:Ls_2} \qquad
      \ls \notin \fv(Q) \cup \fv(R_2)
    }
    {\tr P {\js{e_1(e_2)}} {\exists L \st Q \sep \ls \doteq Ls_3}}

    The function call rule is largely unchanged from the JS version, changes are
    made to introduce backpointer set updates for the object assigned to the
    @this internal variable, the parameter of the function, and created for the
    activation record object.

  \staterule{(This)}
  { P = \scope(Ls_1, \ls, @this, L_1) \sepish \proto(Ls_2, L_1, @this, L_2)
    \sepish (L_2,@this) \pointsto V }
  {\tr P {\js{this}} {P \sep \rv \doteq V}}

  The This rule is the same as in the JS language. The object that this is bound
  to is determined by the value of the internal @this variable, set on
  function activation records and the object at the bottom of the scope chain.

  \subsubsection{SES-specific Expressions}
  \staterule{(Restricted evaluation)}
    {
      \tr{P}{\js{e_1}}{R_1 \sep \rv \doteq V_1} \\
      R_1 = S_1 \sep \getValue(\_, V_1, V_2) \sep V_2 \dotin \Strings \\
      \parse(V_2) = \js{e_3} \\
      \tr{R_1}{\js{e_2}}{R_2 \sep \rv \doteq V_3 \sep \ls \doteq Ls} \\
      R_2 = \left(\begin{array}{l}
        S_2 \sep \getValue(\_, V_3, V_4) \sep V_4 \dotin \loc \sep V_4 \bp s
        \cup \{(L,@this)\} \sep {} \\
        \exists L \st R_3 \sep
        \newobj_L(@proto,@this,\vardecls(\_,L,\js{e_3})) \sep {} \\
        \obj_L(@this: V_4, @proto: \nil) \sep \defs(\_,L,\js{e_3}) \\
      \end{array}\right) \\
      R_3 = (\ls \doteq L:[V_4]) \\
      \tr{R_2}{\js{e_3}}{\exists L \st Q \sep R_3}
    }
    {\tr{P}{\js{reval(e_1,e_2)}}{\exists L \st Q \sep \ls \doteq Ls}}

    This is a new definition. The first subexpression, the untrusted code, is
    resolved to a string
    which is parsed into a suitable form for analysis (details out of
    scope of this formalisation). The second subexpression is resolved to an
    object, which is used as the base object for a new scope chain. The
    untrusted code is then executed within the context of the new scope chain.
    Its pre- and post-conditions are incorporated into those of the whole
    expression.

    We assume that a Hoare triple for the untrusted code is available, or
    derivable. If not, an arbitrary triple could be substituted, say one that is
    designed to be \emph{maximally malicious}. If the security of the restricted
    evaluation of this program can be proven, then all programs run in this
    context should be safe.

    It is the case that the object resolved by the second subexpression of
    a call to restricted evaluation should be considered security critical. It
    is reachability from this object to privileged objects that should be
    checked by axioms that verify security properties.

  \staterule{(Freeze)}
    {
      \tr{P}{\js{e}}{Q \sep \rv \doteq V_1 \sep (V_2, @frozen) \pointsto V_3} \\
      Q = \getValue(Ls,V_1,V_2) \sep S
    }
    {\tr{P}{\js{freeze(e)}}{Q \sep (V_2, @frozen) \pointsto \true \sep \rv \doteq V_2}}

    The rule for Freeze is straightforward, it simply requires permission to the
    @frozen internal field of the object being frozen, and asserts that it is
    set to true in the postcondition.

  \staterule{(Recursive freeze)}
    {
      \tr{P}{\js{e}}{Q \sep \rv \doteq V_2 \sep \auxDefGet(V_2, \{\})} \\
      Q = \getValue(Ls,V_1,V_2) \sep S
    }
    {\tr{P}{\js{def(e)}}{Q \sep \rv \doteq V_2 \sep \auxDefSet(V_2, \{\}, \true)}}

  Similar to the freeze expression, the recursive freeze requires permission to
  access the @frozen field on accessible objects. It achieves this through use
  of the $\auxDefGet$ and $\auxDefSet$ predicates:
  \begin{align*}
    \auxDefGet(V,s) & \triangleq V \notdotin \loc \lor V \dotin s \\
    \auxDefGet(V,s) & \triangleq V \notdotin s \land (V, @frozen) \pointsto \_
      \sep (\bigsepish_{x_n \in \uvars} (V,x_n) \pointsto V' \sepish
      \auxDefGet(V', s\cup\{V\})) \\
    \auxDefSet(V,s,b) & \triangleq V \notdotin \loc \lor V \dotin s \\
    \auxDefSet(V,s,b) & \triangleq V \notdotin s \land (V, @frozen) \pointsto b
      \sep (\bigsepish_{x_n \in \uvars} (V,x_n) \pointsto V' \sepish
      \auxDefSet(V', s\cup\{V\}, b))
  \end{align*}
  These predicates traverse the user-accessible fields of the given object, $V$,
  recursively. In the case of the $\auxDefGet$ predicate, to assert that the
  @frozen field is present in the footprint of the precondition, and in the case
  of $\auxDefSet$ that the @frozen field is set to $b$.

  Termination of the predicate is ensured by collecting traversed objects into
  the set $s$ for the recursive calls, the recursive definition cannot be
  satisfied if the same location is encountered a second time.

  The sepish operator, $\sepish$, must be used to collect the heap cells into a
  single assertion. Although we guarantee to terminate on the second traversal
  of a loop, we must also take into account heap structures that have the same
  object reachable via different paths, for example the object \js{\{a:l,b:l\}}
  would be unsatisfiable under the predicates if $\sep$ were used, since they
  are not guaranteed to be disjoint. Use of
  $\sepish$ is sound in the update case, since the updates are consistent, the
  same value is written to all the overlapping heap cells.

\chapter{Program Proofs}
\label{chap:proofs}
Some design patterns for programming defensively against malicious code of an
unknown nature have been developed for use with the Restricted Evaluation
command. We have proven the correctness of the security properties of these
design patterns using the logic developed in this project.

Both these patterns are designed to wrap security-critical interfaces or objects
before being passed into untrusted code. They provide a means of revoking access
to those objects when desired by the trusted portion of the code.

Both patterns are capable of being extended to more situation-specific wrappers,
for example to deny access to a particular function of an object, or to censor
sensitive data as it is passed between contexts.

All of these patterns operate by encapsulating the resources to be protected
within the scope of a function. Since the scope is inaccessible by
programmatic means from outside the body of the function, the function can be
passed to be untrusted code and the reference remain safe. This relies on the
correct operation of the function to not disclose the reference by mistake.

These design patterns lie at the boundary between trusted and untrusted
code, their verification is critical to guaranteeing security of any program
that wishes to share data and privileged functions with untrusted code.

We also demonstrate two styles of proofs, the first demonstrating the logic as
being extremely precise with a symbolic execution style of assertion generation, the
second showing that the logic is also able to provide more succinct proofs that
highlight only the key aspects of evaluation.

\section{Caretaker}
The Caretaker, or Revocable Reference pattern is a rudimentary pattern that
demonstrates the ability to encapsulate a reference within a function. The
implementation is shown in Figure~\ref{fig:code:caretaker}.

\begin{figure}[h!]
  \label{fig:code:caretaker}
\begin{verbatim}var RevocableRef = function(ref) {
  var protected = ref;
  var access = function(field) { return protected[field]; }
  var kill = function() { protected = null; }
  return {access: access, kill: kill};
}\end{verbatim}
  \caption{Implementation of the Caretaker pattern}
\end{figure}

When invoked by the trusted code with an object to protect, two functions are
returned, an \js{access} function and a \js{kill} function. The trusted code keeps the
\js{kill} function to call when the reference is to be revoked. The \js{access} function
is passed to the untrusted code.
The \js{access} function permits fields of the protected object to be accessed until
the Caretaker is killed, at which point it must stop functioning.

The only guarantee that the Caretaker pattern provides is that the \js{access}
function will not succeed if the reference has been revoked.
The Caretaker pattern provides no guarantees for return values from
the protected object, it is therefore not suitable for use with objects that may
potentially return bare references into the trusted code.
For example, a function that returns itself in response to a
particular call would trivially violate the security of the Caretaker.

We have derived assertions for the Caretaker pattern through application of the
inference rules to the expressions in a mechanical style (shown in
Figures~\ref{rr-main}-\ref{rr-access}). This approach produces
a significant number of assertions, of which only a few are relevant to the
proof of the trivial security property. Nonetheless, the derivation is a
useful demonstration of how the sets of backpointer references are built up by
the inference rules. 

The derivation proves that once killed, the Caretaker's \js{access} function will no
longer function, the post-condition of \js{kill} includes $(L, \js{protected})
\pointsto \nil$. The pre-condition of \js{access} requires that $(L,
\js{protected}) \pointsto P \sep P \not\doteq \nil$. Finally, we note that from
the postcondition of \js{RevocableRef} that the only references to $L$ are
$(A, @scope)$ and $(K, @scope)$, namely the externally-inaccessible scope chains
of the \js{access} and \js{kill} functions. We have already verified that these
functions are correct, so the property holds.

\begin{adjustbox}{width=\textwidth,height=\textheight,keepaspectratio,rotate=0,caption=Caretaker main body proof,label=rr-main,figure=p}
  $\begin{array}{l}
    \jsvar{RevocableRef = function(ref) \{}\\
    \indblock{
      \logic{
        % function
        \obj_R(@body: \lambda\js{ref}.\{\dots\} , @scope: Ls) \sep {} \\
        % ar
          \exists L \st \ls \doteq L:Ls \sep {} \\
          \fullobj_L\left(
            \begin{array}{l}
              \js{ref}: V,@this: \_,@proto: \nil,@frozen: \false,\\
              \js{protected}: \und,\js{access}: \und,\js{kill}: \und %  and defs()
            \end{array}
          \right) \sep {} \\
        % bp
          L \bp \{\} \sep V \bp s_1 \cup \{(L,\js{ref})\} \sep
          \lop \bp s_3 \sep \lfp \bp s_4 \sep
          \scopeBps(Ls, ss)
      } \\
      \step{Frame/elim/cons ($V \bp s_1 \impl V \bp s_1 \cup s_2$)} \\
      \logic{
        % var(protected,ref), assign(@frozen)
          \obj_L(\js{protected}: \und, \js{ref}: V, @frozen: \false) \sep {} \\
        % assign
          V \bp s_1 \cup \{(L,\js{ref}), (L, \js{protected})\} \sep \ls\doteq{L:Ls}
      } \\
      \jsvar{protected = ref;} \\
      \logic{
        % var *2
          \obj_L(\js{protected}: V, \js{ref}: V, @frozen: \false) \sep {} \\
        % assign
          V \bp s_1 \cup \{(L,\js{ref}), (L, \js{protected})\}\sep \ls\doteq{L:Ls}
      } \\
      \step{Frames/cons} \\
      \logic{
        % var(access), assign(@frozen)
          \obj_L(\js{access}: \und, @frozen: \false) \sep {} \\
        % fun pre
        \lop \bp s_3 \sep \lfp \bp s_4 \sep
        L \bp \{\} \sep \scopeBps(Ls,ss) \sep \ls\doteq{L:Ls}
      } \\
      \jsvar{access = function(field) \{b_a\};} \\
      \logic{
        \exists A, A_p \st \left(\begin{array}{l}
          % var, assign
          \obj_L(\js{access}: A, @frozen: \false) \sep
            A \bp s_7 \sep s_7 \doteq \{(L, \js{access})\} \sep {} \\
          % fun pre
          \lop \bp s_5 \sep s_5 \doteq s_3 \cup \{(A_p, @proto)\} \sep \lfp \bp s_6 \sep s_6 \doteq s_4
            \cup \{(A, @proto)\} \sep {} \\
            L \bp \{(A,@scope)\} \sep \scopeBpsUpd(Ls, ss, ss_1, \{(A,@scope)\}) \sep {} \\
          % fun post
          \fullobj_{A_p}(@proto: \lop) \sep A_p \bp \{(A, \js{prototype})\} \sep {}\\
          \newfun_A(\ls, \js{field}, b_a, A_p)
        \end{array}\right) \sep \ls\doteq{L:Ls}
      } \\
      \step{Frames/cons} \\
      \logic{
        % var(access), assign(@frozen)
          \obj_L(\js{kill}: \und, @frozen: \false) \sep {} \\
        % fun pre
        \lop \bp s_5 \sep \lfp \bp s_6 \sep
        L \bp \{(A, @scope)\} \sep \scopeBpsUpd(Ls, ss_1) \sep \ls\doteq{L:Ls}
      } \\
      \jsvar{kill = function() \{b_k\};} \\
      \logic{
        \exists K, K_p \st
        \left(\begin{array}{l}
          % var, assign
          \obj_L(\js{kill}: K, @frozen: \false) \sep
            K \bp s_8 \sep s_8 \doteq \{(L, \js{kill})\} \sep {} \\
          % fun pre
          \lop \bp s_9 \sep s_9 \doteq s_5 \cup \{(K_p, @proto)\} \sep \lfp \bp s_6
            \cup \{(K, @proto)\} \sep {} \\
            L \bp \{(A,@scope),(K,@scope)\} \sep {} \\
            \scopeBpsUpd(Ls, ss_1, ss_2, \{(K,@scope)\}) \sep {} \\
          % fun post
          \fullobj_{K_p}(@proto: \lop) \sep K_p \bp \{(K, \js{prototype})\} \sep {} \\
          \newfun_K(\ls, \_, b_k, K_p)
        \end{array}\right)\sep \ls\doteq{L:Ls}
      } \\
      \step{Frames} \\
      \logic{
        \obj_L(\js{access}:A, \js{kill}:K) \sep
        A \bp s_7 \sep K \bp s_8 \sep \lop \bp s_9 \sep \ls \doteq L : Ls
      } \\
      \js{return\ \{access: access, kill: kill\};} \\
      \logic{
        \exists O \st
        \left(\begin{array}{l}
          \obj_L(\js{access}:A, \js{kill}:K) \sep {} \\
          \fullobj_O(@proto: \lop, \js{access}: A, \js{kill}: K) \sep {} \\
          A \bp s_7 \cup \{(O, \js{access}\} \sep K \bp s_8 \cup \{(O, \js{kill})\}
            \sep {} \\
          \lop \bp s_9 \cup \{(O,@proto)\} \sep O \bp \{\} \sep \rv \doteq O
        \end{array}\right)
        \sep \ls \doteq L : Ls
      } \\
      \step{Frames} \\
      \logic{
        \obj_R(@body: \lambda\js{ref}.\{\dots\} , @scope: Ls) \sep {} \\
        \exists L, A, A_p, K, K_p, O \st {} \\
        \fullobj_L\left(\begin{array}{l}
          \js{ref}: V,@this: \_,@proto: \nil,@frozen: \false,\\
          \js{protected}: V,\js{access}: A,\js{kill}: K
        \end{array}\right) \sep {} \\
        \fullobj_{A_p}(@proto: \lop) \sep \fullobj_{K_p}(@proto: \lop) \sep {} \\
        \newfun_A(\ls,\js{field},b_a,A_p) \sep \newfun_K(\ls,\_,b_k,K_p) \sep {} \\
        \fullobj_O(@proto: \lop, \js{access}: A, \js{kill}: K) \sep {} \\

        \lop \bp s_3 \cup \{(A_p,@proto),(K_p,@proto),(O,@proto)\} \sep {} \\
        \lfp \bp s_4 \cup \{(A,@proto),(K,@proto)\} \sep {} \\
        A   \bp \{(L,\js{access}),(O,\js{access})\} \sep
        A_p \bp \{(A,\js{prototype}) \sep {} \\
        K   \bp \{(L,\js{kill}),(O,\js{kill})\} \sep
        K_p \bp \{(K,\js{prototype}) \sep {} \\
        V   \bp s_1 \cup \{(L,\js{ref}),(L,\js{protected})\} \sep {} \\
        \scopeBpsUpd(Ls,ss,ss_2,\{(A,@scope),(K,@scope))\}) \sep {} \\
        L   \bp \{(A,@scope),(K,@scope)\} \sep
        O   \bp \{\} \sep

        \rv \doteq O \sep \ls \doteq L : Ls
      }
    }
    \js{\}}
  \end{array}$
\end{adjustbox}

\begin{figure}
  \[
    \begin{array}{l}
      \jsvar{kill = function() \{} \\
      \indblock{
        \logic{
          (K, @body) \mapsto \lambda \_.\{b_k\} \sep (K, @scope) \mapsto L:Ls \sep {} \\
          \exists L' \st \ls \doteq L' : L : Ls \sep L' \bp \{\} \sep {} \\
          \fullobj_{L'}(@proto:\nil, @frozen:\false, @this:\_) \sep {} \\
          \obj_L(\js{protected}: V, @frozen: \false)
        } \\
        \step{Frame/exists} \\
        \logic{
          obj_{L'}(\js{protected}: \none, @proto: \nil) \sep {} \\
          obj_L(\js{protected}: V, @frozen: \false) \sep {} \\
          \ls \doteq L':L:Ls
        } \\
        \js{protected = null;} \\
        \logic{
          obj_{L'}(\js{protected}: \none, @proto: \nil) \sep {} \\
          obj_L(\js{protected}: \nil, @frozen: \false) \sep {} \\
          \ls \doteq L':L:Ls \sep \rv \doteq \nil
        } \\
        \step{Frame/exists} \\
        \logic{
          (K, @body) \mapsto \lambda \_.\{b_k\} \sep (K, @scope) \mapsto L:Ls \sep {} \\
          \exists L' \st \ls \doteq L' : L : Ls \sep L' \bp \{\} \sep {} \\
          \fullobj_{L'}(@proto:\nil, @frozen:\false, @this:\_) \sep {} \\
          \obj_L(\js{protected}: \nil, @frozen: \false) \sep \rv \doteq \nil
        }
      }
      \js{\}}
    \end{array}
  \]
  \caption{Caretaker \js{kill} function body}
  \label{rr-kill}
\end{figure}

\begin{figure}
  \[
    \begin{array}{l}
      \jsvar{access = function(field) \{} \\
      \indblock{
        \logic{
          (A, @body) \mapsto \lambda \js{field}.\{b_a\} \sep
            (A, @scope) \mapsto L:Ls \sep {} \\
          \exists L' \st \ls \doteq L' : L : Ls \sep
            L' \bp \{\} \sep \bpGen(X, L', \js{field}, s) \sep {} \\
          \fullobj_{L'}(@this: \_, @proto: \nil, @frozen: \false, \js{field}: X)
            \sep {} \\
          (L, \js{protected}) \mapsto P \sep P \not\doteq \nil \sep P \dotin \loc
            \sep X \dotin \uvars
        } \\
        \step{Frame/exists} \\
        \logic{
          \obj_{L'}(\js{protected}: \none, @proto: \nil, \js{field}: X) \sep {} \\
          (L, \js{protected}) \mapsto P \sep P \not\doteq \nil \sep P \dotin \loc
            \sep {} \\
          X \dotin \uvars \sep \ls \doteq L':L:Ls
        } \\
        \js{return\ protected[field];} \\
        \logic{
          \obj_{L'}(\js{protected}: \none, @proto: \nil, \js{field}: X) \sep {} \\
          (L, \js{protected}) \mapsto P \sep
          P \not\doteq \nil \sep P \dotin \loc \sep {} \\
          X \dotin \uvars \sep \ls \doteq L':L:Ls \sep \rv \doteq \fld{P}{X}
        } \\
        \step{Frame/exists} \\
        \logic{
          (A, @body) \mapsto \lambda \js{field}.\{b_a\} \sep
            (A, @scope) \mapsto L:Ls \sep {} \\
          \exists L' \st \ls \doteq L' : L : Ls \sep
            L' \bp \{\} \sep \bpGen(X, L', \js{field}, s) \sep {} \\
          \fullobj_{L'}(@this: \_, @proto: \nil, @frozen: \false, \js{field}: X)
            \sep {} \\
          (L, \js{protected}) \mapsto P \sep P \not\doteq \nil \sep P \dotin \loc
            \sep X \dotin \uvars \sep \rv \doteq \fld{P}{X}
        }
      }
      \js{\}}
    \end{array}
  \]
  \caption{Caretaker \js{access} function body}
  \label{rr-access}
\end{figure}

\clearpage

\section{Membrane}
\label{sec:membrane}
The Membrane pattern is an enhanced version of Caretaker, but in addition to
protecting all
interactions with a particular object, a Membrane also protects all subsequent
products of interactions with the protected object.

This is achieved by wrapping all returned objects in a Membrane Reference that
shares its `kill switch' with the parent Membrane. When the Membrane is killed,
all references to the parent object and any references that have been passed as
a result of the parent are revoked simultaneously.

A simple Membrane implementation is shown in Figure~\ref{fig:code:membrane}.

\begin{figure}[h!]
\begin{verbatim}var Membrane = function(ref) {
  var killed = false;

  var MembraneRef = function(ref) {
    var access = function(field) {
      if(!killed) return MembraneRef(ref[field]);
    }
    if(primitive(ref)) { return ref; } else { return access; }
  }

  var access = MembraneRef(ref);
  var kill = function() { killed = true; }
  return {access: access, kill: kill};
}\end{verbatim}
  \caption{Implementation of the Membrane pattern}
  \label{fig:code:membrane}
\end{figure}

In order to prove the correctness of the Membrane, we take a more principled
approach than we did for the considerably simpler Caretaker. We begin by
defining predicates that will help us simplify the desired specifications for
the Membrane functionality.

We begin by defining the assertion for an instance of a Membrane, along with the
more refined definitions of an alive and a dead membrane. These definitions
encapsulate the Membrane's activation record, $M$, on which the kill switch and
supporting functions are initially defined, one of which is the MembraneRef
constructor, $F_{MR}$, which is used recursively by the Membrane implementation.
\begin{align*}
  MembraneInstance(M, F_{MR}, F_K) & \triangleq \obj_M(\js{ref}: \_, @this: \_, @proto: \nil,
      \js{MembraneRef}: F_{MR}, \js{kill}: F_{K}) \sep {} \\
      & \phantom{{} \triangleq {}} \newfun_{F_{MR}}(M:\_, \js{ref}, \lambda_{MR}, \_) \\
  alive(M, F_{MR}, F_K) & \triangleq MembraneInstance(M, F_{MR}, F_K) \sep (M, killed) \pointsto \false \\
  dead(M, F_{MR}, F_K) & \triangleq MembraneInstance(M, F_{MR}, F_K) \sep (M, killed) \pointsto \true
\end{align*}

The kill function $F_K$ for the Membrane $M$, is simply defined as:
\[ kill(F_K, M) \triangleq \fun_{F_K}(M:\_, \_ , \lambda_K, \_) \]

We next define predicates for the MembraneRef objects, $MR$, that are created as
a result of using the Membrane, $M$. Each MembraneRef protects a potentially
different object, $r$. $F_A$ is the corresponding access function for the
MembraneRef. Finally, it is specified that the only objects that may point at
$r$ are in the sets $S$, references in the trusted code, and $T$, references
created by the Membrane.

Note that the $\newfun$ predicate cannot be used to describe $F_A$ here, since
it is passed into untrusted code. Although the untrusted code may add new fields
to the function, it will have no impact upon the proof, as the function only
accesses variables through its scope chain. In the case of additional fields
being added, the $\newobj$ part of the $\newfun$ predicate would prevent proof
of any such code.
\begin{align*}
  MRef_{MR}(M,r,F_A,S,T) &\triangleq \fun_{F_A}(MR:M:\_, \js{field}, \lambda_A, \_) \sep {} \\
         &\phantom{{} \triangleq {}} \fullobj_{MR}(\js{ref}: r, \js{access}: F_A, @this: \_, @proto: \nil) \sep
    r \bp S \cup T
\end{align*}

Finally, we must define the set of references, $T$, created by the Membrane
which are permitted to have references to protected objects. The $Ls$ parameter
is a list of $MRef$ objects constructed by the Membrane, $M$. The list $Ls$ is
updated in the specification of \js{MembraneRef}.
\begin{align*}
  MRset_M([], \{(M, \js{ref})\}) &\triangleq \lemp \\
  MRset_M(L:Ls, Ts') &\triangleq MRset_M(Ls, Ts) \sep Ts' \doteq Ts \cup \{(L, \js{ref})\}
\end{align*}

We now specify the properties required of Membrane and its associated functions.
The \js{Membrane} function should take a reference, and produce an alive
Membrane, with a kill function, an access function for the reference, and return
these in a simple object.
\[
  \begin{array}{l}
    \logic{
      (\_, \js{ref}) \pointsto ref \sep
      ref \bp S \sep
      ref \dotin \loc \sep
      \ls \doteq Ls
    } \\
    \js{Membrane(ref);} \\
    \logic{
      (\_, ref) \pointsto ref \sep
      ref \dotin \loc \sep
      \ls \doteq Ls \sep {} \\
      \exists M, F_K, F_A, F_{MR}, MR, L \st \args{
        alive(M, F_{MR}, F_K) \sep
        kill(F_K,M) \sep
        MRef_{MR}(M, ref, F_A, S, T) \sep {} \\
        MRset([MR],T) \sep
        \fullobj_L(\js{access}:F_A,\js{kill}:F_K) \sep
        \rv \doteq L
      }
    }
  \end{array}
\]
The kill function specification is self-explanatory.
\[
  \begin{array}{l}
    \logic{
      (\_,\js{kill}) \pointsto F_K \sep kill(F_K,M) \sep (dead(M, F_{MR}, F_K)
      \lor alive(M, F_{MR}, F_K))
    } \\
    \js{kill();} \\
    \logic{
      (\_,\js{kill}) \pointsto F_K \sep kill(F_K,M) \sep dead(M, F_{MR}, F_K)
    } \\
  \end{array}
\]
The internal MembraneRef function takes a Membrane, the set of objects
constructed by the membrane, and a reference to be protected. It leaves the
Membrane unchanged, constructs a new MRef object, extends the list of MRef
objects constructed, and returns the MRef's associated access function.
\[
  \begin{array}{l}
    \logic{
      (\_,\js{MembraneRef}) \pointsto F_{MR} \sep
      alive(M, F_{MR}, F_K) \sep
      MRset_M(Ls,T) \sep {} \\
      (\_, \js{ref}) \pointsto ref \sep
      ref \dotin \loc \sep
      ref \bp S \cup T
    } \\
    \tjs{MembraneRef(ref);} \\
    \logic{
      (\_,\js{MembraneRef}) \pointsto F_{MR} \sep
      alive(M, F_{MR}, F_K) \sep {} \\
      \exists R, F_A \st \args{
        MRset_M(R:Ls,T') \sep
        MRef_{R}(M,ref,F_A,S,T') \sep \rv \doteq F_A
      } \sep {} \\
      (\_, \js{ref}) \pointsto ref \sep
      ref \dotin \loc
    } \\
  \end{array}
\]
The access function is inherently associated with an MRef object, which it must
have permission to. The field being accessed must be a user-accessible field on
the object protected by the MRef and it must point to an object.
\[
  \begin{array}{l}
    \logic{
      (\_,\js{access}) \pointsto F_{A_1} \sep
      alive(M, F_{MR}, F_K) \sep
      MRef_{R_1}(M,ref,F_{A_1},S_1,T_1) \sep
      MRset(Ls,T_1) \sep {} \\

      (\_, \js{field}) \pointsto F \sep F \dotin \uvars \sep
      \getValue(\_, \fld{ref}{F}, V) \sep
      V \dotin \loc \sep
      V \bp S_2
    } \\
    \js{access(field);} \\
    \logic{
      (\_, access) \pointsto F_{A_1} \sep
      alive(M, F_{MR}, F_K) \sep
      MRef_{R_1}(M,ref,F_{A_1},S_1,T_2) \sep {} \\
      \exists R_2, F_{A_2} \st \args{
        MRset(R_2:Ls,T_2) \sep
        MRef_{R_2}(M,V,F_{A_2},S_2,T_2) \sep
        \rv \doteq F_{A_2}
      } \sep {} \\
      (\_, \js{field}) \pointsto F \sep F \dotin \uvars \sep
      \getValue(\_, \fld{ref}{F}, V) \sep
      V \dotin \loc
    }
  \end{array}
\]

Proof outlines for these specifications are shown in
figures~\ref{fig:proof:membrane} and \ref{fig:proof:membraneref}.

\begin{figure}
  \[
    \begin{array}{l}
      \tjs{var Membrane = function(ref) \{} \\
        \indblock{
          \logic{
            \fullobj_M \args{
              \js{ref}: ref, @this: \_, @proto: \nil, \js{killed}: \und, \\
              \js{MembraneRef}: \und, \js{access}: \und, \js{kill}: \und
            } \sep {} \\
            MRset_M([], T) \sep
            ref \bp S \cup T \sep
            ref \dotin \loc \sep
            \ls \doteq M:Ls
          } \\
          \tjs{var killed = false;} \\
          \tjs{var MembraneRef = function(ref) \{...\}} \\
          \tjs{var kill = function() \{...\}} \\
          \logic{
            alive(M, F_{MR}, F_K) \sep
            \newobj_M(\js{ref},@this,@proto,\js{killed},\js{MembraneRef},
                   \js{access},\js{kill}) \sep {} \\

            kill(F_K,M) \sep
            (M,\js{access}) \pointsto \und \sep {} \\

            MRset_M([], T) \sep
            ref \bp S \cup T \sep
            ref \dotin \loc \sep
            \ls \doteq M:Ls
          } \\
          \tjs{var access = MembraneRef(ref);} \\
          \logic{
            alive(M, F_{MR}, F_K) \sep
            \newobj_M(\js{ref},@this,@proto,\js{killed},\js{MembraneRef},
                   \js{access},\js{kill}) \sep {} \\

            kill(F_K,M) \sep
            (M,\js{access}) \pointsto F_A \sep {} \\

            MRset_M([MR], T') \sep
            MRef_{MR}(M,ref,F_A,S,T') \sep
            ref \dotin \loc \sep
            \ls \doteq M:Ls
          } \\
          \tjs{return \{access: access, kill: kill\};} \\
          \logic{
            alive(M, F_{MR}, F_K) \sep
            \newobj_M(\js{ref},@this,@proto,\js{killed},\js{MembraneRef},
                   \js{access},\js{kill}) \sep {} \\

            kill(F_K,M) \sep
            (M,\js{access}) \pointsto F_A \sep {} \\

            MRset_M([MR], T') \sep
            MRef_{MR}(M,ref,F_A,S,T') \sep
            ref \dotin \loc \sep {} \\

            \fullobj_L(\js{access}:F_A,\js{kill}:F_K) \sep
            \ls \doteq M:Ls \sep
            \rv \doteq L
          }
        }
        \tjs{\}} \\
        \\
        \tjs{var kill = function() \{} \\
        \indblock{
          \logic{
            \exists A \st \ls \doteq A : M : \_ \sep
            \fullobj_A(@proto: \nil, @this: \_) \sep
            (alive(M, F_{MR}, F_K) \lor dead(M, F_{MR},F_K))
          } \\
          \tjs{killed = true;} \\
          \logic{
            \exists A \st \ls \doteq A : M : \_ \sep
            \fullobj_A(@proto: \nil, @this: \_) \sep
            dead(M, F_{MR}, F_K)
          }
        }
        \tjs{\}}
    \end{array}
  \]
  \caption{Proof outlines for \js{Membrane} and its nested \js{kill} function}
  \label{fig:proof:membrane}
\end{figure}

\begin{figure}
  \[
    \begin{array}{l}
      \tjs{var MembraneRef = function(ref) \{} \\
        \indblock{
          \logic{
            \exists R \st \args{
              alive(M, F_{MR}, F_K) \sep  \fullobj_{R}(\js{ref}: ref, @this: \_, @proto: \nil, \js{access}:
                \und) \sep {} \\
              ref \dotin \loc \sep
              MRset_M(R:Ls,T') \sep ref \bp S \cup T' \sep \ls \doteq R:M:Ls
            }
          } \\
          \tjs{var access = function(field) \{...\}} \\
          \logic{
            \exists R,F_A \st \args{
              alive(M, F_{MR}, F_K) \sep \fullobj_{MR}(\js{ref}: ref, @this: \_, @proto: \nil, \js{access}:
                F_A) \sep {} \\
              \newfun_{F_A}(R:M:Ls, \js{field}, \lambda_A, \_) \sep ref \dotin
              \loc \sep {} \\
              MRset_M(R:Ls,T') \sep ref \bp S \cup T' \sep \ls \doteq R:M:Ls
            }
          } \\
          \tjs{if(primitive(ref)) \{...\} else \{} \\
            \indblock{
              \tjs{return access;} \\
            }
          \tjs{\}} \\
          \logic{
            \exists R,F_A \st \args{
              alive(M, F_{MR}, F_K) \sep \fullobj_{R}(\js{ref}: ref, @this: \_, @proto: \nil, \js{access}:
                F_A) \sep {} \\
              \newfun_{F_A}(R:M:Ls, \js{field}, \lambda_A, \_) \sep ref \dotin
              \loc \sep {} \\
              MRset_M(R:Ls,T') \sep ref \bp S \cup T' \sep \ls \doteq R:M:Ls
                \sep {} \\
              \rv \doteq F_A
            }
          } \\
          \step{subst} \\
          \logic{
            \exists R,F_A \st \args{
              alive(M, F_{MR}, F_K) \sep MRset_M(R:Ls,T') \sep MRef_{R}(M,ref,F_A,S,T') \sep
                {} \\
              ref \dotin \loc \sep rv \doteq F_A \sep \ls \doteq R:M:Ls
            }
          }
        }
      \tjs{\}} \\
      \\
      \tjs{var access = function(field) \{} \\
        \indblock{
          \logic{
            \exists A \st \ls \doteq A : R_1 : M : \_ \sep
            \fullobj_A(\js{field}: field, @proto:\nil, @this:\_) \sep {} \\
            alive(M, F_{MR}, F_K) \sep
            MRef_{R_1}(M,ref,F_{A_1},S_1,T_1) \sep
            MRset(Ls,T_1) \sep {} \\

            field \dotin \uvars \sep
            (ref, field) \pointsto V \sep
            V \dotin \loc \sep
            V \bp S_2
          } \\
          \tjs{if(!killed) \{} \\
            \indblock{
              \logic{
                \exists A \st \ls \doteq A : R_1 : M : \_ \sep {} \\
                alive(M, F_{MR}, F_K) \sep
                MRef_{R_1}(M,ref,F_{A_1},S_1,T_1) \sep
                MRset_M(Ls,T_1) \sep {} \\

                (A, \js{field}) \pointsto field \sep
                field \dotin \uvars \sep {} \\

                (ref, field) \pointsto V \sep
                V \dotin \loc \sep
                V \bp S_2 \cup T_1
              } \\
              \tjs{return MembraneRef(ref[field]);} \\
              \logic{
                \exists A,R_2,F_{A_2} \st \args {
                  \ls \doteq A : R_1 : M : \_ \sep
                  alive(M, F_{MR}, F_K) \sep {} \\

                  MRef_{R_1}(M,ref,F_{A_1},S_1,T_1) \sep
                  MRef_{R_2}(M,V,F_{A_2},S_2,T_2) \sep {} \\

                  MRset_M(R_2:Ls,T_2) \sep
                  \rv \doteq F_{A_2} \sep {} \\

                  (A, \js{field}) \pointsto field \sep
                  field \dotin \uvars \sep
                  (ref, field) \pointsto V \sep
                  V \dotin \loc
                }
              } \\
            }
          \tjs{\}} \\
            \logic{
              \exists A,R_2,F_{A_2} \st \args{
                \ls \doteq A : R_1 : M : \_ \sep
                \fullobj_A(\js{field}: field, @proto:\nil, @this:\_) \sep {} \\

                alive(M, F_{MR}, F_K) \sep
                MRef_{R_1}(M,ref,F_{A_1},S_1,T_2) \sep {} \\

                MRset(R_2:Ls,T_2) \sep
                MRef_{R_2}(M,V,F_{A_2},S_2,T_2) \sep
                \rv \doteq F_{A_2} \sep {} \\

                field \dotin \uvars \sep
                (ref, field) \pointsto V \sep
                V \dotin \loc
              }
            }
        }
      \tjs{\}}
    \end{array}
  \]
  \caption{Proof outlines for the nested \js{MembraneRef} and \js{access} functions}
  \label{fig:proof:membraneref}
\end{figure}

\section{Example use of Caretaker and Membrane}
Assume we want to allow an untrusted program read access to \js{cookies} until
potentially sensitive data is stored in them by \js{updateCookies}.

\begin{verbatim}
cookies.password = '';
var cookieMembrane = Membrane(cookies);
var updateCookies = function(name, value) {
  cookieMembrane.kill();
  cookies[name] = value;
}

reval(maliciousCode, {cookies: cookieMembrane.access});
if (userHasEnteredPassword) {
  updateCookies('password', userPassword);
}
\end{verbatim}

The malicious code is able to access the cookies through the
\js{cookieMembrane.access} function%
\footnote{
  Note that one disadvantage with the Membrane is that the field accessor now has
  function-call syntax. The upcoming ES6 standard has a \emph{Proxies} feature
  that allows for intercepting standard syntax on an object and redirecting the
  input to a custom event handler. This will help make Membrane access functions
  \emph{transparent} to the user
}
passed into it as the \js{cookies} parameter of the \emph{imports} object.
Note that \js{maliciousCode} is unable to access any of the global variables
such as \js{userPassword}.

This specification for \js{maliciousCode} is \emph{maximally malicious}. The
$chaos$ predicate traverses every path that can be followed from $l$, and does
not specify what changes have been made.
\[
  \begin{array}{l}
    \logic{\ls \doteq [l] \sep traverse(l)} \\
    \js{maliciousCode();} \\
    \logic{\ls \doteq [l] \sep chaos(l)}
  \end{array}
\]
Despite this malicious specification of \js{maliciousCode}, we can prove that the
global password field remains untouched.

To do this, we specify a general verification condition that privileged
objects are not passed to the untrusted code, by ensuring that the imports
object is not reachable by traversing $\bp$ assertions from the privileged
objects.

In cases where untrusted code is allowed restricted access to privileged
objects, there will be an object which is only pointed to by @scope fields
of a (some) function(s). In this case, the privileged object is protected
according to the specifications of those function(s).

For example, a backpointer path for the \js{cookies} object is:
\begin{align*}
  cookies \bp & \{(l_g, \js{cookies}), (membraneref, \js{ref})\} \sep {} \\
  membrane \bp & \{(access, @scope)\} \sep {} \\
  access \bp & \{(imports, \js{cookies})\}
\end{align*}
From the middle line we can see that the cookies are protected by the membrane's
access function.


\chapter{Evaluation}
The nature of this project means that it is somewhat hard to evaluate the
efficacy of the developed logic. Nonetheless, we can judge aspects based on:
direct comparison with other published works; discussion of the applicability of
the extension to separation logic to other languages; 

\section{Operational Semantics}
The SES Operational Semantics were developed from a mixture of sources of
specification material, from the original JavaScript Operational Semantics, from
a \emph{very} loosely described informational document about SES, through to
examination of the source code of the JavaScript implementation of SES.
Where these sources differed, we opted to match the behaviour of the JavaScript
implementation.

Since completion of the semantics, a technical report
accompanying~\cite{ses-semantics} was discovered to contain a well-defined
version of the SES Operational Semantics (hereon Taly's semantics).

Direct comparison of our Operational Semantics and Taly's
showed general equivalence between the two languages albeit Taly's made fewer
generalisations and axiomatized more of the built-in commands.

Notably, the command whose precise semantics we were least sure of, the
restricted evaluation command, was found to be equivalent to those published by
Taly.

\section{Program Logic}
As discussed in section~\ref{sec:backpointers}, the $\bp$ operator cannot be a
pure operator as a result of the unintentional intersection of backpointer sets
that would occur as demonstrated in figure~\ref{fig:proof-bp-broken}. This issue
was remedied by giving the $\bp$ a footprint in the heap.

Depending on viewpoint, this solution may be considered a nice way around an
otherwise serious issue with the logic, or a horrendous \emph{hack} that depends
upon an internal variable defined upon the abstract heap just for the logic.

Ideally, more effort should have been spent to find a more general way to
introduce a footprint to this operator.

\section{Program Proofs}
The two program proofs presented in chapter~\ref{chap:proofs} demonstrate that
depending upon the purpose and generation method, proof derivations can be
either hugely verbose and hard to follow if produced mechanically with little
abstraction, or if some simple abstractions are applied, can be followed
relatively easily with a casual reading.

Dr. Mark Miller from Google, the designer of the SES language, visited Imperial
to give a seminar on the future of SES on the web in March 2013. We met to
discuss verification of SES programs using this logic. We walked through the
proof of the Caretaker pattern to ensure that the model of the language and the
verification logic were consistent with his own understanding of the language.

The meeting was beneficial as it confirmed that the model of the language and
the logic were indeed in line with his understanding of JavaScript and SES. The
meeting was also beneficial for revealing that the definition of an early
version of $\scopeBps$ was highly unintuitive, although the aim was correct.

\section{`Groundwork'}
It may have been noted by the reader that although we have defined
\js{freeze}, \js{def}, \js{reval}, and $\boxwand$, they are not actually
extensively used in this project.

\js{freeze} and \js{def} were defined for use for passing immutable interfaces
to untrusted code. Their definition and axiomatization was driven by an aim
to prove the makeContractHost algorithm\cite{contract-host-algo}. Unfortunately,
this aim was dropped when it transpired that earlier, \emph{easier} program
proofs would take far longer than expected to produce. It is hoped that a proof
could be developed before the presentation of this project.

\js{reval} is mentioned in passing during the discussion of the Caretaker and
Membrane proofs, using a combination of them and \js{reval} guarantees the
isolation properties proven on these design patterns. Untrusted code that is
executed outside of a \js{reval} command can trivially reach the protected
object by traversing the heap structure from the global object.

The $\boxwand$ logical operator was produced with the knowledge that the
conventional $\wand$ operator is used to generate a weakest precondition in the
proof of soundness of the Assignment inference rule.

\chapter{Conclusions \& Future Work}
\section{Conclusions}
\begin{itemize}
  \item We have specified the semantics of a model of the SES language.
  \item We have produced the first program logic for SES
  \item We have produced the first separation logic-based program logic that can
    reason about object capability security through backpointers
  \item We have produced the first proofs for some object capability programs
  \item We have shown that proofs in the program logic match closely match the
    intuitions of a programmer who has extensively studied the language and
    program
  \item And, along the way, we have discovered the duality of uses for the wand
    operator
\end{itemize}

\section{Future Work}
There are many directions in which work on the SES language and verification
logic can be pursued to follow the developments of the JavaScript language, the
distributed variant of SES, or to improve and understand the logic further.

\subsection{Extend the SES Language Model}
This model of the SES language was based upon old and generalised ES3 semantics,
in order to demonstrate the extensions to the logic in a simple way.
However, a project is underway within the Reliable Web research group to
precisely specify the ES5.1 specification.
Properly formalising this model of SES on top of these new semantics using the
same theorem proving tools would give a boost to the accuracy of the logic and
provide benefits from the tooling that is being built around these semantics and
logic.

The SES language is only one portion of a bigger plan for web programming, with
the emergence of JavaScript on the server, there is a demand for communications
primitives that are easier to use. One extreme of this is the extension of the
object model of the language across the network to other machines. This is
possible through a syntax-transparent communication library. In this case there
is a definite motivation for the SES component of DrSES.

This new, distributed language poses new issues in program verification since
communication also has to be considered. A potential method for reasoning about
programs written in this style may involve a fusion of the SES logic with a
system suitable for reasoning about communications such as Session Types.

And finally, there's always a large number of other interesting algorithms to
verify, Sealable References are another common sample program often used in
research papers into object capability-style languages. As a challenge during
his visit, he posed us to prove all the algorithms he uses in his presentations
-- several of these would prove quite a challenge as they rely on complex
cascades of callback functions.

\subsection{Logic}
After the project had begun, we noted that there was newly published
material~\cite{Ley-Wild2013SAS} that also introduces non-local heap state to the
logical assertion satisfaction relations. However, the paper does not warrant
the use of $\wand$, so they left it undefined in their logic.

We believe there is further research to be made into the definitions of $\wand$
and $\boxwand$ in the presence of non-local heap state. We consider that it may
be possible to define a more general version of the wand to support all of its
common use cases as well as maintaining the right-adjoint property.

We have not considered how the new operators act in the presence of negation, it
is already considered that negation acts in unintuitive ways in separation
logic. It may be interesting to further see how the backpointer reacts under
negation.

If we can extend the logic to support some aspects of non-locality, can we
add another tier to the heap states to reduce the coarseness of the local vs.
global reasoning. For example, is it possible to split the heap into three state
environments of local, trusted, untrusted? What further operators would we
require to express useful assertions in this logic?

\subsection{Verification of the JavaScript-based SES implementation}
The implementation of SES as a layer on top of JavaScript also poses its own
interesting problems. SES inherently depends upon the JavaScript specification
as its basis, but the JavaScript specification process is also directed to
maintaining the properties required to enable SES to be implemented securely.
These properties are usually very subtle, such as ensuring that the
language-defined data structures and primitives do not inadvertently allow for
pointer leakage or covert channels of communication within the language.

As an indirect result of this project a potential bug~\cite{es6-bug} has been discovered in the
upcoming ECMAScript specification by the ECMAScript research group at Imperial.
This demonstrates that although the
specification committee should now have a good insight into the security
implications of the language's design decisions, errors are still being made as
a result of a poor, informal description of the security properties that ought
to hold for the language. With a formal specification of these properties, we
could prove that the formally specified language definitely does hold these
properties.

A proof that the SES implementation is correct is currently well beyond the
capabilities of the JavaScript program logic, as the SES implementation reasons
about the particular implementation-specific semantics of the JavaScript engine
on which it is run, and attempts to fix any semantics that it deems unsafe.

Program logics to date assume that there is one correct semantics for a language
under which a program runs, JavaScript poses the unique possibility where a
program may run under similar, but subtly different semantics. Research into
extension of the existing JavaScript program logic to support variable semantics
would be a novel research project.

\section{Automated Program Verification}
In general, the problem of finding specifications for programs is undecidable, however because 
separation logic is structural, this provides additional power for automated
reasoning programs to leverage. Theories behind Symbolic
Execution~\cite{symb-exec} and
Biabduction~\cite{1480917} for separation logic were developed by O'Hearn et al.
as a means of analysing the shapes of footprints of assertions.

Applications of these methods have shown them to be suitable
for use in practical and scalable automated reasoning tools for many languages
including C and Java~\cite{smallf,slayerp,spacep,1449782}.

It is anticipated that such theory can be applied to the JavaScript separation
logic to provide realistic tools for automated JavaScript verification.

\subsection{Comparisons with Other Techniques}
Considerable amounts of research has been performed into dynamic information
flow analysis for JavaScript programs through instrumentation of the
interpreter~\cite{js-ifa}. Abstract interpretation for information flow analysis
on JavaScript has also been attempted~\cite{js-ifa2} successfully.
It was noted
during the construction of figure~\ref{rr-main} that the building of the
backpointer sets `felt' very much like the execution of abstract interpretation
algorithms as discussed (and practiced) in the 4th Year
Course~\cite{prog-anal,popa}. Drawing a more formal comparison between the two
techniques may assist in the development or adoption of already-existing
techniques from the Abstract Interpretation field into this.

It has been suggested~\cite{pubsdoc:rolesForOwners} that Ownership Types would
be a suitable means of verifying the security properties of SES. It would be
interesting to see if Ownership Types for this purpose can be expressed using
the logic developed in the course of this project.

\bibliography{bibliography}
\bibliographystyle{plainnat}

\appendix
\chapter{Notation}
\begin{display}{Notation: Sorts and Constants.}
\entry{H {\;\in\;} \loc\times\vars\rightharpoonup\vals}{Heaps.}\\[\gap]
\entry{l {\;\in\;} \locb\triangleq \loc\cup\{\nil\}}{Locations.}\\[\gap]
\entry{L {\;\in\;} \scopechains\triangleq \loc^n}{Scope chains.}\\[\gap]
\entry{\lgo}{Global object.}\\[\gap]
\entry{\lop}{\js{Object.prototype}.}\\[\gap]
\entry{\left\{\begin{array}{l}\protop,\thisp,\\
             \fscopep,\bodyp,@bp\end{array}\right\}\in\ivars}{Internal variables.}\\[\gap]
\entry{x\in\vars\triangleq \ivars\disju\uvars}{Variables.}\\[\gap]
\entry{\js x\in\uvars\subseteq \Strings}{User variables.}\\[\gap]
\entry{\njs{r} {\;\in\;} \valsr \triangleq \vals\cup \refs}{Return values.}\\[\gap]
\entry{\njs{v} {\;\in\;} \vals \triangleq \uvals\cup\locb\cup \scopechains \cup \sortfun}{Semantic values.}\\[\gap]
\entry{\js{v} {\;\in\;} \uvals\triangleq \sortnum\cup \Strings \cup \sortundef \cup \{\nil\}}{User values.}\\[\gap]
\entry{\js{n} {\;\in\;} \sortnum}{Numbers.}\\[\gap]
\entry{\js{m} {\;\in\;} \Strings}{Strings.}\\[\gap]
\entry{\und {\;\in\;} \sortundef}{Undefined.}\\[\gap]
\entry{l\sv\js{x} {\;\in\;} \refs}{References.}\\[\gap]
\entry{\lambda\js{x.e} {\;\in\;} \sortfun}{Function code.}\\[\gap]
\entry{\js{e} {\;\in\;} \sortexp}{Expressions.}
\end{display}

\chapter{Operational Semantics}
\label{app:opsems}
\newcommand{\auxf}[2]{#1 & \triangleq & #2}
\begin{display}{Auxiliary Functions}
  $\begin{array}{lcl}
  \auxf{\obj(l,l')}{l \pointsto \{@proto: l', @frozen: \false\}}\\
  \auxf{\istrue(v)}{v\not\in\{0,\emptystr,\nil,\und,\false\}}\\
  \auxf{\isfalse(v)}{v\in   \{0,\emptystr,\nil,\und,\false\}}\\
  \auxf{\fun(l',L,\js x,\js e,l)}{l'\mapsto\{@proto: \lfp, \js{prototype}:
        l, @scope: L, @body: \lambda \js{x.e}, @frozen: \false\}}\\
  \auxf{\pickThis(H, \fld{l}{x})}{l \phantom{\und} \quad
        \text{if } (l,@this) \in \dom(H)}\\
  \auxf{\pickThis(H, r)}{\und \phantom{l} \quad \text{otherwise}}\\
  \auxf{\objOrGlob(l)}{l\phantom{_{op}} \quad l\in \loc}\\
  \auxf{\objOrGlob(v)}{\lop \quad v\not\in\loc}\\
  \auxf{\getBase(l,l')}{l' \qquad l \in \loc}\\
  \auxf{\getBase(l,v)}{v \qquad v\not\in\loc}\\

  \auxf{\act(l,\js{x},v,\js{e},l'')}{l\pointsto\{\js{x}:
    v,@this: l'',@proto: \nil\}\disju \defs(\js{x},l,\js{e})}\\

  \auxf{\auxDef(H,l,s)}{\begin{cases}
      \emp & l \in s \lor l \not\in \loc\\
      (l,@frozen) \mapsto \true \cup \bigcup_{(l,x_n) \in H, x_n \in \uvars}
      \auxDef(H,H(l,x_n),s \cup\{l\}) & \mbox{otherwise} \\
    \end{cases}}\\
  \auxf{\ReadWrite(H, l)}{H(l, @frozen) = \false} \\
  \end{array}$
\end{display}
  \begin{display}{Local Variable Declarations}
    \jaxiom{\defs(\js{x},l,\jsvar{y})}{(l,y) \pointsto \und \quad$ \= if
      $\js{x} \neq \js{y}} \\
    \jaxiom{\defs(\js{x},l,\js{e_1 = e_2})}{\defs(\js{x},l,\js{e_1})} \\
    \jaxiom{\defs(\js{x},l,\js{e_1;e_2})}{\defs(\js{x},l,\js{e_1}) \cup
      \defs(\js{x},l,\js{e_2})} \\
    \jaxiom{\defs(\js{x},l,\js{if(e_1)\{e_2\}\{e_3\}})}
      {\defs(\js{x},l,\js{e_2}) \cup \defs(\js{x},l,\js{e_3})} \\
    \jaxiom{\defs(\js{x},l,\js{while(e_1)\{e_2\}})}{\defs(\js{x},l,\js{e_2})} \\
    \jaxiom{\defs(\js{x},l,\js{e})}{\emp$ \> otherwise$}
  \end{display}

  \begin{display}{Heap Update $H[H']$}
    \jaxiom{H[\emp]}{H} \\
    \jaxiom{H[(l,x) \pointsto v]}{H \disju (l,x) \pointsto v \qquad \mbox{if }
      (l,x) \not\in \dom(H)} \\
    \jaxiom{H[(l,x) \pointsto v \disju H']}{H[(l,x) \pointsto v][H']}
  \end{display}

  \begin{display}{Scope resolution: $\scope(H,l,x)$.}
    \jaxiom{\scope(H,{\emptylist},\njs{x})}{\nil}{}\vg
    \jrule{\scope(H,l\cons L,\njs{x})}{l}{\proto(H,l,\njs{x}) \neq \nil}\rsep
    \jrule{\scope(H,l\cons L,\njs{x})}{\scope(H,L,\njs{x})}{\proto(H,l,\njs{x}) = \nil}
  \end{display}
  %
  \begin{display}{Prototype resolution: $\proto(H,l,x)$.}
    \jaxiom{\proto(H,\nil,\njs{x})}{\nil}{}\vg
    \jrule{\proto(H,l,\njs{x})}{l}{(l,\njs{x}) \in\dom(H)}\rsep
    \jrule{\proto(H,l,\njs{x})}{\proto(H,l',\njs{x})}{(l,\njs{x}) \not\in\dom(H) \qquad H(l,@proto) = l'}
  \end{display}
  %
  \begin{display}{Dereferencing values: $\getValue(H,r)$.}
    \jrule{\getValue(H,\njs{r})}{\njs{r}}{\njs{r}\neq \fld{l}{\js{x}}}~
    \jrule{\getValue(H,\fld{l}{\js{x}})}{\und}{\proto(H,l,\js{x}) = \nil\\ l\neq \nil}~
    \jrule{\getValue(H,\fld{l}{\js{x}})}{H(l',\js{x})}{\proto(H,l,\js{x}) = l'\\ l\neq \nil}
  \end{display}

\begin{display}{Operational semantics: $H,L,\js e \evalsto H',v$}
  Dereferencing notation: $H,L,\js{e} \gevalsto H',v \triangleq \exists r. (H,L,\js{e} \evalsto H',r \wedge \getValue(H',r) = v)$.
\vg

  \stateaxiom{(Value)}
  {H,L,\js{v} \evalsto H,v}
\vg

  \stateaxiom{(Variable declaration)}
  {H,L,\jsvar{x} \evalsto H,\und}
\vg

  \staterule{(Object creation)}
    {H_0 = H \disju \obj(l,\lop)\\
     \forall i\in 1..n \st \left(\begin{array}{l}
      H_{i-1},L,\js{e_i}\gevalsto H_i',v_i \\
      H_i = H_i' [ (l,\js{x_i}) \pointsto v_i]\end{array}\right)}
    {H,L,\js{\{x_1:e_1,\dots,x_n:e_n\}} \evalsto H_n,l}
\vg

  \staterule{(Sequence)}
    {H,L,\js{e_1} \evalsto H'',r' \\
     H'',L,\js{e_2} \evalsto H',r}
   {H,L,\js{e_1; e_2} \evalsto H',r}
\vg

  \staterule{(Binary operator)}
    {H,L,\js{e_1} \gevalsto H'',v_1 \\
     H'',L,\js{e_2} \gevalsto H',v_2\\
     v_1 \oplus v_2 = v}
   {H,L,\js{e_1 \oplus e_2} \evalsto H',v}
\vg

  \staterule{(Conditional true)}
  {H,L,\js{e_1} \gevalsto H'',v \quad \istrue(v) \\
   H'',L,\js{e_2} \evalsto H',r}
  {H,L,\js{if(e_1)\{e_2\} else \{e_3\}} \evalsto H',r}
\qquad

  \staterule{(Conditional false)}
  {H,L,\js{e_1} \gevalsto H'',v \quad \isfalse(v) \\
   H'',L,\js{e_3} \evalsto H',r}
  {H,L,\js{if(e_1)\{e_2\} else \{e_3\}} \evalsto H',r}
\vg

  \staterule{(While true)}
  {H,L,\js{e_1} \gevalsto H'', v \quad \istrue(v) \\
   H'',L,e_2\js{;while(e_1)\{e_2\}} \evalsto H', v''}
  {H,L,\js{while(e_1)\{e_2\}} \evalsto H',\und}
\qquad

  \staterule{(While false)}
  {H,L,\js{e_1} \gevalsto H', v \quad \isfalse(v)}
  {H,L,\js{while(e_1)\{e_2\}} \evalsto H',\und}
\vg

  % l' \neq \nil condition given in ES5-8.7.2.3.a Strict-mode null SetValue (and
  % all-mode GetValue)
  \staterule{(Variable)}
    {\scope(H,L,x) = l' \quad l' \neq \nil}
    {H,L,\js x \evalsto H, \fld{l'}{x}}
\vg

  \staterule{(Member access)}
  {H,L,\js e \gevalsto H',l' \\
   l' \neq \nil}
  {H,L,\js{e.x} \evalsto H', \fld{l'}{x}}
\qquad

  \staterule{(Computed member access)}
  {H,L,\js{e_1} \gevalsto H'',l' \\
   l' \neq \nil \\
   H'',L,\js{e_2} \evalsto H', x}
  {H,L,\js{e_1[e_2]} \evalsto H',\fld{l'}{x}}
\vg

  \staterule{(Assignment)}
  {H,L,\js{e_1} \evalsto H_1,\fld{l}{x} \qquad
   \ReadWrite(H_1, l) \\
   H_1,L,\js{e_2} \gevalsto H_2, v \\
   H' = H_2[(l,x) \pointsto v]}
  {H,L,\js{e_1 = e_2} \evalsto H', v}
\vg

  \staterule{(Function creation)}
  {H' = H \disju \obj(l,\lop) \disju \fun(l',L,\js{x},\js{e},l)}
  {H,L,\jsfun{x}{e} \evalsto H',l'}
\vg

  \staterule{(Named function creation)}
  {H' = H \disju \obj(l,\lop) \disju \fun(l',l_1 \cons L,\js{x},\js{e},l) \disju
    l_1 \pointsto \{@proto:\nil, y:l'\}}
  {H,L,\jsfun[y]{x}{e} \evalsto H',l'}
\vg

%\COMMENT{There's no need to $\proto$ in this now that we've disallowed prototype
%  chains from scope lists (we hit the base AR before we need to start
%  traversing) (Also assume that $\lgo$ has $@this$ set directly) We can also drop
%  $\getValue$ to just use $\scope$?}
  \staterule{(This)}
  {\scope(H,L,@this)=l \\
   (l,@this) \pointsto l'}
  {H,L,\js{this} \evalsto H,l'}
\vg

  \staterule{(Function call)}
  {H,L,\js{e_1} \evalsto H_1,r_1\qquad
   \pickThis(H_1,r_1)=l_2\qquad
   \getValue(H_1,r_1)=l_1\\
   H_1(l_1,@body)=\lambda \js{x.e_3}\qquad
   H_1(l_1,@scope)= L'\\
   H_1,L,\js{e_2} \gevalsto H_2,v\\
   H_3 = H_2\disju\act(l,\js x,v,\js{e_3},l_2) \\
   H_3,l \cons L',\js{e_3} \gevalsto H',v'}
  {H,L,\js{e_1(e_2)} \evalsto H',v'}
\vg

  \staterule{(Object construction)}
  {H,L,\js{e_1} \gevalsto H_1,l_1 \qquad
   l_1\neq \nil\qquad
   H_1(l_1,@body)=\lambda \js{x.e_3}\\
   H_1(l_1,@scope)= L'\qquad
   H_1(l_1,\js{prototype})= v\\
   H_1,L,\js{e_2} \gevalsto H_2,v_1  \qquad
   l_2 = \objOrGlob(v) \\
   H_3 = H_2\disju \obj(l_3,l_2) \disju\act(l,\js x,v_1,\js{e_3},l_3)\\
   H_3,l \cons L',\js{e_3} \gevalsto H',v_2\qquad
   \getBase(l_3,v_2) = l'}
  {H,L,\jsnew{e_1}{e_2} \evalsto H',l'}
\vg

  \staterule{(Restricted evaluation)}
  {H,L,\js{e_1} \gevalsto H_1,v \qquad
   \js{e_3} = \parse(v) \\
   H_1,L,\js{e_2} \gevalsto H_2, l \\
   H_3 = H_2 \disju l' \pointsto \{@this:l, @proto:\nil\} \disju
     \defs(\_, l', \js{e_3}) \\
   H_3, l' \cons [l], \js{e_3} \gevalsto H',v }
  {H,L,\js{reval(e_1, e_2)} \evalsto H',v}
\vg

  \staterule{(Freeze)}
  {H,L,\js{e} \gevalsto H'', l\\
   H' = H''[(l, @frozen) \pointsto \true]}
  {H,L,\js{freeze(e)} \evalsto H', l}
\vg

  \staterule{(Recursive freeze)}
  {H,L,\js{e} \gevalsto H'', l \\
   H' = H''[\auxDef(H'', l, \{\})]}
  {H,L,\js{def(e)} \evalsto H', l}
\end{display}

\chapter{Program Logic}
\newcommand{\defline}[2]{#1 & \qquad & \text{#2}\\}
\begin{display}{Notation: Sorts and Constants}
%
\entry{\env\in\lvars \rightharpoonup \lvals}{Logical environment.}\\[\gap]
\entry{\V{v}\in\lvals}{Logical values.}\\[\gap]
\entry{\V{X}\in\lvars}{Logical variables.}\\[\gap]
\entry{\E{E}}{Logical expressions.}\\[\gap]
\entry{\eval{\Expr}^L_{\env}}{Logical evaluation.}\\[\gap]
\entry{av\in\avals \triangleq \vals \cup \{\none\}
}{Abstract values.}\\[\gap]
\entry{h \in (\loc\times\vars) \rightharpoonup\avals}{Abstract heap.}\\[\gap]
\entry{\heval{h}(l,x)}{Heap evaluation.}\\[\gap]
\entry{P}{Logical assertion.}\\[\gap]
\entry{\T{set}}{Generic set.}\\[\gap]
\entry{h,h_g,L,{\env} \satisfies P}{Satisfaction relation.}\\[\gap]
\entry{\tr P{\js e}Q}{Hoare triple.}\\[\gap]
%
\end{display}
\begin{display}{Abstract Values and Abstract Heap.}
$av\in\avals \defeq v \pipe \none$\qquad\qquad\qquad\qquad\qquad\qquad\quad
$h : (\loc\times\vars) \rightharpoonup\avals$\\[\gap]
$\heval{h}(l,x) \triangleq h(l,x) \text{ iff } (l,x) \in \domain(h) \land h(l,x) \not= \none$
\end{display}
%
\begin{display}{Logical Expressions and Evaluation: $\eval{\Expr}^L_{\env}$.}
~~~$ \V{v}\in\lvals \defeq e \pipe l\sv x \pipe av \pipe L $\qquad\qquad\qquad\qquad\qquad\quad
$ \env:\lvars \rightharpoonup \lvals$ \\[\gap]
$\begin{array}{rll}
  \Expr \defeq & \V X& \text{Logical variables} \\
  & \pipe \ls & \text{Scope list} \\
  & \pipe \V{v} & \text{Logical values} \\
  &\pipe \Expr \oplus \Expr%\pipe \Expr - \Expr\pipe \Expr * \Expr\pipe \Expr / \Expr & \text{Arithmetic}\\
  %&\pipe \Expr . \Expr 
        & \text{Binary Operators} \\  
  &\pipe \Expr \cons \Expr &\text{List cons} \\
%  &\pipe \Expr \in \T{set} & \text{Expression Type Checking}\\
  & \pipe \Expr \sv \Expr & \text{Reference construction}\\
  & \pipe \lambda\Expr.\Expr & \text{Lambda values}
\end{array}$\\[\gap]
%
~~~\jaxiom{\eval{\V{v}}^L_{\env}}{\V{v}}
\rsep
%
\jaxiom
{\eval{\ls}^L_{\env}}{ L }
\rsep
\jaxiom{\eval{\V{X}}^L_{\env}  }{{\env}(\V{X})}
\\[\gap]


\jrule
{\eval{{\Expr_1}\cons{\Expr_2}}^L_{\env} }{ \eval{{\Expr_1}}^L_{\env} \cons \List}
{\eval{{\Expr_2}}^L_{\env} = \List }
\rsep
%
\jrule
{\eval{{\Expr_1}\sv{\Expr_2}}^L_{\env}  }{ \eval{{\Expr_1}}^L_{\env} \sv \eval{{\Expr_2}}^L_{\env}}
{\eval{{\Expr_1}}^L_{\env} = l' \land \eval{{\Expr_2}}^L_{\env}=\js{x}}
\\[\gap]


\jrule
{\eval{{\Expr_1}\oplus{\Expr_2}}^L_{\env}  }{ v\primop v'}
{\eval{\Expr_1}^L_{\env} = v \land \eval{\Expr_2}^L_{\env} = v'}
\rsep
%
\jrule
{  \eval{\lambda{\Expr_1}.{\Expr_2}}^L_{\env}  }{  \lambda\eval{{\Expr_1}}^L_{\env}.\eval{{\Expr_2}}^L_{\env}}
{ \eval{{\Expr_1}}^L_{\env}=\js{x} }
%
\end{display}

\newcommand{\asrtline}[3][\pipe]{#1 & #2 & \text{#3}\\}
\begin{display}{Assertions}
  $\begin{array}{rll}
    \asrtline[P ::=]{P \land P \pipe P \lor P \pipe \lnot P \pipe \ltrue \pipe
    \lfalse}{Boolean formulas}
    \asrtline{P \sep P \pipe P \wand P \pipe P \sepish P}{Structural formulas}
    \asrtline{\bigsep_{x \in \T{set}} P(x)}{Iterative $\sep$}
    \asrtline{P \boxwand Q}{Global `hypothesis' formula}
    \asrtline{(E,E) \pointsto E \pipe \lemp}{Heap formulas}
    \asrtline{E \bp E}{Backpointer formula}
    \asrtline{E = E}{Expression equality}
    \asrtline{\forall X \st P \pipe \exists X \st P}{First-order formulas}
  \end{array}$
\end{display}

\newcommand{\assaxiom}[1]{\>$h,h_g,L,\env \satisfies #1$}
\newcommand{\asssat}[3]{\assaxiom{#1}\>$\iff$\>$#2$
  \ifblank{#3}{}{\\\>\>\>\quad$#3$}}
\begin{display}{Assertion satisfaction relation}
\hspace{1.5em} \= $h,h_g,L,\env \satisfies (E_1,E_2) \pointsto E_3$ \= $\iff$ \= \kill

\assaxiom{\ltrue} \\
%    h,h_g,L,\env \not\satisfies \false \\
\asssat{P \land Q}{(h,h_g,L,\env \satisfies P) \land (h,h_g,L,\env \satisfies
Q)}{} \\
\asssat{P \lor Q}{(h,h_g,L,\env \satisfies P) \lor (h,h_g,L,\env \satisfies
Q)}{} \\
\asssat{\lnot P}{\lnot(h,h_g,L,\env \satisfies P)}{} \\
\asssat{P \sep Q}{\exists h_1,h_2 \st h \equiv h_1 \disju h_2 \land{}}
  {(h_1,h_g,L,\env \satisfies P) \land (h_2,h_g,L,\env \satisfies Q)} \\
\asssat{P \sepish Q}{\exists h_1,h_2,h_3 \st
    h \equiv h_1 \disju h_2 \disju h_3 \land{}}
  {(h_1 \disju h_3, h_g, L, \env \satisfies P) \land
    (h_2 \disju h_3, h_g, L, \env \satisfies Q)} \\
\asssat{P \wand Q}{\forall h' \st (h',h_g,L,\env \satisfies P) \land h \disj h'
    \land h' \subseteq h_g}
  {\impl (h \disju h', h_g, L, \env \satisfies Q)} \\
%h,h_g,L,\env \satisfies \upd P}{h, h_g[h], L, \env \satisfies P }\\
%h,h_g,L,\env \satisfies P?Q}{h, h_g[h], L, \env \satisfies P }\\
%P \boxwand Q & \triangleq & P \wand P ? Q }\\
\asssat{P \boxwand Q}{\forall h' \st (h',h_g[h'],L,\env \satisfies P)
    \land h \disj h'}
  {\impl (h \disju h', h_g[h'], L, \env \satisfies Q)} \\
\asssat{\bigsep_{x\in\{\}}P(x)}{h,h_g,L,{\env}\satisfies\lemp}{} \\
\asssat{\bigsep_{x\in\T{set}}P(x)}{y\in\T{set} \land h,h_g,L,{\env} \satisfies
  P(y)\sep(\bigsep_{x\in(\T{set}\setminus y)}P(x))}{} \\
\asssat{\lemp}{h = \emp}{} \\
\asssat{(E_1,E_2) \pointsto E_3}{h \equiv (\evalle{E_1}, \evalle{E_2}) \pointsto
  \evalle{E_3}}{} \\
\asssat{E_1 \bp E_2}{h \equiv (\evalle{E_1}, @bp) \pointsto v \land
  \forall (l,x) \in \dom(h_g) \st h_g(l,x) = \evalle{E_1}}
  {\impl (l,x) \in \evalle{E_2}} \\
\asssat{E_1 = E_2}{\evalle{E_1} = \evalle{E_2}}{} \\
\asssat{\exists\V X \st P}{\exists \V{v} \st h,h_g,L,[{\env} | \V X\takes \V{v}]
  \satisfies P}{} \\
\asssat{\forall\V X \st P}{\forall \V{v} \st h,h_g,L,[{\env} | \V X\takes \V{v}]
  \satisfies P}{}
\end{display}
\paragraph{Definition of Hoare triple and soundness properties}
\[ \tr{P}{\js{e}}{Q} \]
Soundness:
\[ (h, h \disju h_f, L, (\env\setminus\rv) \satisfies P) \land h, L, \js{e} \leadsto h', v
  \impl (h', h' \disju h_f, L, [\env|\rv\takes v] \satisfies Q) \]
Fault avoidance:
\[ h, h_g, L, (\env\setminus\rv) \satisfies P \land h\subseteq h_g \impl h, L, \js{e} \not\leadsto \fault \]
Safety monotonicity:
\[ (h, h_g, L, \env \satisfies P) \land h \disj h' \land h, L, \js{e} \not\leadsto
  \fault \land h\subseteq h_g \impl h \disju h', L, \js{e} \not\leadsto \fault \]
Frame property:
\[ (h, h_g, L, \env \satisfies P) \land h, L, \js{e} \not\leadsto \fault \land
  h \disju h', L, \js{e} \leadsto h_2' \land h\subseteq h_g\impl h, L, \js{e} \leadsto h_2 \land
  h_2' = h_2 \disju h' \]

\label{app:logicauxpreds}
\begin{display}{Auxiliary Predicates}
  \jaxiom{\obj_E(E_1:E'_1,...,E_n:E'_n)}{(E,E_1) \pointsto E'_1 \sep ... \sep
    (E,E_n) \pointsto E'_n} \\
  \jaxiom{\newobj_L(V_1,...,V_n)}{\bigsep_{V \in \vars \setminus
    \{V_1,...,V_n\}} (L,V) \pointsto \none} \\
  \jaxiom{\fullobj_E(E_1:E_1',\dots,E_n:E_n')}
    {\obj_{E}(E_1:E_1',\dots,E_n:E_n') \sep \newobj_{E}(E_1,\dots,E_n)} \\
  \jaxiom{\fun_F(E_1,E_2,E_3,E_4)}{(F, @scope) \pointsto E_1 \sep
    (F, @body) \pointsto \lambda E_2.E_3 \sep (F, \js{prototype}) \pointsto E_4
    \sep (F, @proto) \pointsto \lfp} \\
  \jaxiom{\newfun_{E}(E_1,E_2,E_3,E_4)}{\fun_{E}(E_1,E_2,E_3,E_4) \sep
    \newobj_{E}(@proto, \js{prototype}, @scope, @body)} \\
  \jaxiom{\scopeBpsUpd(Ls, s, s', n)}{\bigsep_{0\leq i<\length(Ls)}(
    \lstitem(i,Ls) \bp \lstitem(i,s') \sep \lstitem(i,s') \doteq \lstitem(i,s)
    \cup n)} \\
  \jaxiom{\scopeBps(Ls, s)}{\scopeBpsUpd(Ls, s, s, \{\})} \\
  \jaxiom{\bpGen(V,\_,\_,\_)}{V \notdotin \loc} \\
  \jaxiom{\bpGen(V,L,x,s)}{V \dotin \loc \sep V \bp \{(L,x)\} \cup s} \\
  \jaxiom{\auxDefGet(V,s)}{V \notdotin \loc \lor V \dotin s} \\
  \jaxiom{\auxDefGet(V,s)}{V \notdotin s \land (V, @frozen) \pointsto \_ \sep (\bigsepish_{x_n \in
    \uvars} (V,x_n) \pointsto V' \sepish \auxDefGet(V', s\cup\{V\}))} \\
  \jaxiom{\auxDefSet(V,s,b)}{V \notdotin \loc \lor V \dotin
    s} \\
  \jaxiom{\auxDefSet(V,s,b)}{V \notdotin s \land (V, @frozen) \pointsto b \sep (\bigsepish_{x_n \in
    \uvars} (V,x_n) \pointsto V' \sepish \auxDefSet(V', s\cup\{V\}, b))} \\
  \jaxiom{\ReadWrite(L)}{(L, @frozen) \pointsto \false} \\
  \jaxiom{\decls(X,L,\js{e})}{\js{x_1},...,\js{x_n} \quad \text{where }
    (L,\js{x_i}) \in \dom(\defs(X,L,\js{e}))} \\
  \jaxiom{\pickThis(\fld{L}{\_},L)}{(L,@this) \pointsto \none} \\
  \jaxiom{\pickThis(\fld{L}{\_},\und)}{\exists V \st (L,@this) \pointsto V \sep V
    \not\doteq \none}
\end{display}

\begin{display}{Local Variable Definitions}
  \jaxiom{\defs(X,L,\jsvar{y})}{(L,\js{y}) \pointsto \und \sep X \not\doteq \js{y}} \\
  \jaxiom{\defs(X,L,\js{e_1 = e_2})}{\defs(X,L,\js{e_1})} \\
  \jaxiom{\defs(X,L,\js{e_1;e_2})}{\defs(X,L,\js{e_1}) \sepish
    \defs(X,L,\js{e_2})} \\
  \jaxiom{\defs(X,L,\js{if(e_1)\{e_2\}\{e_3\}})}
    {\defs(X,L,\js{e_2}) \sepish \defs(X,L,\js{e_3})} \\
  \jaxiom{\defs(X,L,\js{while(e_1)\{e_2\}})}{\defs(X,L,\js{e_2})} \\
  \jaxiom{\defs(X,L,\js{e})}{\lemp \quad \text{otherwise}}
\end{display}

\begin{display}{Inference Rules}
  \stateaxiom{(Declaration)}
    {\tr {\lemp} {\jsvar{x}} {\rv \doteq \und}}
  \vg

  \stateaxiom{(Value)}
    {\tr {\lemp} {\js{v}} {\rv \doteq \js{v}}}
  \vg

  \staterule{(Variable)}
    {P = \scope(Ls_1, \ls, \js{x}, L) \sepish
      \getValue(Ls_2, \fld{L}{\js{x}}, V) \sep L \not\doteq \nil}
    {\tr P {\js{x}} {P \sep \rv \doteq \fld{L}{\js{x}}}}
  \vg

  \staterule{(Member Access)}
    {\tr P {\js{e}} {Q \sep \rv \doteq V} \quad Q = R \sep \getValue(Ls, V, L)
      \sep L \not\doteq \nil \sep L\dotin\loc}
    {\tr P {\js{e.x}} {Q \sep \rv \doteq \fld{L}{\js{x}}}}
  \vg

  \staterule{(Computed Access)}
    {
      \tr {P} {\js{e_1}} {R \sep \rv \doteq V_1} \quad R = S_1 \sep
      \getValue(Ls_1, V_1, L) \sep L \not\doteq \nil \sep L\dotin\loc\\
      \tr {R} {\js{e_2}} {Q \sep X \dotin \uvars \sep \rv \doteq V_2}
      \quad Q = S_2 \sep \getValue(Ls_2, V_2, X)
    }
    {\tr {P} {\js{e_1[e_2]}} {Q \sep \rv \doteq \fld{L}{X}}}
  \vg

  \staterule{(Bin Op)}
    {
      \tr P {\js{e_1}} {R \sep \rv \doteq V_1} \quad R = S_1 \sep \getValue(Ls_1,
      V_1, V_3) \\
      \tr R {\js{e_2}} {Q \sep \rv \doteq V_2} \quad Q = S_2 \sep \getValue(Ls_2,
      V_2, V_4) \\
      V = V_3 \mathbin{\bar\oplus} V_4
    }
    {\tr P {\js{e_1} \oplus \js{e_2}} {Q \sep \rv \doteq V}}
  \vg

  \staterule{(Assign)}
    {
      \tr P {\js{e_1}} {R \sep \rv \doteq \fld{L}{X}} \\
      \tr R {\js{e_2}} {Q \sep (L,X) \pointsto V_3 \sep \rv \doteq V_1} \\
      Q = S \sep \getValue(Ls, V_1, V_2) \sep \ReadWrite(L) \sep
      \bpGen(V_2, L, X, s)
    }
    {\tr P {\js{e_1 = e_2}} {Q \sep (L,X) \pointsto V_2 \sep \rv \doteq V_2}}
  \vg

  \staterule{(This)}
  { P = \scope(Ls_1, \ls, @this, L_1) \sepish \proto(Ls_2, L_1, @this, L_2)
    \sepish (L_2,@this) \pointsto V }
  {\tr P {\js{this}} {P \sep \rv \doteq V}}
  \vg

  \staterule{(Function)}
  {
    P = \lop \bp s_1 \sep \lfp \bp s_2 \sep \scopeBps(\ls, ss) \\
      Q = \exists L_1,L_2 \st \left(\begin{array}{l}
        % L_1 - prototype object
        \fullobj_{L_1}(@proto:\lop) \sep L_1 \bp \{(L_2, \js{prototype})\} \sep
          \lop \bp s_1 \cup \{(L_1,@proto)\}  \sep {} \\
        % L_2 - function
        \newfun_{L_2}(\ls, \js{x}, \js{e}, L_1) \sep L_2 \bp \{\} \sep
          \lfp \bp s_2 \cup \{(L_2,@proto)\} \sep {} \\
        % Other bits
        \rv \doteq L_2 \sep
        \scopeBpsUpd(\ls, ss, ss', \{(L_2, @scope)\})
    \end{array}\right)
  }
  {\tr P {\jsfun{x}{e}} Q}
  \vg

  \staterule{(Named Function)}
    {
      P = \lop \bp s_1 \sep \lfp \bp s_2 \sep \scopeBps(\ls, ss)  \\
      Q = \exists L_1, L_2, L_3 \st \left(\begin{array}{l}
        % L_1 - prototype object
        \fullobj_{L_1}(@proto:\lop) \sep L_1 \bp \{(L_2, \js{prototype})\} \sep
          \lop \bp s_1 \cup \{(L_1,@proto)\}  \sep {} \\
        % L_2 - function
        \newfun_{L_2}((L_3:\ls), \js{x}, \js{e}, L_1) \sep
          L_2 \bp \{(L_3,\js{y})\} \sep
          \lfp \bp s_2 \cup \{(L_2,@proto)\} \sep {} \\
        % L_3 - new scope object
        \fullobj_{L_3}(@proto:\nil,\js{y}:L_2) \sep L_3 \bp \{(L_2,@scope)\} \sep {} \\
        % Other bits
        \rv \doteq L_2 \sep
        \scopeBpsUpd(\ls, ss, ss', \{(L_2, @scope)\})
      \end{array}\right)
    }
    {\tr P {\jsfun[y]{x}{e}} Q}
  \vg

  \staterule{(Object)}
    {
      \forall i \in 1..n \st \left(\begin{array}{l}
        P_i = R_i \sep \getValue(Ls_i, Y_i, X_i) \sep
          X_i \bp s_i \\
        \tr {P_{i-1}} {\js{e_i}} {P_i \sep \rv \doteq Y_i} \\
      \end{array}\right) \\
      P_n = R \sep \lop \bp s_{op} \\
      Q = R \sep
      \exists L \st \left(\begin{array}{l}
        \newobj_L(@proto, \js{x_1},...,\js{x_n}) \sep {} \\
        \bigsep_{1 \leq i \leq n} (
          (L, \js{x_i}) \pointsto X_i \sep \bpGen(X_i, L,\js{x_i}, s_i)
        ) \sep {} \\
        (L,@proto) \pointsto \lop \sep \lop \bp s_{op} \cup \{(L,@proto)\} \sep {} \\
        \rv \doteq L \sep L \bp \{\} \\
      \end{array}\right) \\
      \js{x_1} \neq \dots \neq \js{x_n} \qquad \rv \not\in \fv(P_n)
    }
    {\tr {P_0} {\js{\{x_1:e_1, ..., x_n:e_n\}}} Q}
  \vg

  \staterule{(Function Call)}
    {
      \tr P {\js{e_1}} {R_1 \sep \rv \doteq F_1} \\
      R_1 = \left(\begin{array}{l}
          S_1 \sepish \pickThis(F_1, T) \sepish \getValue(Ls_1, F_1, F_2) \sep {} \\
          (F_2, @body) \pointsto \lambda X.\js{e_3} \sep (F_2, @scope) \pointsto
          Ls_2
      \end{array}\right) \\
      \tr {R_1} {\js{e_2}} {R_2 \sep \bpGen(T,s) \sep \bpGen(V_2,s') \sep
        \ls \doteq Ls_3 \sep \rv \doteq V_1} \\
      R_2 = S_2 \sep \getValue(Ls_4, V_1, V_2) \\
      R_3 = R_2 \sep \exists L \st \left(\begin{array}{l}
          (L, X) \pointsto V_2 \sep \bpGen(V_2, L, X, s') \sep {} \\
          (L, @this) \pointsto T \sep \bpGen(T,L,@this,s) \sep
          (L, @proto) \pointsto \nil \sep {} \\
          \defs(X,L,\js{e_3}) \sep
          \newobj_L(@proto,@this,\js{x},\vardecls(X, L, \js{e_3})) \sep {} \\
          L \bp \{\} \sep
          \ls \doteq L:Ls_2
      \end{array}\right) \\
      \tr {R_3} {\js{e_3}} {\exists L \st Q \sep \ls \doteq L:Ls_2} \qquad
      \ls \notin \fv(Q) \cup \fv(R_2)
    }
    {\tr P {\js{e_1(e_2)}} {\exists L \st Q \sep \ls \doteq Ls_3}}
  \vg

  \staterule{(New)}
    {}
    {}
%  \vg

  % SES-specific
  \staterule{(Restricted evaluation)}
    {
      \tr{P}{\js{e_1}}{R_1 \sep \rv \doteq V_1} \\
      R_1 = S_1 \sep \getValue(\_, V_1, V_2) \sep V_2 \dotin \Strings \\
      \parse(V_2) = \js{e_3} \\
      \tr{R_1}{\js{e_2}}{R_2 \sep \rv \doteq V_3 \sep \ls \doteq Ls} \\
      R_2 = \left(\begin{array}{l}
        S_2 \sep \getValue(\_, V_3, V_4) \sep V_4 \dotin \loc \sep V_4 \bp s
        \cup \{(L,@this)\} \sep {} \\
        \exists L \st R_3 \sep
        \newobj_L(@proto,@this,\vardecls(\_,L,\js{e_3})) \sep {} \\
        \obj_L(@this: V_4, @proto: \nil) \sep \defs(\_,L,\js{e_3}) \\
      \end{array}\right) \\
      R_3 = (\ls \doteq L:[V_4]) \\
      \tr{R_2}{\js{e_3}}{\exists L \st Q \sep R_3}
    }
    {\tr{P}{\js{reval(e_1,e_2)}}{\exists L \st Q \sep \ls \doteq Ls}}
  \vg

  \staterule{(Freeze)}
    {
      \tr{P}{\js{e}}{Q \sep \rv \doteq V_1 \sep (V_2, @frozen) \pointsto V_3} \\
      Q = \getValue(Ls,V_1,V_2) \sep S
    }
    {\tr{P}{\js{freeze(e)}}{Q \sep (V_2, @frozen) \pointsto \true \sep \rv \doteq V_2}}
  \vg

  \staterule{(Recursive freeze)}
    {
      \tr{P}{\js{e}}{Q \sep \rv \doteq V_2 \sep \auxDefGet(V_2, \{\})} \\
      Q = \getValue(Ls,V_1,V_2) \sep S
    }
    {\tr{P}{\js{def(e)}}{Q \sep \rv \doteq V_2 \sep \auxDefSet(V_2, \{\}, \true)}}
  \vg

  % Control flow
  \staterule{(Sequence)}
    {\tr P {\js{e_1}} R \quad \tr R {\js{e_2}} Q}
    {\tr P {\js{e_1; e_2}} Q}
  \vg

  \staterule{(If True)}
    {\tr P {\js{e_1}}{S\sep \istrue(\E{V_2})\sep\rv\doteq\E{V_1}} \qquad
     S = R \sep \getValue(Ls,\E{V_1},\E{V_2})\\
     \tr {S}{\js{e_2}} Q}
    {\tr P {\js{if(e_1)\{e_2\}else\{e_3\}}} Q}
  \vg

  \staterule{(If False)}
    {\tr P {\js{e_1}}{S\sep \isfalse(\E{V_2})\sep\rv\doteq\E{V_1}} \qquad
     S = R \sep \getValue(Ls,\E{V_1},\E{V_2})\\
     \tr {S}{\js{e_3}} Q}
    {\tr P {\js{if(e_1)\{e_2\}else\{e_3\}}} Q}
  \vg

  \staterule{(While)}
    {\tr P {\js{e_1}} {S\sep\rv\doteq\E{V_1}} \qquad
    S = R \sep \getValue(Ls,\E{V_1},\E{V_2})\\
     \tr {S \sep \istrue(\E{V_2})} {\js{e_2}} P\\
     Q = S \sep \isfalse(\E{V_2})\sep\rv\doteq\und\qquad
     \rv\not\in \fv(R)}
    {\tr P {\js{while(e_1)\{e_2\}}} Q}
  \vg

  % Standard
  \staterule{(Frame)}
    {\tr P {\js{e}} Q}
    {\tr {P \sep R} {\js{e}} {Q \sep R}}

  \staterule{(Consequence)}
    {\tr {P'} {\js{e}} {Q'} \\
     P \impl P' \quad Q' \impl Q}
    {\tr P {\js{e}} Q}
  \vg

  \staterule{(Elimination)}
    {\tr P {\js{e}} Q}
    {\tr {\exists X \st P} {\js{e}} {\exists X \st Q}}

  \staterule{(Disjunction)}
    {\tr{P_1}{\js{e}}{Q_1} \quad \tr{P_2}{\js{e}}{Q_2}}
    {\tr{P_1 \lor P_2}{\js{e}}{Q_1 \lor Q_2}}
\end{display}

\end{document}
