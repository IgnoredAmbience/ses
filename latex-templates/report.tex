
%%% Local Variables: 
%%% mode: latex
%%% TeX-master: t
%%% End: 

% The comment above tells emacs that this file is the only file in
% this LaTeX project.

\documentclass{article}

% The following lines import a bunch of libraries that are useful if
% you do the kinds of maths I do.

\usepackage{amsmath}
\usepackage{relsize}
\usepackage{amssymb}
\usepackage{amsthm}
\usepackage{stmaryrd}
\usepackage{cancel}
\usepackage{wasysym}
\usepackage[usenames,dvipsnames]{color}
\usepackage{xspace}
\usepackage{url}
\usepackage{verbatim}
\usepackage{proof}

\usepackage{xr-hyper}
\usepackage[colorlinks,urlcolor=MidnightBlue,linkcolor=MidnightBlue,citecolor=MidnightBlue,backref]{hyperref}

% The next line imports an example library I wrote, which defines some
% separation logic macros that I like to use. You should write your
% own files like this, for your own work.
\usepackage{separationlogic}

\begin{document}

\title{An Awesome THING}

\author{Someone}

\maketitle

\begin{abstract}
We're doing a really cool thing.
\end{abstract}

\section{Introduction}

Things are cool. Let's do them.

\subsection{A subsection}

We can have subsections too.

And maths:

\[
\frac{\{P\} \js{code} \{Q\}}
{\{K \sep P\} \js{code} \{K \sep Q\}}
\]
\section{You can have citations}

Here are some things that you might want to cite:

\begin{itemize}
\item O'Hearn, Reynolds and Yang invented separation logic~\cite{DBLP:conf/csl/OHearnRY01}.

\item Gardner, Smith and Zarfaty started writing about DOM, and published in
PODS~\cite{Gardner08DOM}.

\item Smith's PhD thesis~\cite{Smith2010} (all about DOM).

\item Gardner and Wheelhouse's work on segments~\cite{segments}.

\item Gardner, Maffeis and Smith's POPL paper about JavaScript~\cite{gms-popl,gms-popl-proofs}.
\end{itemize}

\bibliographystyle{plain}
\bibliography{bibliography}
\end{document}