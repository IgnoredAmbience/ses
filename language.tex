\documentclass[a4paper]{article}
\usepackage{fullpage}
\usepackage{hyperref}
\usepackage{separationlogic}

\begin{document}

\newcommand{\syntaxline}[3][\pipe]{#1 & \js{#2} & \textit{#3}\\}
\begin{display}{Syntax of SES values and expressions, $v, e$}
  $\begin{array}{rll}
    \syntaxline[v ::=]{$n$}{Number}
    \syntaxline{$s$}{String}
    \syntaxline{undefined}{Undefined}
    \syntaxline{null}{Null}
    \syntaxline[e ::=]{$v$}{Value}
    \syntaxline{$x$}{Identifier}
    \syntaxline{\{$x_1$:$e_1$, $\ldots$, $x_n$:$e_n$\}}{Object construction}
    \syntaxline{$e$; $e$}{Sequence}
    \syntaxline{$e \oplus e$}{Binary operator}
    \syntaxline{if($e$) \{$e$\} else \{$e$\}}{Conditional}
    \syntaxline{while($e$) \{$e$\}}{Looping}
    \syntaxline{var $e$}{Variable declaration}
    \syntaxline{$e$ = $e$}{Assignment}
    \syntaxline{$e$[$e$]}{Computed member access}
    \syntaxline{$e$.$x$}{Member access}
    \syntaxline{$e$($e$)}{Function call}
    \syntaxline{this}{this}
    \syntaxline{function($e$) \{$e$\}}{Anonymous function construction}
    \syntaxline{function $x$($e$) \{$e$\}}{Named function construction}
    \syntaxline{new $e$($e$)}{Object (prototypical) construction}
    \syntaxline{reval($e$, $e$)}{Restricted evaluation}
    \syntaxline{freeze($e$)}{Object freeze}
    \syntaxline{def($e$)}{Transitive \js{freeze}}
  \end{array}$
\end{display}

Describe \js{null} and \js{undefined} here.

An object is a location in the heap, it contains a set of values indexed by
fields. The Object Construction syntax produces a new object with fields named
$x_1, \ldots, x_n$ which map to the values of expressions $e_1, \ldots, e_n$
when evaluated at construction.

The sequencing operator specifies that the expression on the left should be
evaluated, followed by the expresion on the right.

Binary operations are evaluated left-to-right, and includes standard
mathematical and string operations. Formalisation of semantics for all of these
are beyond the scope of this project, but will correspond closely to similar
works on the JavaScript family of languages.

The conditional expression evaluates the first sub-expression, if it evaluates
to \js{true} under a boolean casting operation, the second sub-expression is
evaluated and result returned, otherwise the third sub-expression is evaluated
and result returned.

The looping expression executes the body sub-expression whilst the condition
sub-expression evaluates to true. The most-recently returned value of the body
is returned.

Variable declarations are used in parsing stages to determine free-variables
when constructing a new scope. During evaluation, their subexpression is
evaluated (for side-effects), and \js{undefined} is returned.



\end{document}


