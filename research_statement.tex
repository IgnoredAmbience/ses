\documentclass[a4paper]{article}
\usepackage{fullpage, titling, amsmath, footnote, listings}
\usepackage{hyperref}
\makesavenoteenv{tabular}
\setlength{\droptitle}{-50pt}

\title{PhD Research Statement}
\author{Thomas Wood}

\begin{document}
\maketitle

\section{Research interests}
\begin{itemize}
\item Software verification/reasoning
\item Automated software testing / test generation
\item Programming language theory
\end{itemize}
With a particular focus on:
\begin{itemize}
\item JavaScript / web technologies
\item Security properties of programming languages/software
\end{itemize}

\subsection{JavaScript Background}
Whilst JavaScript was once a small language for mere form validation on web
pages, the growth of the internet in the past decade has pushed the web-browser
and the language forward to be one of the most widely-deployed languages.
Its large deployment has led developers to target the language and the web as a
platform on which to build their software. % More here?

JavaScript was conceived with some unusual semantics for object structure, which
can easily trip even those experienced with
using the language into producing software that is vulnerable to attack under
certain conditions.

Another unusual aspect of JavaScript is that there are multiple competing
implementations of the language, each with a significant share of the end users.
JavaScript is an interpreted language, so the behaviour of a program written in
the language is determined by the language specification, but also by
the particular implementation of the language on which the software is
run. Thanks to the distributed nature of the web, the multiple implementations
of JavaScript are forced to inter-operate and web software developers must
ensure that they pick a subset of the language's features that are fully
supported by their target users.

JavaScript's specification is large and complex, it is common for
different implementations of the language to be incorrect with respect to the
specification, or to interpret the specification in subtly different ways.
Although this is very unlikely to happen for the language's core functionality,
it is a regular occurrence for bugs to be detected in the lesser used `corners'
of the language. This may seem a small or insignificant issue, but
deviations from the specification or differences in behaviour between
implementations is an oft-used attack vector for malicious parties.

The JsCert project, joint between Imperial and INRIA, is in the process of
producing an executable formal semantics for the (current) ECMAScript 5.1
specification of the JavaScript language.
The construction of these semantics is designed to support future tools for
reasoning about and assisting developers to work with JavaScript.

I detail two projects of interest to me that build upon
the work of the JsCert project.

\subsection{Semantically-guided Language Test Suite Generation}
Many popular programming languages are developed without formal specifications,
instead using a mixture of 3 informal methods: by a specification written in English,
through a canonical implementation, or through an executable test suite of code
examples and expected behaviours.

To address the issue of non-conformant implementations, JavaScript has recently
moved to supplement its English specification with a test suite. The test suite
currently comprises of around 10,000 tests, which notionally cover the entire
specification. However, as the suite is largely handwritten, it is incomplete,
either due to incorrect assumptions that the test-writer may have, or simply a
lack of inspiration for all possible failure
cases. Indeed, several bug reports are currently open, noting potentially
poor test coverage for a number of parts of the specification.

Furthermore, since the JavaScript specification is not machine-readable, the
test suite developers have no means of verification of the test suite's
completeness against the specification. One proposed solution is to use the
coverage of implementations as a measure of coverage of the specification, but
this is likely to produce spurious results. The group developing the test suite
recognise that the specification coverage problem is likely to be worthy of
academic interest\footnote{\url{https://bugs.ecmascript.org/show_bug.cgi?id=56}}.

Given an executable specification of the language, it would be possible to run
the tests directly against the specification and hence estimate the proportion
of the language that the tests cover. It would also be possible to break down
the coverage results to discover any portions of the specification
that should be considered for further testing.

Secondly, is it possible to automatically produce new, interesting test cases
for the language from the formal specification, and can these new tests tell us
anything new about the language? % More needed here, too!

% Suitable? {
One notable automated testing method for compilers and interpreters is
\emph{fuzzing}, production of random test data (often syntactically directed by
a grammar, or dictionary), in attempt to trigger a crash condition. The
\texttt{jsfunfuzz} tool, produced by Mozilla, is one particularly successful
fuzzer for the JavaScript language. The purpose of the tool is mainly to find
crash-causing bugs in the interpreter, however it is also used to check
implementation correctness by comparing test results against another implementation.
% }
% Other methods include passing code through a compile/decompile cycle. Also
% used for perf testing implementation improvements.
% For future reference/reading Microsoft SAGE, whitebox fuzzer -
% http://research.microsoft.com/en-us/um/people/pg/public_psfiles/SAGE-in-one-slide.pdf

% Appropriate? (to name my second reference)
One previous effort towards semantic test generation was Allwood and Eisenbach's
``Tickling Java with a Feather'' 2008 paper, in which initial attempts at test
generation for Java were reported. This paper led to REACTOR, a prize-winning
undergraduate 3rd year project of 2008/9. These projects show that the basis for
this approach is reasonable. Both projects left the generation of tests with direct
reference to a formal language specification as an open problem.
% I've failed to find any later references to similar projects
% Note: This was Azalea's UG Group Project

\subsection{Safe Sandboxing of Untrusted JavaScript Code}
An interesting result of the tight integration of JavaScript code and the web is
that code and data can be combined from many different sources. This results in
situations where code written by the developer of a particular webpage relies
upon code from lesser or untrusted third-parties. A common example of this is
the inclusion of advertising banners on webpages, a less common but more
interesting use-case is the execution of user-provided code within a web
application.

JavaScript is a notoriously hard language to secure against untrusted code,
because: a large portion of the language's primitives are mutable; function aliasing
(especially of the \texttt{eval} function, and other primitives) is rampant; and
there is a tendency
to leak pointers to the global object (from which most other objects can be
reached). This has made JavaScript versions prior to the latest specification
all but impossible to secure against malicious code. Recent additions to the
language primitives to properly support sandboxing implementations have
simplified this impossible problem to just a \emph{very difficult} one.

As part of their Caja project, Google have (loosely) specified the Secure
ECMAScript (SES) variant of JavaScript to provide sandboxing primitives that
permit controlled interaction between trusted and untrusted code.

My masters' thesis project is examining the additions required to a JavaScript
program verification logic to additionally support reasoning about the security
implications of interactions across the sandbox boundary.

% Is Mark representative of Google as a whole? :D
% The stuff below also exists, now. Probably needs a tense change.
Google have lofty ambitions for the future of the SES language, producing it as
one ideal view for the future of programming in the browser. When
tightly-coupled with a communication layer, the concept of a distributed object
capability language becomes a possibility. Such a language permits the
implementation of interesting algorithms such as multi-party mutually-suspicious
escrow exchange, or contract agreement (where the contract may be executable
code).

Verification of these types of algorithms is a new, interesting problem, as
their proofs will possibly involve several verification techniques that are
currently rarely used together. For example a process calculus for the
distributed reasoning, combined with higher-order reasoning for contract
agreement, and some means of representing a transferable notion of a user's
`trust' in other parties participating in the transaction.
% tentative? not descriptive enough?
\\

The implementation of SES as a layer on top of JavaScript also poses its own
interesting problems. SES inherently depends upon the JavaScript specification
as its basis, but the JavaScript specification process is also directed to
maintaining the properties required to enable SES to be implemented securely.
These properties are usually very subtle, such as ensuring that the
language-defined data structures and primitives do not inadvertently allow for
pointer leakage or covert channels of communication within the language.

As a result of sharing the intuitions I developed about SES with the rest of the
group, another researcher in the group has discovered a potential bug in the
upcoming ECMAScript specification. This demonstrates that although the
specification committee should now have a good insight into the security
implications of the language's design decisions, errors are still being made as
a result of a poor, informal description of the security properties that ought
to hold for the language. With a formal specification of these properties, we
could prove that the formally specified language definitely does hold these
properties.
% This ignores implementation-specific language extensions

A proof that the SES implementation is correct is currently well beyond the
capabilities of the JavaScript program logic, as the SES implementation reasons
about the particular implementation-specific semantics of the JavaScript engine
on which it is run, and attempts to fix any semantics that it deems unsafe.
Program logics to date assume that there is one correct semantics for a language
under which a program runs, JavaScript poses the unique possibility where a
program may run under similar, but subtly different semantics. Research into
extension of the existing JavaScript program logic to support variable semantics
would be a novel research project.

\section{Relevant experience}
My third-year group project focused on automated regression test generation for
programs written in Ruby, a duck-typed language. The project was able to produce
rudimentary test suites for programs that could detect regressions in simple
programs. The project used mock objects to intercept method calls on objects
passed to programs, in attempt to infer the expected input types.

This project made me aware of the common challenges associated with test
generation, such as the requirement to prune isomorphic tests, an exponential
search space of test programs, and the desire for a `stable' set of produced
tests when modifying the program under test.
\\

My fourth year individual project concerned SES, a secure subset of the
JavaScript language. I developed a program logic capable of proving desired
security properties of programs
written in SES in the object-capability style.

Through this project I gained a great deal of background knowledge on the
JavaScript language semantics and the related security requirements. This
project also gave me grounding in Separation Logic, a key tool in
the construction of program logics used for the verification of software.
\\

Teaching experience through position as 1st Year Undergraduate Lab Helper in
2011-2012.

\section{Funding}
I am seeking a studentship with the department, I am unable to self-fund.
\end{document}


